Auf $V$ führen wir folgende Norm ein 
\begin{definition}
  \[
    \norm{\lambda}_V:=\sup_{\mu\in V'} \frac{\langle \lambda, \mu\rangle}{\norm{\mu}_{V'}}
  \]
  Aus $\norm{\mu}_{V'}^2 = \langle \hat{M}^{-1}\mu, \mu\rangle$ folgt mit direktem Rechnen
  \[
    \norm{\lambda}_V^2 = \langle \hat{M}\lambda,\lambda\rangle \\
    \Rightarrow P^T\hat{M}: V\to V's.p.d.
  \]
\end{definition}

\begin{lemma}
  Für jedes $w\in W$ exisitiert ein eindeutig bestimmtes $z_w\in \ker(S), s.d. B(w+z_w)\in V'$
  Weiterhin gilt: %(Der Sprung von w und zw ist in V')
  \[
    \norm{Bz_w}_Q \leq \norm{Bw}_Q
  \]
  \label{}
\end{lemma}

\begin{proof}
  \[
    B(w+z_w)\in V' \overset{\text{Def. } V'}{\Leftrightarrow} \langle B(w+z_w), Bz\rangle_Q = 0 \quad \forall z\in \ker(S)
  \]

  Diese Galerkinbedinung erlaubt es $z_w\in \ker(S)$ als Lösung des Variationsproblems 
  \begin{equation}
    \label{eqn:stern}
    \tag{$*$}
    \langle B^TQBz_w,z\rangle = -\langle B^TQBw,z\rangle \quad \forall z\in\ker(S)
  \end{equation}
  zu betrachten.\\
  
  Frage: Ist dies denn überhaupt lösbar? \\

  Da $\ker(S) \cap \ker(B) = \{0\}$ und $Q$ s.p.d., ist $B^TQB$ s.p.d. auf $\ker(S)$. Daher exisitiert eine eindeutige Lösung $z_w\in\ker(S)$, die $\norm{B(w+z_w)}_Q^2$ über $z\in\ker(S)$ minimiert. 
  
  Die Orthogonaleigenschaft \eqref{eqn:stern} garantiert auch
  \begin{align*}
    \norm{Bz_w}_Q^2 &= |\langle Bz_w,Bz_w\rangle_Q| = |-\langle Bw, Bz_w\rangle_Q|\\
    &\overset{C.S.}{\leq} \norm{Bw}_Q \norm{Bz_w}_Q\\
    \Rightarrow \norm{Bz_w}_Q \leq \norm{Bw}_Q
  \end{align*}
\end{proof}

CHRISTOPHER
