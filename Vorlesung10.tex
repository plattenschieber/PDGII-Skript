Auf $V$ führen wir folgende Norm ein 
\begin{definition}
  \[
    \norm{\lambda}_V:=\sup_{\mu\in V'} \frac{\langle \lambda, \mu\rangle}{\norm{\mu}_{V'}}
  \]
  Aus $\norm{\mu}_{V'}^2 = \langle \hat{M}^{-1}\mu, \mu\rangle$ folgt mit direktem Rechnen
  \[
    \norm{\lambda}_V^2 = \langle \hat{M}\lambda,\lambda\rangle \\
    \Rightarrow P^T\hat{M}: V\to V's.p.d.
  \]
\end{definition}

\begin{lemma}
  Für jedes $w\in W$ exisitiert ein eindeutig bestimmtes $z_w\in \ker(S), s.d. B(w+z_w)\in V'$
  Weiterhin gilt: %(Der Sprung von w und zw ist in V')
  \[
    \norm{Bz_w}_Q \leq \norm{Bw}_Q
  \]
  \label{}
\end{lemma}

\begin{proof}
  \[
    B(w+z_w)\in V' \overset{\text{Def. } V'}{\Leftrightarrow} \langle B(w+z_w), Bz\rangle_Q = 0 \quad \forall z\in \ker(S)
  \]

  Diese Galerkinbedinung erlaubt es $z_w\in \ker(S)$ als Lösung des Variationsproblems 
  \begin{equation}
    \label{eqn:stern}
    \tag{$*$}
    \langle B^TQBz_w,z\rangle = -\langle B^TQBw,z\rangle \quad \forall z\in\ker(S)
  \end{equation}
  zu betrachten.\\
  
  Frage: Ist dies denn überhaupt lösbar? \\

  Da $\ker(S) \cap \ker(B) = \{0\}$ und $Q$ s.p.d., ist $B^TQB$ s.p.d. auf $\ker(S)$. Daher exisitiert eine eindeutige Lösung $z_w\in\ker(S)$, die $\norm{B(w+z_w)}_Q^2$ über $z\in\ker(S)$ minimiert. 
  
  Die Orthogonaleigenschaft \eqref{eqn:stern} garantiert auch
  \begin{align*}
    \norm{Bz_w}_Q^2 &= |\langle Bz_w,Bz_w\rangle_Q| = |-\langle Bw, Bz_w\rangle_Q|\\
    &\overset{C.S.}{\leq} \norm{Bw}_Q \norm{Bz_w}_Q\\
    \Rightarrow \norm{Bz_w}_Q \leq \norm{Bw}_Q
  \end{align*}
\end{proof}



Wir benötigen folgende Darstellung:
\begin{enumerate}
\item
$\langle F\lambda,\lambda\rangle = \sup_{w\in \text{range}(S)} \frac{\langle \lambda, Bw \rangle }{|w|^2_S}$
\item
$\langle \hat M\lambda,\lambda\rangle = \sup_{w\in W,\, Bw \in V'} \frac{\langle \lambda, Bw \rangle }{|P_D w|^2_S}$
\end{enumerate}
\begin{proof}
\begin{enumerate}
\item
\begin{align*}
\lambda \in V&=\{ \mu \in U : \, \langle \mu,Bz \rangle =0 \, \forall z \in \text{ker}(S) \} \\
 \Rightarrow &  \, B^T\lambda \in\text{ range } (S) , \, S^+ =: S^{-1/2}S^{-1/2} \\
\Rightarrow & \, S^{-1/2}B^T\lambda \text{ ist wohldefiniert}
\end{align*}
Dann gilt:
\begin{align*}
 \langle F\lambda, \lambda \rangle &= \langle S^+ B^T\lambda , B^T \lambda \rangle = \langle S^{-1/2} B^T\lambda , S^{-1/2}B^T \lambda \rangle \\
&= || S^{-1/2} B^T \lambda||^2_{l_2} \\
&= \sup_{v \in \text{range} (S)} \frac{\langle S^{-1/2}B^T\lambda,v\rangle^2}{||v||^2_{l_2}} \quad \text{(Dualnorm)} \\
&= \sup_{w \in \text{range} (S)} \frac{\langle B^T\lambda,w \rangle^2}{|w|^2_S} 
\end{align*}
\end{enumerate}
\begin{align*}
\langle \hat M \lambda , \lambda \rangle &= ||\lambda||^2_V = \sup_{\mu \in V'} \frac{\langle \lambda , \mu \rangle^2}{||\mu ||^2_{V'}} \\
&= \sup_{\mu \in V'} \frac{\langle \lambda , \mu \rangle^2}{\langle {\hat M}^{-1} \mu, \mu \rangle} \\
&= \sup_{w \in W,\, Bw \in V'}	 \frac{\langle \lambda , Bw \rangle^2}{|P_D w|^2_S}
\end{align*}
wobei wir benutzt haben:
\[ \langle {\hat M}^{-1} \mu, \mu \rangle \underbrace{=}_{\mu =Bw} = \langle (BD^{-1}B^T)^{-1} B D^{-1} S \underbrace{D^{-1} B^T (B D^{-1} B^T)^{-1} B}_{=P_D w}w, Bw \rangle \]
\end{proof}
Wir benötigen folgende Abschätzung für $P_D$:
\begin{lemma}(3.2.6)\\
Sei $w \in \text{ range}(S)$, dann gilt:
\[ |P_D w |^2_S \leq C(1+\text{log} \left( \frac{H}{h}\right))^2|w|^2_S \]
Hierbei ist $C$ eine positive Konstante unabhängig von $H,h$ und $S$.
\end{lemma}
\begin{proof}
später in der Vorlesung!
\[ H_i := \text{diam} (\Omega_i),\,  h^{(i)}_j := \text{diam} T^{(i)}_j,\, h_i := \max_j h^{(i)}_j , \, \frac{H}{h} := \max_j \left( \frac{H_i}{h_i} \right) \]
\end{proof}
