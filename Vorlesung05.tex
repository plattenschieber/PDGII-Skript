Christoph

\end{enumerate}
\end{proof}


\begin{algorithmus}[PCG-Verfahren]
  Init: $x^{(0)}\in \R$ (Startvektor)
  $r^{(0)} = b-Ax^{(0)}$ (Startresiduum)
  $y^{(0)}-M^{-1}r^{(0)}$
  $p^{(0)}=y^{(0)}$
  Iteration: $k=0,1,2,\cdots$ (bis Konvergenzkriterium erfüllt ist) 
  \begin{align*}
    \alpha^{(k)}&= \frac{(p^{(k)},p^{(k)})}{p^{(k)},Ap^{(k)}}\\
    x^{(k+1)}&= x^{(k)}+\alpha^{(k)}p^{(k)}\\
    r^{(k+1)}&= r^{(k)}+\alpha^{(k)}Ap^{(k)}\\
    x^{(k+1)}&= M^{-1}r^{(k+1)}\\
    \beta^{(k)}&= \frac{(y^{(k+1)},Ap^{(k)})}{(p^{(k)},Ap^{(k)}}\\
    p^{(k+1)}&= y^{(k+1)}-\beta^{(k)}p^{(k)}
  \end{align*}
\end{algorithmus}

\begin{bemerkung}
  Dieser Algorithmus benötigt nicht die Anwendung von $M^{-1/2}$ auf einen Vektor, sondern nur die von $M^{-1}$. Es handelt sicht \underline{nicht} um das CG-Verfahren angewendet auf $M^{-1}A$
\end{bemerkung}

\chapter{Das Modellproblem}
\section{Die Modellgleichung und Finite Elemente}
Sei $\Omega\in\R^d$ ein Polygon oder Polyeder, $d=2,3$
\begin{equation}
  \begin{aligned}
    -\nabla (u(x) \nabla u) &= f &\quad \inn \Omega\\
    u &= 0 &\quad \auf \partial\Omega_D\\
    \frac{\partial u}{\partial v} &= g &\quad \partial\Omega_N
  \label{}
  \end{aligned}
\end{equation}
wobei $\partial\Omega = \partial\Omega_D \cup \partial\Omega_N$ mit $\partial\Omega_D\cap\partial\Omega_N=\emptyset$ und $|\partial\Omega .. .$FEHLT
Betrachte das zugehörige Variationsproblem:
\begin{problem}
  Finde
  \begin{align*}
    u\in H_0^1(\Omega,\partial\Omega_D) &= \left\{ v\in H^1(\Omega): v|_{\partial\Omega_D} =0 \right\}\\
    \text{s.d. } a(u,v) &= F(v) \quad \forall v\in H_0^1(\Omega,\partial\Omega_D)
  \end{align*}
  mit $a(u,v) := \int_{\Omega} a(x) \nabla u \nabla v \ud x, 
     \quad F(v):=\int_\Omega f\cdot v \ud x + \int_{\partial\Omega_N}g\cdot v \ud \sigma$ 
\end{problem}

Dieses Variationsproblem soll mit konformen finiten Elementen gelöst werden. Wenn nichts anderes gesagt wird, wenden wir lineare Dreiecks- oder Tetraederelemente an. \\
Sei $V^h\subset V := H_0^1(\Omega,\partial\Omega)$ der zugehörige Finite Elemente Raum. 

Christoph

\begin{proof}
  \begin{enumerate}
    \item Zunächst beweisen wir die Aussage auf dem Referenzelement $\hat{T}$ und wenden dann die Transformationsformel für Integrale an.
      Sei $F_T: \hat{T} \to T, \hat{x}\to B\hat{x}+d$ für jedes Element $T\in \tau_h$. $\hat{\Pc}$ sei der lokale Ansatzraum. $\hat{\Pc}$ ist endlichdimensional, daher sind dort alle Nenner äquivalent, somit auch $\norm{\hat{v}}_{l_2}$ und $\norm{\hat{v}}_{L^2(\hat{T})}, : \exists \hat{c_1}, \hat{c_2} > 0$ s.d.
      \begin{equation} %(*)
        \hat{c_1} \norm{\hat{v}}^2_{L^2(\hat{T})} \leq \sum_{i=1}^{M} |\hat{v}(a_i)|^2 \leq \hat{c_2} \norm{\hat{v}}^2_{L^2(\hat{T})}
        \label{}
      \end{equation}
      wobei $\hat{a_i} = $Freiheitsgrade im Referenzelement $\hat{T}$ (Bild vom Ref. Dreieck mit Ecken $\hat{a_1},\hat{a_2},\hat{a_3}$)
      Aus Analysis III, bzw. Numerik PDG I $\Rightarrow \vol(T) = |\det B| \cdot \vol(\hat{T})$

      $(\tau_h)$ ist regulär, also $h_t \leq \sigma \rho_T, \quad \forall T\in \tau_h$
      $\Rightarrow \exists \tilde{c_1}, \tilde{c_2}, \tilde{c_3} >0 $ s.d. 
      %\begin{equation}[**] %(**)
      \[
        \tilde{c_1} h_T^d \leq \tilde{c_3} \rho_T^d \leq |\det(B)| \leq \tilde{c_2} h_T^d \tag{$\star$}
      \]
        \label{ }
      %\end{equation}
      Durch die Transformationsformel erhalten wir für $v \in \Pc_T$
      \begin{align}
        \hat{c_1} \norm{v}_{L^2(T)}^2 \overset{\hat{v}=v^\circ F_T}{=} \hat{c_1} |det (B)| \norm{\hat{v}}_{L^2}(\hat{T})^2 
        \overset{*,**}{\leq} \tilde{c_2} h_T^d \sum_{i=1}^{M} |\hat{v}(\hat{a_i})|^2
        \overset{a_i = F_T(\hat{a}_i)}{\underset{\hat{v}=v \circ F_T}{=}} \tilde{c_2} h_T^d \sum_{i=1}^{M}|v(a_i)|^2 = c_2 h_T^d \sum_{i=1}^{M} |\hat{v}(\hat{a_i})|^2
      Christoph
    \end{align}
  \end{enumerate}
\end{proof}
