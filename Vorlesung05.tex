Weiterhin gilt, da $|T_k(x)|\leq 1$:
\[|q(\lambda)| \leq \frac{1}{\left|T_{k+1} \left( \frac{\lambda_{max}+\lambda_{min}}{\lambda_{max}+\lambda_{min}} \right) \right| }\]
\item[10)] 
\[\Rightarrow \max_{\lambda \in \sigma (A)} |q(\lambda)| \leq \frac{1}{\frac{1}{2}(x+\sqrt{x^2-1})^k} = 2\left( \frac{\sqrt{\kappa} +1}{\sqrt{\kappa}-1} \right)^{-k} \]
wobei $x:=\frac{\lambda_{max}+\lambda_{min}}{\lambda_{max}+\lambda_{min}} = \frac{\kappa +1}{\kappa -1} $.
\end{enumerate}

Schlussendlich folgt aus 1)-10):
\[\norm{e^{(k+1)}}_A \leq 2 \left( \frac{\sqrt{\kappa}-1}{\sqrt{\kappa}+1} \right) \norm{e^{(0)}}_A\]
\end{proof}

\underline{Vorkonditioniertes CG-Verfahren}(preconditioned cg-method, pcg)\\
\underline{Geg.:} $M \in \R^{n \times n}, \text{ s.p.d. } \Rightarrow \, M=M^{1/2}M^{1/2}$\\
Betrachte folgendes System:
\[M^{-1/2}AM^{-1/2}y=M^{-1/2}b, \, x=M^{-1/2}y \]
\[\Leftrightarrow Ax=b \]
Da $\tilde A:= M^{-1/2}AM^{-1/2} $ s.p.d., könne wir das CG-Verfahren \underline{formal} auf $\tilde Ay=\tilde b=: M^{-1/2}b$ anwenden. Die Lösung $x$ ist dann $x=M^{-1/2}y$.\\
Die Konvergenzrate ist bestimmt durhc $\sqrt{\kappa (M^{-1/2}AM^{-1/2})}$ und somit gilt 
\[ \norm{e^{(k+1)}}_{\tilde A} \leq 2 \left( \frac{\sqrt{\kappa(M^{-1} A)}-1}{\sqrt{\kappa(M^{-1}A)+1}}\right)^k \norm{e^{(0)}}_{\tilde A} \]
Durch Anwenden des CG-Verfahrens auf $\tilde A y =\tilde b$ ergibt sich folgender Algorithmus:

\begin{algorithmus}[PCG-Verfahren]
  Init: $x^{(0)}\in \R$ (Startvektor)
  $r^{(0)} = b-Ax^{(0)}$ (Startresiduum)
  $y^{(0)}-M^{-1}r^{(0)}$
  $p^{(0)}=y^{(0)}$
  Iteration: $k=0,1,2,\cdots$ (bis Konvergenzkriterium erfüllt ist) 
  \begin{align*}
    \alpha^{(k)}&= \frac{(p^{(k)},p^{(k)})}{p^{(k)},Ap^{(k)}}\\
    x^{(k+1)}&= x^{(k)}+\alpha^{(k)}p^{(k)}\\
    r^{(k+1)}&= r^{(k)}+\alpha^{(k)}Ap^{(k)}\\
    x^{(k+1)}&= M^{-1}r^{(k+1)}\\
    \beta^{(k)}&= \frac{(y^{(k+1)},Ap^{(k)})}{(p^{(k)},Ap^{(k)}}\\
    p^{(k+1)}&= y^{(k+1)}-\beta^{(k)}p^{(k)}
  \end{align*}
\end{algorithmus}

\begin{bemerkung}
  Dieser Algorithmus benötigt nicht die Anwendung von $M^{-1/2}$ auf einen Vektor, sondern nur die von $M^{-1}$. Es handelt sicht \underline{nicht} um das CG-Verfahren angewendet auf $M^{-1}A$
\end{bemerkung}

\chapter{Das Modellproblem}
\section{Die Modellgleichung und Finite Elemente}
Sei $\Omega\in\R^d$ ein Polygon oder Polyeder, $d=2,3$
\begin{equation}
  \begin{aligned}
    -\nabla (u(x) \nabla u) &= f &\quad \inn \Omega\\
    u &= 0 &\quad \auf \partial\Omega_D\\
    \frac{\partial u}{\partial v} &= g &\quad \partial\Omega_N
  \label{}
  \end{aligned}
\end{equation}
wobei $\partial\Omega = \partial\Omega_D \cup \partial\Omega_N$ mit $\partial\Omega_D\cap\partial\Omega_N=\emptyset$ und $|\partial\Omega .. .$FEHLT
Betrachte das zugehörige Variationsproblem:
\begin{problem}
  Finde
  \begin{align*}
    u\in H_0^1(\Omega,\partial\Omega_D) &= \left\{ v\in H^1(\Omega): v|_{\partial\Omega_D} =0 \right\}\\
    \text{s.d. } a(u,v) &= F(v) \quad \forall v\in H_0^1(\Omega,\partial\Omega_D)
  \end{align*}
  mit $a(u,v) := \int_{\Omega} a(x) \nabla u \nabla v \ud x, 
     \quad F(v):=\int_\Omega f\cdot v \ud x + \int_{\partial\Omega_N}g\cdot v \ud \sigma$ 
\end{problem}

Dieses Variationsproblem soll mit konformen finiten Elementen gelöst werden. Wenn nichts anderes gesagt wird, wenden wir lineare Dreiecks- oder Tetraederelemente an. \\
Sei $V^h\subset V := H_0^1(\Omega,\partial\Omega)$ der zugehörige Finite Elemente Raum. 

Das zugehörige diskrete Modellproblem lautet dann:
\begin{equation}
	\text{Finde } u_h \in V^h, \text{ s.d. } a(u_h,v_h)=F(v_h) \forall \, v_h \in V^h 
\end{equation}
Bekanntermaßen führt dies auf ein lin. GS.:
\[ Ku=b \]
mit $u=(u_1,\dots , u_N)^T , \, u_h = \sum_{i=1}^N u_i \varphi_i $, wobei $(\varphi_i)_{i=1,\dots , N}$ die nodale Basis sei und $K:= (a(\varphi_i,\varphi_j))_{i,j=1,\dots ,N}$ (Steifigkeitsmatrix) und $b:= (F(\varphi_1),\dots, F(\varphi_N))^T$ (Lastvektor)\\

\section{Die Konditionszahl und ihre Abschätzungen}
\underline{Ziel:} 
\begin{enumerate}
\item $\kappa (K)$ nach oben abzuschätzen
\item $\kappa (M)$ nach oben abzuschätzen, wobei $M:= \big((\varphi_i,\varphi_j)\big)_{i,j=1,\dots N}$ die Masseenmatrix ist.
\end{enumerate}•
Auf $V^h$ führen wir zunächst folgende Norm ein:
\[ \norm{v}_{0,h} := \left( \sum_{T \in \tau_h} h^d_T \sum_{a_i \in T} |v(a_i)|^2 \right)^2 \]
$(\tau_h)_h$ sei eine \underline{reguläre} Familie von Triangulierungen von $\Omega$.

\begin{satz}
\begin{enumerate}
\item
 Es existieren Konstanten $c_1,c_2 >0$ unabh. von $h$, s.d. für alle $v \in V^h$ gilt:
\[ c_1 \norm{v}^2_{L^2(\Omega)} \leq \norm{v}^2_{0,h} \leq c_2 \norm{v}^2_{L^2(\Omega)} \]
\item
Es existiert eine von $h$ unabh. Konstante $C>0$, s.d. für alle $v \in V^h$ gilt:
\[ \norm{v}_{H^1(\Omega)} \leq \frac{C}{(\min_{T \in \tau_h})} \norm{v}_{L^2(\Omega)} \]
(\textbf{Inverse Ungleichung})
\end{enumerate}
\end{satz}


\begin{proof}
  \begin{enumerate}
    \item Zunächst beweisen wir die Aussage auf dem Referenzelement $\hat{T}$ und wenden dann die Transformationsformel für Integrale an.
      Sei $F_T: \hat{T} \to T, \hat{x}\to B\hat{x}+d$ für jedes Element $T\in \tau_h$. $\hat{\Pc}$ sei der lokale Ansatzraum. $\hat{\Pc}$ ist endlichdimensional, daher sind dort alle Nenner äquivalent, somit auch $\norm{\hat{v}}_{l_2}$ und $\norm{\hat{v}}_{L^2(\hat{T})}, : \exists \hat{c_1}, \hat{c_2} > 0$ s.d.
      \begin{equation} %(*)
        \hat{c_1} \norm{\hat{v}}^2_{L^2(\hat{T})} \leq \sum_{i=1}^{M} |\hat{v}(a_i)|^2 \leq \hat{c_2} \norm{\hat{v}}^2_{L^2(\hat{T})}
        \label{}
      \end{equation}
      wobei $\hat{a_i} = $Freiheitsgrade im Referenzelement $\hat{T}$ (Bild vom Ref. Dreieck mit Ecken $\hat{a_1},\hat{a_2},\hat{a_3}$)
      Aus Analysis III, bzw. Numerik PDG I $\Rightarrow \vol(T) = |\det B| \cdot \vol(\hat{T})$

      $(\tau_h)$ ist regulär, also $h_t \leq \sigma \rho_T, \quad \forall T\in \tau_h$
      $\Rightarrow \exists \tilde{c_1}, \tilde{c_2}, \tilde{c_3} >0 $ s.d. 
      %\begin{equation}[**] %(**)
      \[
        \tilde{c_1} h_T^d \leq \tilde{c_3} \rho_T^d \leq |\det(B)| \leq \tilde{c_2} h_T^d \tag{$\star$}
      \]
        \label{ }
      %\end{equation}
      Durch die Transformationsformel erhalten wir für $v \in \Pc_T$
      \begin{align}
        \hat{c_1} \norm{v}_{L^2(T)}^2 \overset{\hat{v}=v^\circ F_T}{=} \hat{c_1} |det (B)| \norm{\hat{v}}_{L^2}(\hat{T})^2 
        \overset{*,**}{\leq} \tilde{c_2} h_T^d \sum_{i=1}^{M} |\hat{v}(\hat{a_i})|^2
        \overset{a_i = F_T(\hat{a}_i)}{\underset{\hat{v}=v \circ F_T}{=}} \tilde{c_2} h_T^d \sum_{i=1}^{M}|v(a_i)|^2 = c_2 h_T^d \sum_{i=1}^{M} |\hat{v}(\hat{a_i})|^2
      Christoph letzter Satz
    \end{align}
    
    Durch Summation über die Elemente $T\in \tau_h$ erhalten wir:
    \begin{align*}
      \hat{c_1} \norm{v}_{L^2(\Omega)}^2 &= \hat{c_1} \sum_{T\in \tau_h} \norm{v}_{L^2(T)}^2 \\
                                         &\leq \tilde{c_2} \sum_{T\in\tau_h} h_t^d \sum_{i=1}^{M} |v(a_i)|^2
                                         &= \tilde{c} \norm{v}^2_{0,h}
    \end{align*}
    Analaog erhalten wir:
    \begin{align*}`
      \tilde{c_2}\norm{v}_{0,h}^2 &= \tilde{c_2} \sum_{T\in\tau_h} h_T^d \sum_{i=1}^{M} |v(a_i)|^2
      &\leq \sum_{T\in\tau_h} \frac{\tilde{c_2}}{\tilde{c_1}} \hat{c_2} \norm{v}_{L^2(T)}^2
      &= \frac{\tilde{c_2}}{\tilde{c_1}} \hat{c_2} \norm{v}^2_{L^2(\Omega)}
    \end{align*}
    \[
      \Rightarrow \sqrt{\frac{\hat{c_1}}{\tilde{c_2}}} \norm{v}_{L^2(\Omega)} \leq \norm{v}_{0,h} \leq \sqrt{\frac{\hat{c_2}}{\tilde{c_1}}}\norm{v}_{L^2(\Omega)}
    \]
  \item[Beweis der inversen Ungleichung]
    Transformation auf Referenzelement und dann Transformationsformel benutzen: AnalysisIII/Numerik PDGI
    \begin{equation*}
      |\hat{v}|^2_{\hat{H}(\hat{\Omega})} \leq c \norm{\tilde{B}}^2 |\det \tilde{B}|^{-1} |v|^2_{H^1(\Omega)}
    \end{equation*}
    Setze hier $T:=\hat{\Omega}, \hat{T}:=\Omega, B:=\tilde{B}^{-1}$. Somit gilt dann für $T\in\tau_h$
    \begin{align}
      \norm{v}^2_{H^1(T)} &\underset{*}{\leq} c\norm{B^{-1}}^2 |\det B| \norm{\hat{v}}^2_{H^1(\hat{T})}
      &\underset{*}{\leq} c\norm{B^{-1}}^2 \norm{v}^2_{L^2(\hat{T})}
      &= c\norm{B^{-1}}^2 \norm{v}^2_{L^2(T)}
      &\leq c h^{-2}_T \norm{v}^2_{L^2(T)}
      \label{}
    \end{align}
    Es exisitiert eine Konstante $C_A>0$, unabhängig von $h$, so dass für jeden Knoten in $\tau_h$ die Anzahl der Elemente, zu denen dieser Knoten gehört, durch $C_A$ von oben beschränkt ist. 
  \end{enumerate}
\end{proof}
