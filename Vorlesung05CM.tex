Weiterhin gilt, da $|T_k(x)|\leq 1$:
\[|q(\lambda)| \leq \frac{1}{\left|T_{k+1} \left( \frac{\lambda_{max}+\lambda_{min}}{\lambda_{max}+\lambda_{min}} \right) \right| }\]
\item[10)] 
\[\Rightarrow \max_{\lambda \in \sigma (A)} |q(\lambda)| \leq \frac{1}{\frac{1}{2}(x+\sqrt{x^2-1})^k} = 2\left( \frac{\sqrt{\kappa} +1}{\sqrt{\kappa}-1} \right)^{-k} \]
wobei $x:=\frac{\lambda_{max}+\lambda_{min}}{\lambda_{max}+\lambda_{min}} = \frac{\kappa +1}{\kappa -1} $.

\end{enumerate}
Schlussendlich folgt aus 1)-10):
\[\norm{e^{(k+1)}}_A \leq 2 \left( \frac{\sqrt{\kappa}-1}{\sqrt{\kappa}+1} \right^k \norm{e^{(0)}}_A\]

\end{proof}

\underline{Vorkonditioniertes CG-Verfahren}(preconditioned cg-method, pcg)\\
\underline{Geg.:} $M \in \R^{n \times n}, \text{ s.p.d. } \Rightarrow \, M=M^{1/2}M^{1/2}$\\
Betrachte folgendes System:
\[M^{-1/2}AM^{-1/2}y=M^{-1/2}b, \, x=M^{-1/2}y \]
\[\Leftrightarrow Ax=b \]
Da $\tilde A:= M^{-1/2}AM^{-1/2} $ s.p.d., könne wir das CG-Verfahren \underline{formal} auf $\tilde Ay=\tilde b=: M^{-1/2}b$ anwenden. Die Lösung $x$ ist dann $x=M^{-1/2}y$.\\
Die Konvergenzrate ist bestimmt durhc $\sqrt{\kappa (M^{-1/2}AM^{-1/2})}$ und somit gilt 
\[ \norm{e^{(k+1)}}_{\tilde A} \leq 2 \left( \frac{\sqrt{\kappa(M^{-1} A)}-1}{\sqrt{\kappa(M^{-1}A)+1}}\right)^k \norm{e^{(0)}}_{\tilde A} \]
Durch Anwenden des CG-Verfahrens auf $\tilde A y =\tilde b$ ergibt sich folgender Algorithmus:

%JERO


Das zugehörige diskrete Modellproblem lautet dann:
\begin{equation}
	\text{Finde } u_h \in V^h, \text{ s.d. } a(u_h,v_h)=F(v_h) \forall \, v_h \in V^h 
\end{equation}
Bekanntermaßen führt dies auf ein lin. GS.:
\[ Ku=b \]
mit $u=(u_1,\dots , u_N)^T , \, u_h = \sum_{i=1}^N u_i \varphi_i $, wobei $(\varphi_i)_{i=1,\dots , N}$ die nodale Basis sei und $K:= (a(\varphi_i,\varphi_j))_{i,j=1,\dots ,N}$ (Steifigkeitsmatrix) und $b:= (F(\varphi_1),\dots, F(\varphi_N))^T$ (Lastvektor)\\

\section{Die Konditionszahl und ihre Abschätzungen}
\underline{Ziel:} 
\begin{enumerate}
\item $\kappa (K)$ nach oben abzuschätzen
\item $\kappa (M) nach oben abzuschätzen, wobei $M:= \big((\varphi_i,\varphi_j)\big)_{i,j=1,\dots N}$ die Masseenmatrix ist.
\end{enumerate}•
Auf $V^h$ führen wir zunächst folgende Norm ein:
\[ \norm{v}_{0,h} := \left( \sum_{T \in \tau_h} h^d_T \sum_{a_i \in T} |v(a_i)|^2 \right)^2 \]
$(\tau_h)_h$ sei eine \underline{reguläre} Familie von Triangulierungen von $\Omega$.

\begin{satz}
\begin{enumerate}
\item
 Es existieren Konstanten $c_1,c_2 >0$ unabh. von $h$, s.d. für alle $v \in V^h$ gilt:
\[ c_1 \norm{v}^2_{L^2(\Omega)} \leq \norm{v}^2_{0,h} \leq c_2 \norm{v}^2_{L^2(\Omega)} \]
\item
Es existiert eine von $h$ unabh. Konstante $C>0$, s.d. für alle $v \in V^h$ gilt:
\[ \norm{v}_{H^1(\Omega)} \leq \frac{C}{(\min_{T \in \tau_h})} \norm{v}_{L^2(\Omega)} \]
(\textbf{Inverse Ungleichung})
\end{enumerate}
\end{satz}

%JERO
