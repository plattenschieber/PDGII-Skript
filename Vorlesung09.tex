\[
  K^{(i)}=:K=
  \begin{pmatrix}
    K_{II}& K_{I\Gamma}\\
    K_{\Gamma I}& K_{\Gamma\Gamma}
  \end{pmatrix}
  =
  \begin{pmatrix}
    I& 0\\
    K_{\Gamma I} K_{II}^{-1}& I
  \end{pmatrix}
  \begin{pmatrix}
    K_{II}& 0\\
    0& S_{\Gamma\Gamma}
  \end{pmatrix}
  \begin{pmatrix}
    I& K_{II}^{-1}K_{I\Gamma}\\
    0& I
  \end{pmatrix}\\
  \text{mit } S_{\Gamma\Gamma}=K_{\Gamma\Gamma}-K_{\Gamma I}K_{II}^{-1}K_{I\Gamma}
\]
$K^{(i)}$ ist positiv semidefinit $\underset{\overset{\text{Silvester}}{\text{Trägheitssatz}}}{\Longrightarrow}$ $S_{\Gamma\Gamma}=S_{\Gamma\Gamma}^{(i)}$ auch positiv semidefinit. Dazu mache man sich klar, dass $K_{II}$ positiv definit ist. Somit können wir $K^+$ wie folgt darstellen:

\[
  (*) K^+ = 
  \begin{pmatrix}
    I& K_{II}^{-1}K_{I\Gamma}\\
    0& I
  \end{pmatrix}
  \begin{pmatrix}
    K_{II}^{-1}& 0\\
    0& S_{\Gamma\Gamma}^+
  \end{pmatrix}
  \begin{pmatrix}
    I& 0\\
    -K_{\Gamma I}K_{II}^{-1}& I
  \end{pmatrix}\\
  \text{mit } S_{\Gamma\Gamma}S_{\Gamma\Gamma}^+S_{\Gamma\Gamma}=S_{\Gamma\Gamma}
\]

\begin{align*}
  \Rightarrow KK^+K &= 
  K
  \begin{pmatrix}
    I& -K_{II}^{-1}K_{I\Gamma}\\
    0& I
  \end{pmatrix}
  \begin{pmatrix}
    K_{II}^{-1}& 0\\
    0& S_{\Gamma\Gamma}^+
  \end{pmatrix}
  \begin{pmatrix}
    I& 0\\
    -K_{\Gamma I}K_{II}^{-1}& I
  \end{pmatrix}
  K\\
  &=
  K 
 \begin{pmatrix}
    I& -K_{II}^{-1}K_{I\Gamma}\\
    0& I
  \end{pmatrix}
  \begin{pmatrix}
    K_{II}^{-1}& 0\\
    0& S_{\Gamma\Gamma}^+
  \end{pmatrix}
  \begin{pmatrix}
    K_{II}& 0\\
    0& S_{\Gamma\Gamma}
  \end{pmatrix}
  \begin{pmatrix}
    I& K_{II}^{-1}K_{I\Gamma}\\
    0& I
  \end{pmatrix}\\
  &=
  \begin{pmatrix}
    I& 0\\
    K_{\Gamma I}K_{II}^{-1}& I
  \end{pmatrix}
  \begin{pmatrix}
    K_{II}& 0\\
    0&  \underbrace{S_{\Gamma\Gamma}S_{\Gamma\Gamma}^+S_{\Gamma\Gamma}}_{=S_{\Gamma\Gamma}}
  \end{pmatrix}
  \begin{pmatrix}
    I& K_{II}^{-1}K^{I\Gamma}\\
    0& I
  \end{pmatrix}\\
  &= K
\end{align*}


und somit ist $K^+$, wie in (*) definiert, eine Pseudoinverse von K.

Benutzen wir die Darstellung (*) und $B=(0 \, B_\Gamma)$,so:
\[ F=BF^+B^T =\underbrace{ (0 \, B^T_\Gamma)
 \begin{pmatrix}
I & - K^{-1}_{I\Gamma} K_{I \Gamma} \\
0 & I
\end{pmatrix}
\begin{pmatrix}
K^{-1}_{II} & 0 \\
0 & S^+_{\Gamma \Gamma}
\end{pmatrix}}_{=(0 \, B_\Gamma)}
\underbrace{
\begin{pmatrix}
I & 0 \\
-K_{\Gamma \Gamma }K^{-1}_{II} & I 
\end{pmatrix}
\begin{pmatrix}
0 \\ B^T_\Gamma
\end{pmatrix}}_{=\begin{pmatrix}
0 \\ B^T_\Gamma
\end{pmatrix}}
= B_\Gamma S^+_{\Gamma \Gamma B^T_\Gamma
\]
Analog: $d=BK^+f = BS^+_{\Gamma \Gamma}f_\Gamma$.\\
Sofern keine Verwechslung möglich ist, lassen wir den Index $\Gamma$ fort und schreiben:
\[ F=BS^+B^T,\, d=BS^+f,\, S=\text{diag}(S^{(i)}) \]
\underline{Zur Erinnerung:}
\begin{enumerate}
\item
$W_i := W^h(\partial \Omega_i),\, W := \prod_{i=1}^N W_i $
\item
${\hat M}^{-1} := (BD^{-1}B^T)^{-1} BD^{-1}SD^{-1}B^T (BD^{-1}B^T){-1} $
\item
$P_D := D^{-1}B^T (BD^{-1}B^T)^{-1}B
\end{enumerate}

\begin{lemma}(3.2.1)\\
Für jedes $\mu \in U=\text{range} (B)$ existiert ein $\hat w \in \text{range}(P_D)$,s.d. $\mu = B\hat w$.
\end{lemma}
\begin{proof}
Sei $\mu \in U \, \Rightarrow \, \exist \tilde w \in W,\text{ s.d. } \mu =B\tilde w$.\\
Sei $\hat w := P_D\tilde w \in \text{range}(P_D)$, dann gilt:
\[ B\hat w =BP_D\tilde w \underbrace{=}_{P_D \text{ sprungerh.}} B\tilde w = \mu \]
\end{proof}

Sei $\hat{W}$ der Unterraum von $W$, dessen Elemente auf $\Gamma$ stetig sind. 

\begin{lemma}%3.2.2
  Für alle $w\in W$ gilt: 
  \[
    E_{D}w := w-P_Dw = (I-P_D)w \in \hat{W}
  \]
  d.h. $E_D w$ ist stetig über $\Gamma$ hinweg.
\end{lemma}

\begin{proof}
 Beweis über die Sprungerhaltende Eigenschaft. 
 \[
   BE_Dw=Bw-BP_Dw \underset{P_D\text{ sprungerh.}}{=} Bw-Bw =0
 \]
\end{proof}

\begin{lemma} %3.2.3
  Sei $w\in W$, dann ist $E_Dw(x)$ für $x\in\Gamma^h$ (also ein FE-Knoten auf $\Gamma$) der $D$-gewichtete Mittelwert aller Werte von $w$ in $x$
\end{lemma}

\begin{proof}
  Für ein $w\in W \overset{\text{Lemma 3.2.2}}{\Longrightarrow} E_Dw\in \hat{W}$ (also stetig). Sei $x\in\Gamma^h$ und $e_x\in\hat{W}$ wie folgt definiert:
  \[
    e_x(y) := 
    \begin{cases}
      1& x=y\\
      0& \text{sonst}
    \end{cases},
    \quad B_\Gamma e_x =0
  \]
  Der $D$-gewichtete Mittelwert von $w(x)$ ist wie folgt: (D ist eine Blockdiagonalmatrix; die Blöcke haben entweder x vertreten, oder gar nicht)
  \begin{align*}
    e_x^TDP_Dw = (P_D^TDe_x)^Tw =(B^T(BD^{-1}B^T)^{-1} \underbrace{B\underbrace{D^{-1}D}_{=I}e_x}_{=0})^Tw=0\\
    \Rightarrow e_x^TD_Dw = e_x^TDw=D\text{-gewichteter Mittelwert von W in x}
  \end{align*}
\end{proof}

Es gilt also 
\begin{equation}%3.6
  E_Dw = \sum_{x\in \Gamma^h} \langle e_x, Dw \rangle e_x
\end{equation}

Da $E_D$ nicht von der Wahl von $B$ abhängt, gilt dies auch für $P_D$!\\
Für $P_Dw=:v=(v_i)_{i=1,\ldots,N}\in W$ ergibt sich aus (3.6) folgende Beziehung für $x\in \partial\Omega_{i,h}, i=1,\cdots,N$
\[ %Zentraler Zugang zu dem analytischen Beweis von \ldots {BOXED?}
  (P_Dw(x))^{(i)} = v^{(i)}(x) = \sum_{j\in \mathcal{N}_x} \delta_j^+(x) (w^{(i)}(x)-w^{(j)}(x))
\]

Für unsere Analysis benötigen wir folgende Norm auf $V'=\range(P^T)$. Sei $\mu\in V'$, dann definieren wir
\[%seminorm, weil wir nicht triviale Kerne haben
  \norm{\mu}_{V'}^2 := |D^{-1}B^T(BD^{-1}B^T)^{-1}\mu|_S^2 = \langle\hat{M}^{-1}\mu,\mu \rangle
\]
mit $|u|_S^2:=\langle u,u \rangle_S = u^TSu$ und $S$ Seminorm auf $W$

\begin{lemma}(3.2.4)\\
$|| \cdot ||_{V'}$ definiert eine Norm auf $V'$.
\end{lemma}
\begin{proof}
$||\cdot ||_{V'}$ ist offensichtlich eine Seminorm.\\
\underline{b.z.z.:} $||\mu||_{V'} =0 \Rightarrow \, \mu=0$ \\
Sei $\mu \in V'$ bel. mit $||\mu||_{V'} =0$.\\
$^{Lem. 3.2.1}\Rightarrow \, \exist \hat w\in\text{ range}(P_D)$ mit $\mu =B\hat w;\, \hat w =P_D \hat w $\\
\[ \Rightarrow 0=||\mu||_{V'} =|| B\hat w||_{V'} =| \underbrace{D^{-1}B^T(BD^{-1}B^T)^{-1}B}_{=P_D}\hat w|_S =|P_D \hat w|_S =|\hat w|_S \]
\[\Rightarrow \hat w \in \text{ker}(S) \]
Andererseits: $\mu \in V' = \{ \lambda \in U : \langle \lambda,Bz\rangle_Q=0 \, \forall \, z \in \text{ker}(S) \}$\\
\[ \rightarrow ||\mu||_Q^2 =\langle \mu,\mu \rangle_Q = \langle \mu, B\underbrace{\hat w}_{\in \text{ker}(S)} \rangle_Q =0 \, \Rightarrow \mu=0 \]
\end{proof}
\underline{Bemerkung:} $P{\hat M}^{-1} : V' \to V$ ist sym und pos. def.. \\
\begin{proof}
\begin{itemize}
\item \underline{Sym:}
\[ \mu \in V' \Rightarrow \langle P{\hat M}^{-1}\mu,\mu\rangle = \langle {\hat M}^{-1}P^T\mu,P^T\mu\rangle = \langle P^T\mu,{\hat M}^{-1}P^T\mu\rangle = \langle \mu, P {\hat M}^{-1} \mu \rangle \]
\item
\underline{pos. Def.:}\\
Sei $\lambda \in V'=\text{range}(P^T)$\\
\[\Rightarrow  = \langle P^T {\hat M}^{-1}\lambda,\lambda \rangle\langle {\hat M}^{-1}\lambda,\lambda \rangle = || \lambda ||^2_{V'} > 0 \, \forall \lambda \in V',\, \lambda \neq 0 \]
mit lem. 3.2.4. folgt die Beh.
\end{proof}
