\begin{algorithmus}[Konjugiertes Gradientenverfahren (cg-Verfahren)]
  \[
    \alpha^{(k)} := \frac{(p^{(k)},r^{(k)})}{(p^{(k)},p^{(k)})},\quad
    \beta^{(k)} := \frac{(p^{(k)},r^{(k+1)})_A}{(p^{(k)},p^{(k)})_A}\\
  \]
  \[
    \text{mit } p^{(0)}:=r^{(0)}, r^{(0)}=b-Ax^{(0)}
    k=0,1,2,\ldots\\
    p^{(0)}:= r^{(0)} = b-Ax^{(0)}\\
  \]
  berechne $\alpha^{(k)}$
  \[
    x^{(k+1)}:=x^{(k)}+\alpha^{(k)}p^{(k)}\\ (*)
    r^{(k+1)}:=r^{(k)} -\alpha^{(k)}Ap^{(k)} (**)
  \]
  berechne $\beta^{(k)}$
  \[
    p^{(k+1)}:=r^{(k+1)}-\beta^{(k)}p^{(k)} (***)
  \]
  Es gilt auch (Übungen):
  \[
    \alpha^{(k)} = \frac{(r^{(k)},r^{(k)})}{(Ap^{(k)},p^{(k)})}, \beta^{(k)}=\frac{(r^{(k+1)},r^{(k+1)})}{(r^{(k)},r^{(k)})}
  \]
\end{algorithmus}

\begin{satz}[1.3.3]
  Sei A s.p.d. Dann konvergiert das cg-Verfahren und es gilt folgende Fehlerabschätzung:
  \[
    \|e^{(k)}\|_A \leq 2 \left( \frac{\sqrt{\kappa}-1}{\sqrt{\kappa}+1} \right)^k \|e^{(0)}\|, \quad k=0,1,\ldots
  \]
  mit $\kappa = \kappa(A) = \frac{\lambda_{max}(A)}{\lambda_{min}(A)}$
\end{satz}

\begin{proof}
  Folgt den Argumenten von Quaternioni/Valle Numerical Mathematics for PDE, Springenr, S.48-50 und Quarteroni/Suer/Sater, Numerical Mathematics, S.154
  \begin{itemize}
    \item[1)] (*)$x^{(k+1)}= x^{(k)} + \alpha^{(k)}p^{(k)} \underbrace{=}_{Argument} x^{(0)} + \sum_{i=0}^{k} \alpha^{(i)}p^{(i)}$
    \item[2)] Sei $x=A^{-1}b$ die exakte Lösung , dann folgt aus 1) 
      \[
        (Ap^{(j)}, x^{(k+1)}-x^{(0)}) = \sum_{i=0}^{k} \alpha^{(i)}\underbrace{(Ap^{(j)},p^{(i)})}_{=0, i\neq j} 
        = \alpha^{(j)}(Ap^{(j)},p^{(j)}) \underbrace{=}_{Def. \alpha^{(j)}} (r^{(j)},p^{(j)}) = (b-Ax^{(j)},p^{(j)})
        = (A(x-x^{(j)}),p^{(j)}) 
        = (x-x^{(0)},Ap^{(j)}) + \underbrace{(\underbrace{x^{(0)}-x^{(j)}}_{\underbrace{=}_{1)}-\sum_{i=0}^{j-1}\alpha^{(i)}p^{(i)}},Ap^{j})}_{=0, \text{ da } (p^{(i)},Ap^{(j)})=0, i=0,\ldots,j-1}
        = (x-x^{(0)},Ap^{(j)})
        \Leftrightarrow (x^{(k+1)}-x^{(0)},p^{(j)})_A = (x-x^{(0)},p^{(j)})_A
      \]
    \item[3)] CHRISTOPH
    \item[5] 
      Also gilt $\|e^{(k+1)}\|_A = \min_{p\in \mathcal{P}^*_{k+1}} \|p(A)e^{(0)}\|_A$, wobei $\mathcal{P}^*_{k+1}=\left\{ q\in \mathcal{P}_{k+1} | q(0)=1 \right\}$
    \item[6] A ist s.p.d. $\Rightarrow A=Q^TDQ, Q$ orthogonale Matrix, $D=diag(\lambda_i), \lambda_i$ EV von $A$\\
      $\Rightarrow p(A) = Q^Tp(D)Q$\\
      $\Rightarrow \|e^{(k+1)}\|^2_A = \min_{p\in \mathcal{p}^*_{k+1}} \|p(A)e^{(0)}\|_A^2 
      =  \min_{p\in \mathcal{p}^*_{k+1}} \|Q^Tp(D)Qe^{(0)}\|_A^2 $\\
      $ \min_{p\in \mathcal{p}^*_{k+1}} \left( D\underbrace{QQ^T}_{=I}p(D)Qe^{(0)}\underbrace{QQ^T}_{=I} \underbrace{p(D)Qe^{(0)}}_{=z} \right)$\\
      $  \min_{p\in \mathcal{p}^*_{k+1}} (Dp(D)z,p(D)z) = \sum_{i=1}^{n}\lambda_i z_i^2(p(\lambda_i))^2$\\
      $\leq  \min_{p\in \mathcal{p}^*_{k+1}} \max_{\lambda_i \in \sigma(A)} (p(\lambda_i))^2 \underbrace{\sum_{i=1}^{n}\lambda_i z_i^2}_{=(PQe^{(0)},Qe^{(0)})=\|e^{(0)}\|_A^2}$
      $= \left(  \min_{p\in \mathcal{p}^*_{k+1}}  \max_{\lambda_i \in \sigma(A)} (p(\lambda_i))^2 \right) \|e^{(0)}\|_A^2$\\
      $\Leftrightarrow \frac{\|e^{(k+1)}\|_A}{\|e^{(0)}\|_A} 
      \leq  \min_{p\in \mathcal{p}^*_{k+1}}  \max_{\lambda_i \in \sigma(A)} |p(\lambda)|$
    \item[6] Als nächstes versuchen wir ein Polynom $q\in \mathcal{P}^*_{k+1}$ zu konstruieren, s.d. $|q(\lambda)|$ möglichst klein ist für $\lambda\in [\lambda_{min}, \lambda_{max}]$. Dazu führen wir die \underline{Tschebyscheff-Polynome} ein (vgl. Martin Haute-Bourgois, Grundlagen d. Num. Math. und WissRech, Vieweg-Teubner, 3. Auflage S.284). 
      Definition  $T_k(x) := \cos(k arccos(x)), \quad k\geq 0, x\in [-1,1]$ Tschebyscheff-Polynome k-ter Ordnung.
  $T_0(x)=1, T_1(x)=x, t:=arccos(x)$
  Behauptung: $T_{k-1}(x) + T_{k+1}(x) = \cos((k-1)t) + \cos ((k+1)t)$
  CHRISTOPH das Ende dieses Punktes plus Punkt 8
  \end{itemize}
\end{proof}


