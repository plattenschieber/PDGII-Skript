
\begin{enumerate}
  \item[1) Katenbeiträge:] Für jede Kante $\varepsilon^{ij}$ gilt mit $(*')$ und $\delta_j^+(x)$ konstant auf $\varepsilon^{ij}$
    \[
      I^h(\theta_{\varepsilon^{ij}} \nu^{(i)} = I^h(\theta_{\varepsilon^{ij}} \delta_j^+(w^{(i)}-w^{(j)}))
    \]
    Wir wollen nun folgende Abschätzung beweisen:
    \begin{align*}
      \rho_i|\mathcal{H}(I^h(\theta_{\varepsilon^{ij}}\delta_j^+(w^{(i)}-w^{(j)})))|^2_{H^1(\Omega_i)}
      &\leq C(1+\log(\frac{H_i}{h_i}))^2 \rho_i \left\{ |\mathcal{H}(w^{(i)})|^2_{H^1(\Omega_i)} + \frac{1}{H_i^2} \norm{\mathcal{H} (w^{(i)})}_{L62(\Omega_i)} \right\}  + 
      C(1+\log(\frac{H_j}{h_j}))^2 \rho_j \left\{ |\mathcal{H}(w^{(j)})|^2_{H^1(\Omega_i)} + \frac{1}{H_j^2} TAFEL \right\}
      \delta_i |\mathcal{H}(I^h(\theta_{\varepsilon^{ij}} \nu^{(i)}))|_{H^1(\Omega_i)}^2 
      = \delta_i |\mathcal{H}(I^h(\theta_{\varepsilon^{ij}}\delta_j^+(w^{(i)}-w^{(j)})))|_{H^{1(\Omega_i)}}^2
      &\underset{\delta_j^+ konst auf \varepsilon^{ij}, Lemma 3.2.13}{\leq} \min(\delta_i,\delta_j) |\mathcal{H} (I^h(\theta_{\varepsilon^{ij}}(w^{(i)}-w^{(j)})))|_{H^{1}(\Omega_i)}^2
      &\underset{Dreiecks-Ungl.}{\leq} 2 \min(\delta_i,\delta_j) \left( \underbrace{|{\mathcal{H}(I^h(\theta_{\varepsilon^{ij}}w^(i)))}|_{H^1(\Omega_i)}^2}_{=(I)}
      + \underbrace{|{\mathcal{H}(I^h(\theta_{\varepsilon^{ij}}w^{(j)}))}|_{H^1(\Omega_i)}^2}_{=(II)} \right)
    \end{align*}
    Zu (I): 
    \[
      |\mathcal{H} (I^h(\theta_{_\varepsilon^{ij}})w^{(i)})|_{H^1(\Omega_i)}^2 \underset{Lemma 3.2.1.2}{\leq} C(1+\log(\frac{H_i}{h_i}))^2 
      \left\{ |\mathcal{H}(w^{(i)})|_{H^1(\Omega_i)}^2 + \frac{1}{H_i^2} \norm{\mathcal{H}(w^{(i)})}_{L^2(\Omega_i)}^2 \right\}
    \]
    Zu (II): 
    \[
      |\mathcal{H}(I^h(\theta_{\varepsilon^{ij}}w^{(j)}))|_{H^1(\Omega_i)}^2 \leq |E_{ji}(\mathcal{H}(I^h(\theta_{\varepsilon^{ij}}w^{(j)})))|_{H^1(\Omega_i)}^2
      \underset{Fortsetzungssatz}{\leq} C|\mathcal{H}(I^h(\theta_{\varepsilon^{ij}}w^{(j)}))|_{H^1(\Omega_j)}^2 \underset{Kantenlemma}{leq} C(1+\log(\frac{H_j}{h_j})^2) \left\{ |\mathcal{H}(w^{(j)})|_{H^1(\Omega_i)}^2 + \frac{1}{H_j^2} \norm{\mathcal{H}(w^{(j)})}_{L^2(\Omega_j)}^2\right\}

    \]

    Benutzen wir nun Lemma 3.2.11 und $w^{(i)}\in range(S^{(i)}), w^{(j)}\in \range(S^{(j)})$, so erhalten wir:
    \[
      \delta_i |\mathcal{H}(I^h(\theta_{\varepsilon^{ij}})\nu^{(i)})|_{H^!(\Omega_i)}^2
      \leq C(1+\log(/\frac{H_i}{h_i}))^2 |\mathcal{H}(w^{(i)})|_{H^1(\Omega_i)}^2 + C(1+\log(\frac{H_j}{h_j}))^2 |\mathcal{H}(w^{(j)})|_{H^1(\Omega_i)}^2
    \]
\end{enumerate}

begin{proof}(zu Lemma 3.2.6)\\
\underline{zu zeigen:} \[ w \in \range (S) \, \Rightarrow \, |P_Dw|^2_S \leq C(1+ \log (H/h))^2|w|^2_S \]
Sei $w \in \range (S)$, dann $w=(w^{(i)})_{i=1,\dots,N} \, \Rightarrow w^{(i)}\in \range (S)$:\\
Sei $v := P_Dw = (v^{(i)})_{i=1,\dots,N}$. Also : \[v^{(i)}= (P_Dw)^{(i)\]
Dann gilt : 
\begin{align*}
|P_Dw|^2_S &= \sum_{i=1}^N |(P_Dw)^{(i)}|^2_{S^{(i)}} = \sum_{i=1}^N |v^{(i)}|^2_{S^{(i)}} \\
&= \sum_{i=1}^N |\mathcal{H} (v^{(i)})|^2_{K^{(i)}} \underset{=}{\text{Modellproblem}} \sum_i\rho_i |\mathcal{H} (v^{(i)})|^2_{H^1(\Omega_i)}
\end{align*}
\underline{Analog:}
\[|w|^2_S = \sum_{i=1}^N |w^{(i)}|^2_{S^{(i)}} = \sum_{i=1}^N \rho_i |\mathcal{H} (w^{(i)})|^2_{H^1(\Omega_i)} \]
Um die Abschätzung 
\[ |P_Dw|^2_S \leq C(1+\log (H/h))^2 |w|^2_S \]
zu beweisen, genügt es offenbar
\[ \boxed{\rho_i |\mathcal{H} (v^{(i)})|^2_{H^1(\Omega_i)} \leq C(1+\log(H/h))^2 \sum_{j \in I_i} \rho_j |\mathcal{H} (w^{(j)})|^2_{H^1(\Omega_j)}}\]
zu zeigen, wobei $I_i := \left\{ k\in \{1,\dots,N\}: \, \bar{\Omega}_i \cap \bar{\Omega}_k \neq \phi \right\}$.\\
\[(*) \, \begin{aligned}
v^{(i)} &= (P_Dw(x))^{(i)} = \sum_{j \in \mathcal{N}_x} \rho_j \mu^{\dagger}_j (w^{(i)}(x) - w^{(j)}(x)) \\
&= w^{(i)}(x) - \sum_{j \in \mathcal{N}_x}\rho_j \mu^{\dagger}_j w^{(j)}(x) = \left( (w-E_Dw)(x) \right)^{(i)}
\end{aligned} \]
(Dies folgt aus Lemmata 3.2.2 und 3.2.3)\\
Hierbei ist:
\[ mu^{\dagger}_j := \left\{ \begin{aligned} \mu^{-1}_j (x) &: \mu_j(x) \neq 0 \\ 0 &: \text{sonst} \end{aligned} \right, \quad \mu_j := \left\{ \begin{aligned} \sum_{j \in \mathcal{N}_x} \rho_j(x) &: x \in \partial\Omega_{i,h} \cap \partial \Omega_{j,h} \\ \rho_i (x) &: x \in \partial \Omega_{i,h}\cap (\partial\Omega_h\setminus \Gamma_h ) \\ 0 &: \text{ sonst} \end{aligned} \right, 
\]
Wir haben angewendet, dass $\sum_{i=1}^N \rho_i(x)\mu^{\dagger}_i(x) =1\, \forall \, x\in \Gamma_h \cup \partial\Omega_h$. \\
Es gilt: $\delta^{dagger}_i =\rho_i(x)\mu^{\dagger}_i(x) $ \\
(*) reduziert sich damit auf
\[ (*') :\quad v^{(i)} = \sum_{j \in \mathcal{N}_x} \delta^{\dagger}_j(x) (w^{(i)}(x)-w^{(j)}(x)) \quad x\in \partial \Omega_{i,h} \]
Part. der Eins $\{\Theta_{\epsilon^{1,j}},\dots , \Theta_{\epsilon^{?,j}}\}\, \Rightarrow  \, v^{(i)} = \sum_{\epsilon^{ij} \subset \partial\Omega_i} I^h (\Theta_{\epsilon^{ij}}v^{(i)})(x) + \sum_{\epsilon \in \partial\Omega_j} I^h (\Theta_{\epsilon^{ij}}v^{(i)})(x) $. 

