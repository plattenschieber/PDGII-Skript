\documentclass[12pt]{scrbook}

\usepackage{geometry}
\geometry{a4paper,left=2.5cm,right=2.5cm, top=2cm, bottom=3cm, marginpar=2cm} 

%% Sonderzeichen
\usepackage[ngerman]{babel}
\usepackage[utf8]{inputenc}
\usepackage[T1]{fontenc}
\usepackage{amsmath,amssymb,amsfonts,stmaryrd,mathtools,amsthm,esint}
\usepackage{algpseudocode}
\usepackage{dsfont} % bb für Zahlen
%\usepackage{MnSymbol} %adobe minion font

%% minted
%\usepackage{minted}
%\newminted{cpp}{linenos}

%\usepackage{marginnote}
%\renewcommand*{\marginfont}{\footnotesize} 

\usepackage{graphicx}
%\graphicspath{{Bilder/}}
%\usepackage{tikz}
%\usepackage[all]{xy}
%\usetikzlibrary{arrows,calc,shadows,patterns,through,backgrounds}
%\pgfdeclarepatternformonly{mynwlines}{%
%\pgfqpoint{-1pt}{-1pt}}{\pgfqpoint{8pt}{8pt}}{\pgfqpoint{6pt}{6pt}}%
%{
%  \pgfsetlinewidth{0.4pt}
%  \pgfpathmoveto{\pgfqpoint{0pt}{3pt}}
%  \pgfpathlineto{\pgfqpoint{6.1pt}{-3.1pt}}
%  \pgfusepath{stroke}
%}

\usepackage{hyperref}
\usepackage{makeidx}
\usepackage{listings}
\usepackage{enumerate}

%% Counters
%\renewcommand\theequation{\thesection.\arabic{equation}}
%\AtBeginSection[]{ \setcounter{section}{0} }
\numberwithin{equation}{section}


\newcommand{\N}{\mathbb{N}}
\newcommand{\Z}{\mathbb{Z}}
\newcommand{\R}{\mathbb{R}}
\newcommand{\C}{\mathbb{C}}
\newcommand{\F}{\mathbb{F}}
\newcommand\K{\mathbb{K}}
\renewcommand{\c}[1]{\mathcal{#1}} % beliebiger Buchstabe kann mit \op in cal geschrieben werden 
\renewcommand\P{\mathcal{P}} % tolles P, anstatt eines Zeilenumbruchzeichens 
\newcommand\Ac{\mathcal{A}}
\newcommand\Pc{\mathcal{P}}
\newcommand\Lc{\mathcal{L}}
\newcommand\Kc{\mathcal{K}}
\newcommand\Mc{\mathcal{M}}
\newcommand\Tc{\mathcal{T}}
\newcommand\LC{\mathcal{L}}
\newcommand\ii{\mathrm{i}} % schlichtes i, ungeschwungen
\newcommand{\ud}{\,\textnormal{d}} % aufrechtes d fürs Integral
\newcommand{\floor}[1]{\left\lfloor #1 \right\rfloor}
\newcommand{\ceil}[1]{\left\lceil #1 \right\rceil}
%\def\default{}
\renewcommand{\emph}[2][\default]{\ifx#1\default\textbf{#2}\index{#2}\else\textbf{#2}\index{#1}\fi}
%\newcommand{\ocirc}{\, \raisebox{1pt}{\footnotesize\textcircled{$\circ $}}\,}
%\newcommand{\ostar}{\, \raisebox{1pt}{\footnotesize\textcircled{$*$}}\,}

\newcommand{\bs}[1]{\boldsymbol{#1}} % fett gedruckt
\newcommand{\entspr}{\mathop{\widehat{=}}} % entspricht Zeichen
\newcommand{\eps}{\varepsilon} % schönes epsilon 
\def\pre{\textnormal{pre}}
\def\suc{\textnormal{suc}}

% coole Abkürzungen für die Norm und Doppelnorm
\newcommand{\norm}[1]{\|#1\|}
\newcommand{\znorm}[1]{|\!|\!|#1|\!|\!|}

% ein paar Befehle in Normalschrift, statt kursiv
\newcommand\dist{\textnormal{dist}}
\newcommand\im{\textnormal{Im}\,}
\newcommand\re{\textnormal{Re}\,}
\newcommand\ran{\textnormal{Ran}\,}
\renewcommand\ker{\textnormal{Ker}\,}
\newcommand\range{\textnormal{range}\,}
\newcommand\var{\textnormal{Var}}
\newcommand\cov{\textnormal{Cov}}
\newcommand\ggt{\textnormal{ggT}}
\newcommand\periode{\textnormal{Periode}}
\newcommand{\trace}{\textnormal{trace}\,}
\newcommand{\sign}{\textnormal{sign}\,}
\newcommand{\cond}{\textnormal{cond}}
\newcommand{\spanf}{\textnormal{span}\,}
\newcommand{\spann}{\textnormal{span}}
\newcommand{\grad}{\textnormal{grad}\,}
\newcommand{\sgn}{\textnormal{sgn}}
\newcommand{\rd}{\textnormal{rd}}
\newcommand{\diag}{\textnormal{diag}}
\newcommand{\blockdiag}{\textnormal{blockdiag}}
\newcommand{\supp}{\textnormal{supp}}
\newcommand{\fie}{\varphi}
\newcommand{\eins}{\mathds{1}}
\newcommand{\diam}{\mathrm{diam}}
\newcommand{\const}{\mathrm{const}}
\newcommand{\divt}{\mathrm{div}}
\newcommand{\vol}{\mathrm{vol}}
\newcommand{\inn}{\textnormal{ in }}
\newcommand{\auf}{\textnormal{ auf }}
\renewcommand{\i}{\textnormal{i}}
\def\dtilde{\stackrel{\approx}}
\def\dabs{\phantom{a}\quad}


\newcommand{\tn}{\ensuremath{| \! | \! |}}
\newcommand{\operateson}{\rcirclearrowright}
\usepackage{framed,color}                          % Farbe
\setlength{\fboxsep}{0.2cm}
\setlength{\fboxrule}{1pt}
\definecolor{shadecolor}{rgb}{0.91,0.91,1}      % fuer shaded-Umgebungen

\newtheoremstyle{note}% name
  {1ex}  % Space above
  {1ex}  % Space below
  {\sl}  % Body font
  {}     % Indent amount (empty = no indent, \parindent = para indent)
  {\bfseries}  % Thm head font
  {.}    % Punctuation after thm head
  {.5em} % Space after thm head: " " = normal interword space;
         % \newline = linebreak
  {\thmname{#1}\thmnumber{ #2}\bf\thmnote{(#3)}}     % Thm head spec (can be left empty, meaning `normal')

\newtheoremstyle{remark}% name
  {1ex}  % Space above
  {1ex}  % Space below
  {}     % Body font
  {}     % Indent amount (empty = no indent, \parindent = para indent)
  {\bfseries}  % Thm head font
  {.}    % Punctuation after thm head
  {.5em} % Space after thm head: " " = normal interword space;
         % \newline = linebreak
  {\thmname{#1}\thmnumber{ #2}\bf\thmnote{(#3)}}     % Thm head spec (can be left empty, meaning `normal')

\theoremstyle{note}
\newtheorem{amssatz}{Satz}[section]
\newtheorem{amslemma}[amssatz]{Lemma}
\newtheorem{amsproblem}[amssatz]{Problem}
\newtheorem{amsdefi}[amssatz]{Definition}
\theoremstyle{remark}
\newtheorem{amsbsp}[amssatz]{Beispiel}
\newtheorem*{amsbsp*}{Beispiel}
\newtheorem{amsalgo}[amssatz]{Algorithmus}
\newtheorem*{amsalgo*}{Algorithmus}
\newtheorem{amskorollar}[amssatz]{Korollar}
\newtheorem*{bemerkung*}{Bemerkung}
\newtheorem{bemerkung}[amssatz]{Bemerkung}

\newenvironment{satz}[1][]{\begin{amssatz}[#1]\begin{shaded}}
                          {\end{shaded}\end{amssatz}}
\newenvironment{lemma}[1][]{\begin{amslemma}[#1]\begin{shaded}}
                           {\end{shaded}\end{amslemma}}
\newenvironment{problem}[1][]{\begin{amsproblem}[#1]\begin{shaded}}
                           {\end{shaded}\end{amsproblem}}
\newenvironment{korollar}[1][]{\begin{amskorollar}[#1]\begin{shaded}}
                           {\end{shaded}\end{amskorollar}}
\newenvironment{algorithmus}[1][]{\begin{amsalgo}[#1]\begin{shaded}}
                                 {\end{shaded}\end{amsalgo}}
\newenvironment{algorithmus*}[1][]{\begin{amsalgo*}[#1]\begin{shaded}}
                                 {\end{shaded}\end{amsalgo*}}
\newenvironment{definition}[1][]{\begin{amsdefi}[#1]\begin{shaded}}
                                {\end{shaded}\end{amsdefi}}
\newenvironment{beispiel}[1][]{\begin{amsbsp}[#1]\begin{framed}}
                               {\end{framed}\end{amsbsp}}
\newenvironment{beispiel*}[1][]{\begin{amsbsp*}[#1]}
                               {\end{amsbsp*}}




\hypersetup{
    bookmarks=true,         % show bookmarks bar?
    unicode=false,          % non-Latin characters in Acrobat’s bookmarks
    pdftoolbar=true,        % show Acrobat’s toolbar?
    pdfmenubar=true,        % show Acrobat’s menu?
    pdffitwindow=true,      % page fit to window when opened
    pdftitle={Funktionalanalysis},    % title
    pdfauthor={Jeronim Morina},     % author
    pdfsubject={Skript zur Vorlesung im SS 2013},   % subject of the document
    pdfnewwindow=true,      % links in new window
    pdfkeywords={},         % list of keywords
    colorlinks=true,        % false: boxed links; true: colored links
    linkcolor=red,          % color of internal links
    citecolor=green,        % color of links to bibliography
    filecolor=magenta,      % color of file links
    urlcolor=cyan           % color of external links
}
\makeindex
\begin{document}
\begin{titlepage}
\vspace*{\stretch{1}}
\flushright{\Huge\bfseries Numerik partieller Differentialgleichungen II}
\noindent\rule[-1ex]{\textwidth}{4pt}\\[5pt]
\Large Vorlesungsskript WS 2013/2014 \\[3cm]
{\Large \bf Vorlesungsmitschrift von Christopher Max und Jeronim Morina}\\[5mm]
\hfill \textnormal{\date{}}
\vspace{\stretch{1}}
\end{titlepage}

\cleardoublepage
\thispagestyle{empty}
\pagenumbering{roman} \setcounter{page}{1}

\tableofcontents

\chapter*{Vorwort}
Dieses Dokument enthält die Mitschrift von Christopher Max und Jeronim Morina zur Vorlesung ``Numerik partieller Differentialgleichungen II'' im Wintersemester 2014 bei Professor Klawonn. Wir können dem Leser weder Vollständigkeit noch Fehlerfreiheit (von dieser sind wir überzeugt, dass sie definitiv nicht gegeben ist) versprechen. Wir sind jedoch für Verbesserungsvorschläge dankbar, diese können an \href{mailto:morina@jeronim.de}{morina@jeronim.de} geschickt werden.\\


\hfill Bonn, \today

\chapter*{Einleitung}
\noindent Ziele:
\begin{itemize}
  \item Schnellere Algorithmen für Parallelrechner entwickeln
\end{itemize}

\noindent Literaturangaben:
\begin{itemize}
\item 
\end{itemize}


\cleardoublepage
\pagenumbering{arabic} \setcounter{page}{1}

 \chapter[Iterationsverfahren für lineare Gleichungssysteme]{Iterationsverfahren für (große) lineare Gleichungssysteme}
Gegeben: $Ax=b$, $A\in \R^{n\times n}$, $A$ invertierbar, $x,b \in \R^n$ mit $x=A^{-1}b$
Ziel: Konstruktion eines iterativen Verfahren zur Lösung von * der Form:
\begin{equation}
  x^{(k+1)} := \varphi(x^{(k)}),\quad k\in \N, x^{(0)} \in \R^n
  \label{}
\end{equation}
Iterationsfunktion $\varphi: \R^n \to \R^n$ gegeben.

\section{Das Richardson-Verfahren}
Ansatz: Sei $M\in \R^{n\times n}$ invertierbar. Wir betrachten das folgende Splitting:
\begin{equation}
  Mx + \underbrace{(A-M)x}_{=N}= b
  \label{}
\end{equation}
und setzen 
\begin{equation}
  \begin{split}
    Mx^{(k+1)} + Ax^{(k)}     &= b \quad FEHLER? WO IST N?\\
    \Leftrightarrow x^{(k+1)} &= x^{(k)} + M^{-1}(b-Ax^{(k)})\\
    &= x^{(k)} + M^{-1}r^{(k)}
  \end{split}
  \label{}
\end{equation}
mit $ r^{(k)} := b-Ax^{(k)}$ das $k$-te Residuum für $k=0,1,2,\dots$

 Dieses Iterationsverfahren lässt sich durch Einführen eines Relaxationsparameters $\alpha \in \R$ noch verallgemeinern. Sei dazu $x^{(0)}\in \R^n$ gegeben. 

\begin{definition}[Richardson-Verfahren]
   \label{richardson}
   \[
     x^{(k+1)} := x^{(k)} + \alpha M^{-1} r^{(k)} \qquad k=0,1,2,\dots
   \]
   heißt (stationäres) Richardson-Verfahren. Die dazugehörige Iterationsvorschrift lautet
   \[
     \varphi(x) = Bx + c\\
   \]
   wobei $B=I-\alpha M^{-1}A$ und $C=\alpha M^{-1}b$
\end{definition}


Für die Konvergenz kennen wir aus Numerik I folgenden Satz:

\begin{satz}
  \label{itkonvergenz}
  Das Iterationsverfahren \eqref{richardson} konvergiert genau dann, wenn 
  \begin{equation}
    \rho (B) < 1 
    \label{}
  \end{equation}
  wobei $\rho(B) := \max_{\lambda \in \sigma(B)} |\lambda|$ und $\sigma(B) = \left\{ \lambda\in \C: \lambda \text{ ist EW von B} \right\}$
\end{satz}

Hinreichende Konvergenzbedingung ist $\|B\| <1$ wobei $\|\cdot \|$ eine beliebige Matrixnorm ist, die durch eine Vektornorm induziert wird.
Für den Fehler im $k$-ten Schritt $e^{(k)} = x-x^{(k)}$ gilt $\|e^{(k)}\| \leq \eta^k e^{(k)}$ mit $\eta := \|B\|$ STIMMT DIE UNGLEICHUNG?

\begin{satz}
  Das stationäre Richardson-Verfahren \eqref{richardson} mit $\alpha \neq 0$ konvergiert genau dann, wenn 
  \begin{equation}
    2* \frac{Re(\lambda_i)}{\alpha|\lambda_i|^2} > 1 \qquad \forall i=1,\dots, n
    \label{}
  \end{equation}
  wobei $\lambda_i \in \sigma(M^{-1}A)$
\end{satz}

\begin{proof}
  Wende \eqref{itkonvergenz} auf $B=I-\alpha M^{-1}A$ an. Sei $\mu_i \in \sigma(B)$. Dann gilt
  \begin{equation*}
    \begin{split}
      \mu_i = 1-\alpha \lambda_i, \quad \forall \lambda_i \in \sigma(M^{-1}A)\\
      \rho(B) < 1 \Leftrightarrow |1-\alpha\lambda_i| <1, \quad \forall \lambda_i \in \sigma(M^{-1}A)\\
      \Leftrightarrow -2 \alpha Re(\lambda_i) + \alpha^2 |\lambda_i|^2 <0
    \end{split}
  \end{equation*}
  SCHÖNERE AUSRICHTUNG MÖGLICH?
\end{proof}

\begin{satz}
  $M^{-1}A$ habe nur positive reelle Eigenwerte $\lambda_i$, $i=1,\dots, n$ (Invertierbarkeit nicht zwingend erforderlich), die wie folgt geordnet seien:
  \begin{equation*}
    0< \lambda_n \leq \lambda_{n-1} \leq \dots \leq \lambda_1
  \end{equation*}

  \begin{enumerate}[1)]
    \item  Das stationäre Richardson-Verfahren ist genau dann konvergent, wenn 
             \begin{equation*}
                0 < \alpha < \frac{2}{\lambda_1}
              \end{equation*}

    \item Der Spektralradius $\rho(B)$ ist minimal (und damit die Konvergenz am schnellsten), wenn 
      \begin{equation*}
        \alpha = \alpha_{opt} := \frac{2}{\lambda_1+\lambda_n}
      \end{equation*}
      Dann gilt
      \begin{equation*}
        \rho_{opt} = \min_{\alpha} \left( \rho(B_\alpha) \right) 
                   = \frac{\lambda_1-\lambda_n}{\lambda_1+\lambda_n} 
                   = \frac{\kappa - 1}{\kappa+1}= 1 - \frac{2}{\kappa+1}
      \end{equation*}
      mit der Konditionszahl $\kappa = \kappa(M^{-1}A) = \frac{\lambda_1}{\lambda_n}$\\
    FEHLT HIER NOCH ETWAS?
  \end{enumerate}
 \end{satz}

 \begin{proof}
   \begin{equation*}
     \mu_i \in \sigma(B) \Leftrightarrow \mu_i = 1-\alpha \lambda_i, \quad \lambda_i\in \sigma(M^{-1}A)
   \end{equation*}
   Aus Satz 1.1.1:[FÄNGT DIE NUMERIERUNG FÜR SATZ UND DEFINITION UNABHÄNGIG BEI DER SECTION AN?]\\
   Das Richardson-Verfahren konvergiert
   \begin{equation*} 
     \Leftrightarrow |\mu_i| <1, \quad \forall i = 1, \dots, n
     \Leftrightarrow |1-\alpha\lambda_i| < 1\\
     \Leftrightarrow -1 < 1 - \alpha \lambda_i < 1 
     \Leftrightarrow 0 < \alpha < \frac{2}{\lambda_i} 
     \Leftrightarrow 0 < \alpha < \frac{2}{\lambda_1} = \min_i \frac{2}{\lambda_i}
   \end{equation*}
   mit $\rho(B) = \max_i |1-\alpha\lambda_i = \max \left\{ |1-\alpha\lambda_i|, |1-\alpha\lambda_n| \right\}$
   Der kleinste Spektralradius $\rho_{opt}$ ergibt sich durch Schnittpunktbildung der beiden Geraden. Berechne den opt Schnittpunkt von $|1-\alpha\lambda_1$ und $|1-\alpha\lambda_n|$\
BILD
   \begin{equation*}
     -(1-\alpha\lambda_n) = 1-\alpha\lambda_1 
     \Leftrightarrow \alpha_{opt} = \frac{2}{\lambda_1+\lambda_n}
     \Rightarrow \rho_{opt} = 1-\alpha_{opt} \lambda_n = 1 - \frac{2}{\lambda_1+\lambda_n} \lambda_n = \frac{\lambda_1-\lambda_n}{\lambda_1+\lambda_n}
   \end{equation*}
 \end{proof}

 Die Konvergenz des stationären Richardson-Verfahrens ist also abhängig von der Konditionszahl von $M^{-1}A$ bzw. vom größten und kleinsten Eigenwert:
 \begin{equation*}
   \lambda_1 := max \lambda_i, \quad \lambda_i \in \sigma(M^{-1}A)\\
   \lambda_n := min \lambda_i
 \end{equation*}

 \begin{bemerkung}
 Die Matrix $M$ ist frei wählbar und man kann die Kondition des Systems damit a-priori beeinflussen. Daher nennt man sie auch Vorkonditionierer (engl. preconditioner) oder Vorkonditionierungsmatrix
 \begin{itemize}
   \item $M$ sollte eine gute Approximation an $A$ sein und $\kappa(M^{-1}A)$ sollte möglichst klein sein
   \item $M^{-1}$ sollte sich einfach auf einem Vektor anwenden lassen, d.h. mit vertretbarem Rechenaufwand. 
 \end{itemize}
 \end{bemerkung}

 \begin{bemerkung}
 Man spricht beim Verfahren (**) auch vom vorkonditionniertem Richardson-Verfahren ($M\neq I$). 
 Nachteil: Für optimale Konvergenz muss Information über $\lambda_1$ und $\lambda_n$ vorliegen (die Ermittlung dieser, kann mitunter so teuer wie die Berechnung des ganzen Systems sein).
 \end{bemerkung}

 \begin{algorithmus}[Vorkonditioniertes Richardson-Verfahren]
   Init: 1. Geg. $X^{(0)}\in \R^n$ (Startvektor), $\alpha\in\R \\ \{0\}$
   2. Berechne das Startresiduum $r^{(0)} := b-Ax^{(0)}$
   Iteration: while $\|r^{(k)} \leq \varepsilon \|r^{(0)}$ für gegebenes $\varepsilon$\\
   $y^{(k)} := M^{-1} r^{(k)}$\\
 \end{algorithmus}


 
Konventionen: $y^{(k)} = M^{-1}r^{(k)}, x^{(k+1)} = Bx^{(k)} + C, \quad B=I -\alpha M^{-1}A, C=\alpha M^{-1}b$
Voraussetzung: $M^{-1}A$ hat nur positive reelle Eigenwerte $0<\lambda_n \leq \lambda_{n-1}\leq \dots \leq \lambda_1$
Das vorkonditionierte Richardson-Verfahren konvergiert $\Leftrightarrow 0<\alpha<\frac{2}{\lambda_1}$. Die optimale Konvergenzrate liegt dann bei $\alpha=\alpha_{opt} = \frac{2}{\lambda_1+\lambda_n}$
$M$ bzw. $M_{-1}$ bezeichnet man als Vorkonditionierer. Unter der Konditionszahl $\kappa\left( M^{-1}A \right)$ verstehen wir folgenden Ausdruck: $\kappa\left( M^{\frac{-1}{2}} A M^{\frac{-1}{2}} \right) = \frac{\lambda_{max}\left(  M^{\frac{-1}{2}} A M^{\frac{-1}{2}}  \right)}{\lambda_{min}\left(  M^{\frac{-1}{2}} A M^{\frac{-1}{2}}  \right)} \underbrace{=}_{A,M \text{ s.p.d}} \frac{\lambda_{max}(M^{-1}A)}{\lambda_{min}(M^{-1}A)}$

\section{Das Gradientenverfahren}
Voraussetzung: $A\in\R^{n\times n}$, s.p.d.
Dann löst $x\in\R^n$ $Ax=b \Leftrightarrow \phi(x) = \min_{y\in\R^n} \phi(y)$, mit $\phi(y) := \frac12 y^TAy - y^Tb$
Betrachte nun das Richardson-Verfahren, aber mit der Möglichkeit $\alpha=\alpha_k$ in jedem Schritt individuell zu wählen. Dann erhalten wir 
$\Rightarrow x^{(k+1)} = x^{(k)} + \alpha_k M^{-1}r^{(k)}, \quad r^{(k)}:=b-Ax^{(k)} =Ax-Ax^{(k)} = Ae^{(k)}$
Wie zuvor ist: $e^{(k+1)}:= e^{(k+1)}(\alpha_k):=\underbrace{\left( I-\alpha_k M^{-1}A \right)}_{=:B^{(k)}}e^{(k)}$
Sei 
\begin{equation}
  \|x\|_A := \sqrt{x^TAx} = \sqrt{(x,x)_A},\quad (x,y)_A := y^TAx
  \label{}
\end{equation}
Dann gilt
\begin{equation}
  \|e^{(k+1)}\|_A^2 = \left( Ae^{(k+1)}, e^{(k+1)} \right)_2 = \left( r^{(k+1)}, e^{(k+1)} \right)_2\\
  \label{}
\end{equation}
\begin{itemize}
  \item[i] $e^{(k+1)} = \left( I-\alpha_kM^{-1}A \right)e^{(k)} -\alpha_kM^{-1}Ae^{(k)}$
  \item[ii] $r^{(k+1)} = r^{(k)}-\alpha_kAM^{-1}Ae^{(k)}$  
\end{itemize}
\begin{equation}
  \Rightarrow \|e^{(k+1)}\|_A^2 \underbrace{=}_{y^{(k)} = M^{-1}r^{(k)}} \left( r^{(k)}, e^{(k)} \right) -\alpha_k \left( (Ay^{(k)},e^{(k)}) + (r^{(k)},M^{-1}Ae^{(k)}) \right) + \alpha_k^2 (Ay^{(k)}, M^{-1}Ae^{(k)})\\
  \underbrace{\Rightarrow}_{e^{(k)}A^{-1}r^{(k)}} \|e^{(k+1)}\|_A = (r^{(k)},e^{(k)}) - 2\alpha_k (y^{(k)},r^{(k)}) + \alpha_k^2 (Ay^{(k)}, y^{(k)})
  \label{}
\end{equation}

Das Residuum sollte klein werden. Deshalb minieren wir den Fehler in der Energienorm.

Jetzt wählen wir $\alpha_k$ im $k+1$ Iterationsschritt so, dass der Fehler in der $A$-Norm miniert wird, d.h. wir minieren die Funktion 
\begin{equation*}
  f(\alpha_k) := \|e^{(k+1)}(\alpha_k)\|_A^2
\end{equation*}

Betrachte 
\begin{equation}
  0 = \frac{\partial f}{\partial \alpha}(\alpha_k) = -2 (y^{(k)},r^{(k)}) + 2 \alpha_k (Ay^{(k)}, y^{(k)})\\
  \Leftrightarrow \alpha_k = \frac{y^{(k)},r^{(k)}}{Ay^{(k)},y^{(k)}}
  \label{}
\end{equation}

\begin{equation}
  \frac{\partial^2 f}{\partial \alpha_k^2}(\alpha_k) \underbrace{=}_{A spd} 2(Ay^{(k)},y^{(k)}) > 0
  \label{}
\end{equation}
$\alpha_k$ ist tatsächlich ein lokales Minimum

\begin{equation}
  \|e^{(k+1)}\|_A^2 \underbrace{=}_{Def} (Ae^{(k+1)},e^{(k+1)}) = \underbrace{=}_{Def} (A(x-x^{(k+1)},(x-x^{(k+1)}))) \underbrace{=}_{Ax=b} (b,x) - \underbrace{(b,x^{(k+1)}) - (x^{(k+1)},b)}_{=-2(b,x^{k+1})} + (Ax^{(k+1)},x^{(k+1)}) = (b,x) + 2\phi(x^{(k+1)})
  \label{}
\end{equation}

Also ist die lokale Minimierung des Fehlers $e^{(k+1)}$ in der $A$-Norm äquivalent zur Minimierung des Funktionals $\phi(x^{(k+1)}(\alpha_k)$ unter allen Vekoten $X^{(k+1)}(\alpha_k)$ der Form 
$x^{(k)} + \alpha_k M^{-1}r^{\left( k \right)}$ 
durch Lösen $1$-dim. Minimierungsprobleme.

Geometrische Veranschaulichung für $M=I$. Sei $\phi(x)$ die Höhenfunktion.
$\Rightarrow$ Gradientwert: $X^{(k+1)} + \alpha_k r^{(k)} = x^{(k)} - \alpha_k \Grad_x\phi(x^{(k)})$.
Das Gradientenverfahren minimiert lokal in Richtung des steilsten Abstiegs.
$\Rightarrow$ steilster Abstieg $- \nabla_x \phi(x) = - (Ax-b) = b-Ax$
$\Rightarrow -\nabla_x\phi(x^{(k)}) = r^{(k)}$

\begin{satz}[Kantorowitsch Ungleichung]
  Sei $A\in R^{n\times n}$ s.p.d. mit spektraler Konditionszahl $\kappa(A) := \frac{\lambda_{max}(A)}{\lambda_{min}(A)}$. Dann gilt für jeden Vektore $0\neq x\in \R^n$die Ungleichung 
  \begin{equation}
    \frac{(x^TAx)(x^TA^{-1}x)}{(x^tx)^2} \leq \frac{(\lambda_{max}+\lambda_{min})^2}{4 \lambda_{max} \lambda_{min}}
    \label{}
  \end{equation}  
\end{satz}

\begin{proof}
  Die Eigenwerte von $A$ seien geordnet $0<\lambda_{min} = \lambda_1 \leq \lambda_2 \leq \dots \leq \lambda_n = \lambda_{max}$ Diagonalisierung von $A$ durch Orthonormales $Q$
  \begin{equation}
    Q^TAQ = D = \diag_{i=1,..,n} (\lambda_i) \Rightarrow A^{_1} = (QDQ^T)^{-1} = QD^{-1}Q^T\\
    \Rightarrow \frac{(x^TAx)(x^TA^{-1}x)}{(x^Tx)^2}  = \frac{(x^TQDQ^Tx)(x^TQD^{-1}Q^Tx)}{(x^Tx)^2} = \frac{(y^TDy)(y^tD^{-1}y)}{\underbrace{(y^TQ^TQy)}_{=I}}\\
    \frac{(y^TDy)(y^TD^{-1}y)}{(y^Ty)^2} = \frac{(\sum_{i=1}^{n}y)i^2 \lambda_i}{(\sum_{i=1}^{n}y_i^2)} \frac{(\sum_{i=1}^{n}y)i^2 \lambda_i^{-1})}{(\sum_{i=1}^{n}y_i^2)} 
    = \left( \sum_{i=1}^{n} z_i \lambda_i \right) \left( \sum_{i=1}^{n} z_i \lambda_i^{-1} \right), \quad z_i:= \frac{y_i^2}{\sum_{j=1}^{n}y_j^2} \Rightarrow \sum_{i=1}^{n} z_i = 1
    \Rightarrow \sum_{i=1}^{n} z_i \lambda_i, \sum_{i=1}^{n} z_i \lambda_i^{-1} \text{ sind konvexe Kombination}
    \label{}
  \end{equation}
  Sei $g: \lambda \to \frac1\lambda$, dann liegen alle Punkte $(\lambda_i,\frac1\lambda_i)$ auf dem Graphen von g.
  BILD 
  Daher liegen alle Punkte $(\lambda,\frac1\lambda)$ mit $\lambda_{min} < \lambda < \lambda_{max}$ unterhalb der Geraden durch die Punkte $(\lambda_{min}, \frac{1}{\lambda_{min}})$ und $(\lambda_{max}, \frac{1}{\lambda_{max}})$. Der Punkt $\left(  \sum_{i=1}^{n} z_i \lambda_i, \sum_{i=1}^{n} z_i \lambda_i^{-1} \right)$ liegt in der konvexen Hülle der Punkte $(\lambda_i, \frac{1}{\lambda_i}), i=1,..,n$. Daher liegen die konvexen Kombinationen $\left(  \sum_{i=1}^{n} z_i \lambda_i, \sum_{i=1}^{n} z_i \lambda_i^{-1} \right)$ in der schraffierten Fläche, insbesondere unterhalb der Geraden durch $(\lambda_{min}, \frac{1}{\lambda_{min}})$ und $(\lambda_{max}, \frac{1}{\lambda_{max}})$
  $g''(\lambda) > 0 \quad \forall \lambda \in \left[ \lambda_{min},\lambda_{max} \right] \Rightarrow g $ist konvexe Funktion

  $\Rightarrow \lambda \to \frac{1}{\lambda_{min}} + \frac{\frac{1}{\lambda_{max}}-\frac{1}{\lambda_{min}}}{\lambda_{max} - \lambda_{min}} (\lambda-\lambda_{min})$ 
  $= \frac{\lambda_{min} + \lambda_{max}-\lambda}{\lambda_{max} - \lambda_{min}}$
  $\Rightarrow $ FEHLT

  Standardergebnis Analysis: 
  \begin{equation}
    \max_{\lambda_{min} \leq \lambda \leq \lambda_{max}} \left( \lambda \frac{\lambda_{max} + \lambda_{min} - \lambda}{\lambda_{max}\lambda_{min} \right) = \frac{(\lambda_{min}+\lambda_{max})^2}{4\lambda_{max}\lambda_{min}}
    \label{}
  \end{equation}
Hieraus folgt die Behauptung mit $\lambda:= \sum_{i=1}^{n} z_i \lambda_i \in (\lambda_{min}, \lambda_{max})$
\end{proof} 

Folgerung: 
Sei $M\in\R^{n\times n}$ spd, dann gilt 
\begin{equation}
  \frac{(M^{-1}Ax,x)_M}{(x,x)_M} \leq \frac{(\lambda_{min}+\lambda_{max})^2}{4\lambda_{min}\lambda_{max}}
  \label{}
\end{equation}

$\lambda_{min} = $ kleinester Eigenwert von $M^{-1}A$ und $\lambda_{max} = $ größter Eigenwert von $M^{-1}A$

Letzter Eintrag.

 \underline{Folgerung:} $M \in \mathbb{R}^{n \times n}$, s.p.d., dann gilt:

\[\frac{ (M^{-1}Ax,x)_M(A^{-1}Mx,x)_M}{(x,x)^2_M } \leq \frac{(\lambda_{max} + \lambda_{min})^2}{4\lambda_{max}\lambda_{min}} \]
wobei $\lambda_{max}=\lambda_{max}(M^{-1/2}AM^{-1/2}),\, \lambda_{min}=\lambda_{min}(M^{-1/2}AM^{-1/2})$.\\
Es gilt:
\[ A=Q^TDQ,\, A^{1/2}:=Q^TD^{1/2}Q,\, D^{1/2}:=diag_{i=1,\dots, n} ( \sqrt{\lambda_i}) \]
\underline{Beweis der Folgerung:}\\
$y:=M^{1/2}x$
\begin{align*}
\frac{(M^{-1}Ax,x)_M(A^{-1}Mx,x)_M}{(x,x)^2_M} \\
^{y=M^{1/2}x} &= \frac{ (AM^{-1/2}y,M^{-1/2}y)(A^{-1}M^{1/2}y,M^{1/2}y)}{(y,y)^2} \\
&= \frac{ (M^{-1/2}AM^{-1/2}y,y)(M^{1/2}A^{-1}M^{1/2}y,y)}{(y,y)^2} \\
^{\tilde A:=M^{-1/2}AM^{-1/2}} &= \frac{(\tilde A y ,y)(\tilde A^{-1}y,y)}{(y,y)^2} \\
^{\text{Kant. Ungl. } + \tilde A \, s.p.d.} &= \frac{(\lambda_{max}(\tilde A) + \lambda_{min}(\tilde A))^2}{4\lambda_{max}(\tilde A)\lambda_{min}(\tilde A)}
\end{align*}
$\lambda(\tilde A)=\lambda (M^{-1}A)$ \\
%$\hfill \box$
Es ist : $ e^{(k)}:= x-x^{(k)},\, y^{(k)}:= M^{-1}r^{(k)},\, Ax=b,\, r^{(k)}=b-Ax^{(k)}$.
\[ \frac{||e^{(k)}||^2_A - ||e^{(k+1)}||^2_A}{||e^{(k)}||_A} =\frac{(y^{(k)},y^{(k)})^2_M}{(M^{-1}Ay^{(k)},y^{(k)})_M}(A^{-1}My^{(k)},y^{(k)})_M \]
\[\Leftrightarrow ||e^{(k+1)}||^2_A \leq \left( 1-\frac{4\lambda_{max}\lambda_{min}}{(lambda_{max}+\lambda_{min})^2}\right) ||e^{(k)}||^2_A = \frac{(\lambda_{max}-\lambda_{min})^2}{(\lambda_{max}+\lambda_{min})^2} ||e^{(k)}||^2_A \]
\[ \Leftrightarrow || e^{(k+1)}||_A \leq \frac{(\lambda_{max} - \lambda_{min})}{\lambda_{max} + \lambda_{min}} ||e^{(k)}||_A = \left( \frac{\kappa (M^{-1}A) -1 }{\kappa (M^{-1}A)+1} \right) ||e^{(k)}||_A \leq \left( \frac{\kappa (M^{-1}A) -1 }{\kappa (M^{-1}A)+1} \right)^k ||e^{(0)}||_A \]

\begin{satz}
  Beim instationären (vorkonditionierten) Richardson-Verdahren/(vorkonditioniertes) Gradientenverfahren gilt für die Konvergenzrate folgende obere Schranke:
  \begin{equation}
    \kappa \frac{(M^{-1}A)-1}{\kappa(M^{-1}A)+1}
  \end{equation}
  wobei $\kappa(M^{-1}A):= \kappa_2(M^{-1/2}AM^{-1/2}) = \frac{\lambda_{max}(m^{-1/2}AM^{-1/2})}{\lambda_{min}(M^{-1/2}AM^{-1/2})}$
\end{satz}

\begin{algorithmus}[vorkonditioniertes Gradientenverfahren]
  Initialisierung: 
  \begin{itemize}
    \item[1] geg. Startvektor $x^{(0)} \in \R^n$
    \item[2] $r^{(0)}:= b-Ax^{(0)}$ (Startresidium)
  \end{itemize}
  Iteration: $k=0,1,2,\dots$ solange Konvergenzkriterium erfüllt, z.B. 
  $\|r^{(k)}\|_2 \leq \varepsilon \|r^{(0)}\|_2\\
  y^{(k)}:=M^{-1}r^{(k)}\\
  q^{(k)}:=Ay^{(k)}\\
  \alpha^{(k)}:=\frac{(y^{(k)},r^{(k)})}{(q^{(k)},y^{(k)})}\\
  x^{(k+1)}:=x^{(k)}+\alpha^{(k)}y^{(k)}\\
  r^{(k+1)}:=r^{(k)}-\alpha^{(k)}q^{(k)}$
\end{algorithmus}

\begin{bemerkung}
  Man beachte, dass in dem vorliegenden Algorithmus in jeder Iteration jeweils mindestens eine Matrix-Vekotr-Multiplikation mit $A$ bzw. $M^{-1}$ benötigt wird.
\end{bemerkung}

\subsubsection{Das Verfahren der konjugierten Gradienten}
Beim Gradientenverfahren haben wir zwei Phasen kennengelernt, um $x^{(k+1)}$ aus $x^{(k)}$ zu berechnen:
\begin{itemize}
  \item Bestimmen der Suchrichtungen $y^{(k)}$
  \item Berechnen des lokalen Minimums von $\Phi$ in dieser Richtung
\end{itemize}

\begin{bemerkung}[Beobachtung]
  2) ist unabhängig von 1); 
\end{bemerkung}

Ist eine beliebige Suchrichtung $p^{(k)}$ gegeben, so bestimme den Relaxationsparameter $\alpha^{(k)}$ derart, dass $\Phi(x^{(k)}+\alpha^(k)p^{(k)})$ minimal wird. Dies kann man wie beim Gradientenverfahren machen, d.h. 
\begin{equation*}
  \alpha^{(k)} := \frac{(p^(k)),r^{(k)}}{(Ap^{(k)},p^{(k)})}
\end{equation*}

Frage: Kann man $p^{(k)}$ besser wählen als im Gradientenverfahren? (Wie will man Optimalität überhaupt definieren?) Dazu betrachten wir folgende Definition der Optimalität von Suchrichtungen ($\to$ Die Suchrichtung werden orthogonal zu den Resiuuen gewählt)

\begin{definition}
  Eine Richtung $x\in \R^n$ heißt optimal bzgl. einer Richtung $p\in \R^n \equiv \Phi(x) \leq \Phi(x+\lambda p) \quad \forall \lambda \in \R$
  $\Phi(y):= \frac 1 2 y^TAy - b^Ty$
\end{definition}



\underline{Bem. 1.2.2.:}
\[ x \text{ optimal bzgl. } p \, \Leftrightarrow \, p \bot r := B-Ax, \text{ wobei } p \bot r : \Leftrightarrow (p,r)=0 \]
\underline{Beweis:} $ \phi (x)=\frac{1}{2} x^TAx-b^Tx $\\
\[ \lambda \in\mathbb{R} : \, \phi (x + \lambda p) = \phi(x) + \lambda (\underbrace{Ax-b}_{=-r},p)+ \frac{\lambda^2}{2}\underbrace{(Ap,p)}_{\geq 0} \]
Dann gilt:
\begin{align*}
	\phi(x) &\leq \phi(x+\lambda p ) \forall \, \lambda \in\mathbb{R} \\
\Leftrightarrow 0 & \leq -\lambda (r,p)+\underbrace{\frac{\lambda^2}{2}(Ap,p)}_{\geq 0} \forall \, \lambda \in \mathbb{R} \\
\Leftrightarrow (r,p) &=0 \\
\Leftrightarrow r & \bot p
\end{align*}
%$\hfilll \box$
Für das Gradientenverfahren mit $M=I$ gilt:\\
$x^{(k+1)}$ ist optimal bzgl. $r^{(k)}$ ( $\alpha^{(k)}$ wird gerade so gewählt, dass $r^{(k+1)} \bot r^{(k)}$ ist).\\
Im Allgem. ist $x^{(k+2)}$ nicht mehr optimal bzgl. $r^{(k)}$.\\
\underline{Frage:} Gibt es Abstiegsrichtungen die diese Optimalität erhalten?\\
Dazu sei $x^{(k+1)}=x^{(k)}+q$, wobei $x^{(k)}$ optimal sei bzgl. einer Richtung $p$, d.h. $p \bot r^{(k)}$. Wir verlangen nun, dass $x^{(k+1)}$ auch bzgl. $p$ optimal sein soll:
\[ r^{(k+1)} \bot p \, \Leftrightarrow \, 0 = (r^{(k+1)},p) = (r^{(k)}-Aq,p)=-(Aq,p) \]
Also gilt:
\[ r^{(k+1)} \bot p \, \Leftrightarrow \, (q,p)_A=(Aq,p)=0 \, \Leftrightarrow \, q \text{ ist $A-$orth. zu } p \]

\begin{bemerkung}
  Vorkonditionierte Verfahren sind für uns so wichtig, weil das Optimalitätskriterium der Orthogonalität in der Realität auf Grund von Rundungsfehlern (insbesondere bei schlecht konditionierten Problemen) diese verwischen würde. Ein Vorkonditionierer hilft uns dem entgegenzuwirken.
\end{bemerkung}

\begin{bemerkung}[Folgerung]
  Um die Optimalität aufeinander folgender Iterationen zu gewährleisten, müssen die Abstiegsrichtungen gegenseitig $A$-orthogonal sein, d.h. $(p,q)_A=0$. Dies nennt man \underline{$A$-konjugiert} oder \underline{konjugiert} zueinander. Ein Verfahren, welches konjugierte Abstiegsrichtungen verwendet, nennt man \underline{konjugiertes Verfahren}.
\end{bemerkung}

\begin{bemerkung}[vorgehensweise]
  Sei $p^{(0)}:= r^{(0)}$, dann konstruiere Richtungen der Form 
  \begin{equation}
    p^{(k+1)}:=r^{(k+1)}-\beta^{(k)}p^{(k)}, \quad k=0,1,\dots
  \end{equation}
  Wähle dabei $\beta^{(k)} \in \R$, so dass 
  \begin{equation}
    (Ap^{(j)},p^{(k+1)})=0 \quad \forall j=0,1,2,\ldots,k
  \end{equation}
  d.h. $p^{(k+1)}$ soll konjugiert sein zu allen vorherigen Richtungen $p^{(j)}, j=0,\cdots,k$
  Aus ref 1,2 und $j=k$ folgt $\beta^{(k)}=\frac{(p^{(k)},r^{(k+1)})_A}{(p^{(k)},p^{(k)})_A}$
\end{bemerkung}

  Es bleibt zu Zeigen: ref 2 gilt dann auch für $j=0,\cdots,k$
  \begin{proof}
    vollständige Induktion (Übung)
  \end{proof}

 \begin{algorithmus}[Konjugiertes Gradientenverfahren (cg-Verfahren)]
  \begin{align*}
    \alpha^{(k)} := \frac{(p^{(k)},r^{(k)})}{(p^{(k)},Ap^{(k)})},\quad
    \beta^{(k)} := \frac{(p^{(k)},r^{(k+1)})_A}{(p^{(k)},p^{(k)})_A}
  \end{align*}
mit $ p^{(0)}:= r^{(0)}=b-Ax^{(0)}$.\\
Für $k=0,1,2,\ldots $:\\

  Berechne $\alpha^{(k)}$
  \begin{align*}
    x^{(k+1)}&:=x^{(k)}+\alpha^{(k)}p^{(k)} (*)\\
    r^{(k+1)}&:=r^{(k)} -\alpha^{(k)}Ap^{(k)} (**)
  \end{align*}
  Berechne $\beta^{(k)}$
  \[
    p^{(k+1)}:=r^{(k+1)}-\beta^{(k)}p^{(k)} (***)
  \]
  Es gilt auch (Übungen):
  \[
    \alpha^{(k)} = \frac{(r^{(k)},r^{(k)})}{(Ap^{(k)},p^{(k)})}, \beta^{(k)}=\frac{(r^{(k+1)},r^{(k+1)})}{(r^{(k)},r^{(k)})}
  \]
\end{algorithmus}

\begin{satz}[1.3.3]
  Sei A s.p.d. Dann konvergiert das cg-Verfahren und es gilt folgende Fehlerabschätzung:
  \[
    \|e^{(k)}\|_A \leq 2 \left( \frac{\sqrt{\kappa}-1}{\sqrt{\kappa}+1} \right)^k \|e^{(0)}\|, \quad k=0,1,\ldots
  \]
  mit $\kappa = \kappa(A) = \frac{\lambda_{max}(A)}{\lambda_{min}(A)}$
\end{satz}

\begin{proof}
  Folgt den Argumenten von Quaternioni/Valle Numerical Mathematics for PDE, Springenr, S.48-50 und Quarteroni/Suer/Sater, Numerical Mathematics, S.154
  \begin{enumerate}[1)]
    \item (*)$x^{(k+1)}= x^{(k)} + \alpha^{(k)}p^{(k)} \underbrace{=}_{Argument} x^{(0)} + \sum_{i=0}^{k} \alpha^{(i)}p^{(i)}$

    \item Sei $x=A^{-1}b$ die exakte Lösung , dann folgt aus 1) 
      \[
        (Ap^{(j)}, x^{(k+1)}-x^{(0)}) = \sum_{i=0}^{k} \alpha^{(i)}\underbrace{(Ap^{(j)},p^{(i)})}_{=0, i\neq j} 
        = \alpha^{(j)}(Ap^{(j)},p^{(j)}) \underbrace{=}_{Def. \alpha^{(j)}} (r^{(j)},p^{(j)}) = (b-Ax^{(j)},p^{(j)})
        = (A(x-x^{(j)}),p^{(j)}) 
        = (x-x^{(0)},Ap^{(j)}) + \underbrace{(\underbrace{x^{(0)}-x^{(j)}}_{\underbrace{=}_{1)}-\sum_{i=0}^{j-1}\alpha^{(i)}p^{(i)}},Ap^{j})}_{=0, \text{ da } (p^{(i)},Ap^{(j)})=0, i=0,\ldots,j-1}
        = (x-x^{(0)},Ap^{(j)})
        \Leftrightarrow (x^{(k+1)}-x^{(0)},p^{(j)})_A = (x-x^{(0)},p^{(j)})_A
      \]

    \item \begin{lemma}
      \[span\{p^{(0)},\dots p^{(k)} \} = span \{ r^{(0)},Ar^{(0)},\dots A^kr^{(0)}\} =: \underbrace{\Kc_{k+1}(r^{(0)})}_{\text{Krylovraum}} \]
      Offensichtlich gilt: $span \{ p^{(0)},\dots p^{(k)}\} =span \{ r^{(0)},r^{(1)},\dots , r^{(k)}\}$\\
      \end{lemma}

      \begin{proof}
        3) mit vollst. Induktion;\\
        IA: da $p^{(0)}=r^{(0)}$ \\
        $'\subset '$: IV: $ span \{ p^{(0)},\dots , p^{(k)} \} = \Kc_{k+1}(r^{(0)}) $\\
        IS: IV $\Rightarrow Ap^{(k)} \in \Kc_{k+2}(r^{(0)})$\\
        Aus $(\ast \ast)$ also $r^{(k+1)}=r^{(k)}-\alpha^{(k)}Ap^{(k)} \in \Kc_{k+2}(r^{(0)})$ 
        
        \begin{align*}
          ^{(\ast \ast \ast)}\Rightarrow& p^{(k+1)}=r^{(k+1)}-\beta^{(k)}p^{(k)} \in \Kc_{k+2}(r^{(0)}) \\
          \Rightarrow & span \{ p^{(0)},\dots p^{(k+1)} \} \subset \Kc_{k+2}(r^{(0)})
        \end{align*}
        $'\supset '$\\
        IV $span \{p^{(0)},\dots p^{(k)}\} = \Kc_{k+1}(r^{(0)})$ \\
        IS: IV $\Rightarrow A^kr^{(0)} \in span \{ p^{(0)},\dots p^{(k)} \}$ \\
        \begin{align*}
          ^{(\ast \ast)} \Rightarrow & A^{k+1}r^{(0)} \in span \{ r^{(0)},\dots , r^{(k+1)} \} = span \{ p^{(0)},\dots p^{(k+1)} \} \\
          \Rightarrow & '\supset '
        \end{align*}
      \end{proof}

    \item $^{2) + 3)} \Rightarrow \, x^{(k+1)}-x^{(k)}$ ist die Orthogonalprojektion des Anfangsfehlers $ x-x^{(0)} =: e^{(0)}$ auf den Krylovraum $\Kc_{k+1}(r^{(0)}$.\\

    \item es gilt mit 4) und $e^{(k+1)}=x-x^{(k+1)}$:
        \begin{align*}
          ||e^{(k+1)}||A &= || (x-x^{(0)})-(x^{(k+1)} -x^{(0)} ||_A \\
          &= \min_{y \in \Kc_{k+1}(r^{(0)})} ||x-x^{(0)} - y ||_A = \min_{y \in \Kc_{k+1}(r^{(0)})} ||e^{(0)} - y ||_A 
        \end{align*}
        Da $r^{(0)} =A(x-x^{(0)})=Ae^{(0)}$  lässt sich ein bel. $y \in \Kc_{k+1}(r^{(0)})$ schreiben als:
        \[ y= \sum_{j=0}^k \tilde \gamma_j A^jr^{(0)} = \sum_{j=1}^{k+1}\gamma_j A^j e^{(0)} = \tilde p (A)e^{(0)}, \,  p \in \Pc_{k+1} \]
        Also gilt $\|e^{(k+1)}\|_A = \min_{p\in \Pc^*_{k+1}} \|p(A)e^{(0)}\|_A$, wobei $\Pc^*_{k+1}=\left\{ q\in \Pc_{k+1} | q(0)=1 \right\}$

    \item A ist s.p.d. $\Rightarrow A=Q^TDQ, Q$ orthogonale Matrix, $D=diag(\lambda_i), \lambda_i$ EV von $A$\\
      $\Rightarrow p(A) = Q^Tp(D)Q$\\
      $\Rightarrow \|e^{(k+1)}\|^2_A = \min_{p\in \Pc^*_{k+1}} \|p(A)e^{(0)}\|_A^2 
      =  \min_{p\in \Pc^*_{k+1}} \|Q^Tp(D)Qe^{(0)}\|_A^2 $\\
      $ \min_{p\in \Pc^*_{k+1}} \left( D\underbrace{QQ^T}_{=I}p(D)Qe^{(0)}\underbrace{QQ^T}_{=I} \underbrace{p(D)Qe^{(0)}}_{=z} \right)$\\
      $  \min_{p\in \Pc^*_{k+1}} (Dp(D)z,p(D)z) = \sum_{i=1}^{n}\lambda_i z_i^2(p(\lambda_i))^2$\\
      $\leq  \min_{p\in \Pc^*_{k+1}} \max_{\lambda_i \in \sigma(A)} (p(\lambda_i))^2 \underbrace{\sum_{i=1}^{n}\lambda_i z_i^2}_{=(PQe^{(0)},Qe^{(0)})=\|e^{(0)}\|_A^2}$
      $= \left(  \min_{p\in \Pc^*_{k+1}}  \max_{\lambda_i \in \sigma(A)} (p(\lambda_i))^2 \right) \|e^{(0)}\|_A^2$\\
      $\Leftrightarrow \frac{\|e^{(k+1)}\|_A}{\|e^{(0)}\|_A} 
      \leq  \min_{p\in \Pc^*_{k+1}}  \max_{\lambda_i \in \sigma(A)} |p(\lambda)|$

    \item Als nächstes versuchen wir ein Polynom $q\in \Pc^*_{k+1}$ zu konstruieren, s.d. $|q(\lambda)|$ möglichst klein ist für $\lambda\in [\lambda_{min}, \lambda_{max}]$. Dazu führen wir die \underline{Tschebyscheff-Polynome} ein (vgl. Martin Haute-Bourgois, Grundlagen d. Num. Mathematik und WissRech, Vieweg-Teubner, 3. Auflage S.284). 
      Definition  $T_k(x) := \cos(k arccos(x)), \quad k\geq 0, x\in [-1,1]$ Tschebyscheff-Polynome k-ter Ordnung.
  $T_0(x)=1, T_1(x)=x, t:=arccos(x)$
  Behauptung: $T_{k-1}(x) + T_{k+1}(x) = \cos((k-1)t) + \cos ((k+1)t)$
  Somit gilt die rekursive Darstellung
  \[ T_0(x)=1,\, T_1(x)=x,\, T_{k+1}(x)=2xT_k (x) - T_{k-1}(x),\, k=1,2,\dots \]
  $ \Rightarrow T_k \in \Pc_k$ \\

\item Es gilt auch die Darstellung
  \[ T_K (x) =\cosh (k\cdot arccosh (x))\text{ für } x \geq 1,\, T_0(x)=1,\, T_1(x)=x \]
  Für den Rest muss man die Rekursionsformel anwenden:\\
  Setze für $x \geq 1$: $t:=arccosh(x) = ln (x+\sqrt{x^2-1})$\\
  Dann gilt
  \[ T_k(x) = cosh (kt) = \frac{1}{2} (e^{kt} + e^{-kt})\geq \frac{1}{2} (e^t)^k = \frac{1}{2} (x+\sqrt{x^2-1})^k \]

\item Wähle nun $q \in \Pc^*_{k+1}$ wie folgt
  \[ q(\lambda) := \frac{T_{k+1}\left( \frac{2\lambda - \lambda_{min} - \lambda_{max}}{\lambda_{min}-\lambda_{max}} \right) }{T_{k+1} \left( \frac{\lambda_{min}+\lambda_{max}}{\lambda_{max} - \lambda_{min}}\right)} \Rightarrow q(0)=1, q\in\Pc_{k+1} \rightarrow q\in\Pc^*_{k+1} 
  \]


 Christoph

\end{enumerate}
\end{proof}


\begin{algorithmus}[PCG-Verfahren]
  Init: $x^{(0)}\in \R$ (Startvektor)
  $r^{(0)} = b-Ax^{(0)}$ (Startresiduum)
  $y^{(0)}-M^{-1}r^{(0)}$
  $p^{(0)}=y^{(0)}$
  Iteration: $k=0,1,2,\cdots$ (bis Konvergenzkriterium erfüllt ist) 
  \begin{align*}
    \alpha^{(k)}&= \frac{(p^{(k)},p^{(k)})}{p^{(k)},Ap^{(k)}}\\
    x^{(k+1)}&= x^{(k)}+\alpha^{(k)}p^{(k)}\\
    r^{(k+1)}&= r^{(k)}+\alpha^{(k)}Ap^{(k)}\\
    x^{(k+1)}&= M^{-1}r^{(k+1)}\\
    \beta^{(k)}&= \frac{(y^{(k+1)},Ap^{(k)})}{(p^{(k)},Ap^{(k)}}\\
    p^{(k+1)}&= y^{(k+1)}-\beta^{(k)}p^{(k)}
  \end{align*}
\end{algorithmus}

\begin{bemerkung}
  Dieser Algorithmus benötigt nicht die Anwendung von $M^{-1/2}$ auf einen Vektor, sondern nur die von $M^{-1}$. Es handelt sicht \underline{nicht} um das CG-Verfahren angewendet auf $M^{-1}A$
\end{bemerkung}

\chapter{Das Modellproblem}
\section{Die Modellgleichung und Finite Elemente}
Sei $\Omega\in\R^d$ ein Polygon oder Polyeder, $d=2,3$
\begin{equation}
  \begin{aligned}
    -\nabla (u(x) \nabla u) &= f &\quad \inn \Omega\\
    u &= 0 &\quad \auf \partial\Omega_D\\
    \frac{\partial u}{\partial v} &= g &\quad \partial\Omega_N
  \label{}
  \end{aligned}
\end{equation}
wobei $\partial\Omega = \partial\Omega_D \cup \partial\Omega_N$ mit $\partial\Omega_D\cap\partial\Omega_N=\emptyset$ und $|\partial\Omega .. .$FEHLT
Betrachte das zugehörige Variationsproblem:
\begin{problem}
  Finde
  \begin{align*}
    u\in H_0^1(\Omega,\partial\Omega_D) &= \left\{ v\in H^1(\Omega): v|_{\partial\Omega_D} =0 \right\}\\
    \text{s.d. } a(u,v) &= F(v) \quad \forall v\in H_0^1(\Omega,\partial\Omega_D)
  \end{align*}
  mit $a(u,v) := \int_{\Omega} a(x) \nabla u \nabla v \ud x, 
     \quad F(v):=\int_\Omega f\cdot v \ud x + \int_{\partial\Omega_N}g\cdot v \ud \sigma$ 
\end{problem}

Dieses Variationsproblem soll mit konformen finiten Elementen gelöst werden. Wenn nichts anderes gesagt wird, wenden wir lineare Dreiecks- oder Tetraederelemente an. \\
Sei $V^h\subset V := H_0^1(\Omega,\partial\Omega)$ der zugehörige Finite Elemente Raum. 

Christoph

\begin{proof}
  \begin{enumerate}
    \item Zunächst beweisen wir die Aussage auf dem Referenzelement $\hat{T}$ und wenden dann die Transformationsformel für Integrale an.
      Sei $F_T: \hat{T} \to T, \hat{x}\to B\hat{x}+d$ für jedes Element $T\in \tau_h$. $\hat{\Pc}$ sei der lokale Ansatzraum. $\hat{\Pc}$ ist endlichdimensional, daher sind dort alle Nenner äquivalent, somit auch $\norm{\hat{v}}_{l_2}$ und $\norm{\hat{v}}_{L^2(\hat{T})}, : \exists \hat{c_1}, \hat{c_2} > 0$ s.d.
      \begin{equation} %(*)
        \hat{c_1} \norm{\hat{v}}^2_{L^2(\hat{T})} \leq \sum_{i=1}^{M} |\hat{v}(a_i)|^2 \leq \hat{c_2} \norm{\hat{v}}^2_{L^2(\hat{T})}
        \label{}
      \end{equation}
      wobei $\hat{a_i} = $Freiheitsgrade im Referenzelement $\hat{T}$ (Bild vom Ref. Dreieck mit Ecken $\hat{a_1},\hat{a_2},\hat{a_3}$)
      Aus Analysis III, bzw. Numerik PDG I $\Rightarrow \vol(T) = |\det B| \cdot \vol(\hat{T})$

      $(\tau_h)$ ist regulär, also $h_t \leq \sigma \rho_T, \quad \forall T\in \tau_h$
      $\Rightarrow \exists \tilde{c_1}, \tilde{c_2}, \tilde{c_3} >0 $ s.d. 
      %\begin{equation}[**] %(**)
      \[
        \tilde{c_1} h_T^d \leq \tilde{c_3} \rho_T^d \leq |\det(B)| \leq \tilde{c_2} h_T^d \tag{$\star$}
      \]
        \label{ }
      %\end{equation}
      Durch die Transformationsformel erhalten wir für $v \in \Pc_T$
      \begin{align}
        \hat{c_1} \norm{v}_{L^2(T)}^2 \overset{\hat{v}=v^\circ F_T}{=} \hat{c_1} |det (B)| \norm{\hat{v}}_{L^2}(\hat{T})^2 
        \overset{*,**}{\leq} \tilde{c_2} h_T^d \sum_{i=1}^{M} |\hat{v}(\hat{a_i})|^2
        \overset{a_i = F_T(\hat{a}_i)}{\underset{\hat{v}=v \circ F_T}{=}} \tilde{c_2} h_T^d \sum_{i=1}^{M}|v(a_i)|^2 = c_2 h_T^d \sum_{i=1}^{M} |\hat{v}(\hat{a_i})|^2
      Christoph
    \end{align}
  \end{enumerate}
\end{proof}

 \begin{satz}{2.2.3}\\
Es existiert eine Konstante $c>0$, unabhängig von $h$, s.d. 
\[\kappa (M) \leq c \frac{h^d}{(\min_{T \in \tau_h} h_T)^d} \]
wobei $M=(\varphi_i,\varphi_j)_{i,j}$ die Massenmatrix, $(\varphi_i)_i$ die nodale Basis und $\kappa (M)$ die spektrale Konditionszahl sei.
\end{satz}
\begin{proof}
\[\lambda_{max}=\max_{x  \in \mathbb{R}^n\setminus \{ 0 \}} \frac{x^TMx}{x^Tx} , \, \lambda_{min}=\min_{x  \in \mathbb{R}^n\setminus \{ 0 \}} \frac{x^TMx}{x^Tx} \]
$ \frac{x^TMx}{x^Tx}=\frac{x^TMx}{\norm{v}^2_{0,h}}\frac{\norm{v}^2_{0,h}}{x^Tx}$, wobei $v=\sum_{i=1}^n x_i\varphi_i \in V^h$ und es gilt:
\[x^TMx=(v,v)_{L^2(\Omega)}\, ^{Satz 2.2.1,i)} \Rightarrow\, (*) \, c_1 \leq \frac{x^TMx}{\norm{v}^2_{0,h}} \leq c_2 \, \forall x \in \mathbb{R}^n \]
Weiterhin ergibt sich für $x=(v(a_1),\dots,v(a_n))^T$:
\begin{align*}
(**) \, \min_{T \in \tau_h} h^d_T \norm{x}^2_{l_2} &= \min_{T \in \tau_h} h^2_T \sum_{i=1}^n (v(a_i))^2 \\
&= \sum_{T \in \tau_h} h^2_T \sum_{i=1}^n (v(a_i))^2 = \norm{v}^2_{0,h} \\
&=  \sum_{T \in \tau_h} h^2_T \sum_{i=1}^n (v(a_i))^2 \leq c_A h^d \norm{x}^2_{l_2}
\end{align*}
mit $h:= \max_{T \in \tau_h} |h_T| $.
\begin{align*}
^{(*),(**)}\Rightarrow& c_A \min_{T \in \tau_h} h^d_T \leq \frac{x^TMx}{x^Tx} \leq c_2c_1 h^d\\
\Rightarrow & \lambda_{min}(M) \geq c_A \min_{T \in \tau_h} h^d_T, \, \lambda_{max} (M) \leq c_2c_1 h^d \\
\Rightarrow & Beh. 
\end{align*}
\end{proof}
\underline{Bemerkung:} Für eine Familie quasi-uniformer Triangulierungen $(\tau_h)_h$ gilt nach Def.:\\
Es existiert eine Konstante $c>0$, s.d. 
\[ h:= \max_{T \in \tau_h} h_T \leq c h_T \, \forall  T \in \tau_h \, \Rightarrow \kappa (M) \leq \tilde C \]
$\rightarrow$ Für quasi-uniforme Triangulierungen ist die Massenmatrix gut konditioniert bzw. gleichmäßig beschränkt.\\

\begin{satz}%2.2.4
  Sei $K$ die Steifigkeitsmatrix und $M$ die Masseenmatrix. Dann existiert eine von $h$ unabhängige Konstante $c>0$, so dass für die spektrale Konditionszahl folgendes gilt:
  \begin{enumerate}
    \item $\kappa(M^{-1}K) \leq c(\min_{T\in\tau_h} h_T)^{-2}$
    \item $\kappa(k) \leq c(\min_{T\in\tau_h} h_T)^{-2} \kappa(M)$
  \end{enumerate}
\end{satz}

\begin{proof}[``Spielen mit dem Rayleigh Quotienten'']
  Für $x\in\R^n$ gilt
  \[
    \frac{x^TKx}{x^Tx} = \underbrace{\frac{x^TKx}{x^TMx}}_{\underset{y=M^{1/2}x}{=}\frac{y^TM^{-1/2}KM^{-1/2}y}{y^Ty}} \frac{x^TMx}{x^Tx}
  \]
  Es genügt somit zu Zeigen: $\exists c_1, c_2 >0:$
  \[
    c1 \leq \frac{x^TKx}{x^TMx} \leq c_2
  \]
  mit $K=(a(\varphi_i, \varphi_j)_{i,j})$ wie gehabt. 

  Es gilt für $x=(v(a_1),\ldots, v(a_n))^T$:
  \[
    \frac{x^TKx}{x^TMx} = \frac{a(v,v)}{(v,v)_{L^2(\Omega)}}
  \]

  \underline{V-Elliptisch:} $\forall v\in V^h: a(v,v) \geq \alpha \norm{v}^2_{H^1(\Omega)} \geq \alpha \norm{^2_{L^2(\Omega)}}$
  Aus der \underline{Stetigkeit} folgt: $\forall v\in V^h: a(v,v) \leq C \norm{v}^2_{H^1(\Omega)} \underset{\text{inv. Ungl.}{\leq}} c(\min_{T\in\tau_h} h_T)^{-2} \norm{v}^2_{L^2(\Omega)}$
  \[
    \Rightarrow \alpha \leq \frac{a(v,v)}{(v,v)_{L^2(\Omega)}} = \frac{x^TKx}{x^TMx} = \frac{a(v,v)}{(v,v)_{L^2(\Omega)}} \leq c(\min_{T\in\tau_h}h_T)^{-2}
  \]
  $\Rightarrow$ Behauptung
\end{proof}

\underline{Bemerkungen:}
\begin{enumerate}
\item
Für die Konditionszahl der Steifigkeitsmatrix einer regulären Familie von Triangulierungen gilt:
\[ \kappa (K)  \leq C \frac{h^d}{(\min_{t \in \tau_h} h_T)^{d+2}} \]
Für quasi-uniforme Triangulierungen gilt:
\[\kappa (K) \leq C \frac{1}{h^2} \]
\item
Das (nicht vorkonditionierte) CG-Verfahren konvergiert als mit einer Konvergenzrate von $\mathcal{O}(\sqrt{\kappa})=\mathcal{O}(h^{-1})$. \\
$\Rightarrow$ zunehmend schlechtere Konvergenz bei Gitterverfeinerung \\
$\Rightarrow$ Konstruktion effizienter Vorkonditionierer!
\end{enumerate}
\chapter {Das klassische FETI-Verfahren}

\underline{Ziel:} Entwurf eines Vorkonditionierers bzw. eines vorkonditionieren CG-Verfahrens, welches besser konvergiert als CG.\\
$\rightarrow$ Finite Element Tearing Interconnecting \\
\underline{Notation:} Statt $a(x)$ im Modellproblem: $\rho (x)$\\

\subsection{Der Algotrithmus}

\begin{enumerate}
\item
Zerlege $\Omega$ in $N$ nicht-überlappende Teilgebiete $\Omega_i \subset \Omega,\, i=1,\dots N \, : \, \bar \Omega = \Cup_{i=1}^N \bar \Omega_i $. $ \Gamma := \Cup_{i=1}^N \partial \Omega_i \setminus \partial \Omega$ wird \underline{Interface} genannt.\\
Die Zerlegung soll so vorgenommen werden, dass die FE-Knoten (der verschiedenen $\Omega_i$) auf dem Interface übereinstimmen.
\item
Für alle Teilgebiete $\Omega_i$ werden lokale Steifigkeitsmatrizen $K^{(i)}$ und lokale Lastvektoren $f^{(i)}$ aufgestellt; $i=1,\dots N$\\
\underline{Notation:}
\[ K:= \begin{pmatrix}  K^{(1)} & 0 & 0 & \cdots & 0 \\
				0 & K^{(2)}   & 0 & \cdots  &0 \\
				\vdots\\
				 0 & 0 & \cdots & 0 & K^{(N)}
	\end{pmatrix}, \,
  f:= \begin{pmatrix} f^{(1)} \\ \vdots \\ f^{(N)} \end{pmatrix}, \, u:=  \begin{pmatrix} u^{(1)} \\ \vdots \\ u^{(N)} \end{pmatrix}
\]
Formal kann man die Gleichung 
\begin{equation}
Ku=f \quad (*)
\end{equation}
hinschreiben. Da wir auf dem Interface die Steifigkeitsmatrizen $K^{(i)}$ nicht assembliert haben, ist (*) nur bedingt sinnvoll $\rightarrow$ Übungsblatt!\\
Die Lösung von (*), sofern sie existert, ist auf dem Interface merhwertig und im Allgemeinen müssen diese Werte nicht übereinstimmen. Die Lösung von (*) hat auf dem Interface im Allgemeinen Sprünge.\\
\underline{Naheliegend:} Auf dem Interface wird die \textbf{Stetigkeit der Lösung} als Nebenbedingung zusätzlich eingeführt.




%JERO


Für die inneren Knotenwerte $u_I$ müssen keine Bedingungen vorgeschrieben werden. Es gilt also:
\[ B=(b_{ij})_{i,j}\, , \, b_{ij}\in \{ 0,1,-1 \} \text{ und } 0 = (Bu)_k=u^{(i)}_k - u^{(j)}_k \]
bei entsprechender Nummerierung.\\
Sei $W^h(\Omega_i)$ der zu $\Omega_i$ gehörige FE-Raum. Das ursprüngliche FE-Problem, welches aus dem Modellproblem entsteht, ist dann äquivalent zu folgendem Minimierungsproblem:
Fionde $u^* \in W^h := \prod_{i=1}^N W^h (\Omega_i)$, s.d.
\[(3.1) \quad J(u^*) = \min_{v \in W^h, \, Bv=0} J(v) \text{ wobei } J(v):=\frac{1}{2} \langle Kv,v\rangle - \langle f,v \rangle\, , \, \langle v , w \rangle := x^Tv \]
Mit Hilfe Lagrangescher Mulitplikatoren überführen wir (3.1) in ein gemischtes System. Die zugehörige Lagrangefunktion lautet:
\[ \mathcal{L}(u,\lambda)=J(u)+\langle Bu,\lambda \rangle \]
\[^{\text{ notw. Bed.}\Rightarrow \nabla_u \mathcal{L} (u,\lambda)=0,\, \nabla_\lambda \mathcal{L} (u,\lambda)=0  \]
\[\Rightarrow \begin{cases} Ku-f+B^T\lambda &=0 \\ Bu &= 0 \end{cases} \]
\[ \Leftrightarrow \begin{pmatrix} K & B^T \\ B & 0 \end{pmatrix} \begin{pmatrix}u \\ \lambda \end{pmatrix} = \begin{pmatrix} f \\ 0 \end{pmatrix}  \boxed{\text{FETI-MASTERSYSTEM}} \]
Wir betrachten das gemischte System:\\
Finde $(u,\lambda) \in W^h \times U $ mit $U=\text{range}(B)$
\begin{equation}
(3.2) \begin{aligned}Ku+B^T\lambda &= f \\ Bu &= 0 \end{aligned}
\end{equation}
Das Problem ist lösbar, wenn
\[ \text{ker}(K)\cap \text{ker}(B) = \{ 0 \} \]
Dies ist sicherlich der Fall!\\
Angenommen $K$ invertierbar:
\[ \begin{pmatrix} K & B^T \\ B & 0 \end{pmatrix} \rightarrow  \begin{pmatrix} K & B^T \\ 0 & -BK^{-1}B^T \end{pmatrix} = \begin{pmatrix}   K^{(1)} & 0 & 0 & \cdots & 0 & B^{(1)}^T \\
				0 & K^{(2)}   & 0 & \cdots  &0 & B^{(2)}^T\\
				\vdots\\
				 0 & 0 & \cdots & 0 & K^{(N)} & B^{(N)}^T \\
			           &     &      0    &   &            &  \underbrace{\sum_{i=1}^N B^{(i)}{K^{(i)}}^{-1} B^{(i)}}_{\text{Summe lok. Schurkomp.}}
				\end{pmatrix}
\]
Diese Vorgehensweise ist i. Allg. so nicht möglich, da die $K^{(i)}$ i- Allg. pos. semi-definit.\\
(Bsp.: Steifigkeitsmatrix mit hom. Neumannbed.)\\
Daher betrachten wir in (3.2):
\[ Ku=f-B^T\lambda \text{ ist lösbar } \Leftrightarrow f-B^T\lambda \in \text{range}(K) \]
\[ \Leftrightarrow (f-B^T\lambda) \orth \text{ker}(K) \]
\underline{Ann.:}
\[ f-B^T\lambda \in \text{range}(K);\, K^+ : \text{ Pseudoinverse von } K: KK^+K=K \]
Mit Hilfe der Pseudoinversen:
\[(3.3)\quad  u=K^+(f-B^T\lambda)+R\alpha  \]
falls $f-B^T\lambda \in \text{range}(K)$ und $R$ matrix mit vollem Rang mit $\text{range}(R)= \text{ker}(K)$ und $\alpha$ geeignet gewählt.\\
Einsetzen von (3.3) in $Bu=0$ ergibt:
\[ \underbrace{BK^+ B^T}_{=:F} \lambda = \underbrace{BK^+ f}_{=:d} + \underbrace{BR}_{=:G}\alpha \]
\[ \Leftrightarrow \, (3.4) \quad \boxed{ F\lambda= d + G\alpha } \text{ mit geeignetem } $\alpha$ \]
Sei $Q$ eine sym. pos. def. Matrix (z.B. $Q=I$). Dann definieren wir eine Orthogonalprojektion (im $Q$-inneren Produkt)
\[ P:=I-QG(G^TQG)^{-1}G^T \text{ auf } V:= \text{ker}(G^T) \]
Anwendung von $P^T$  auf (3.4):
\[ P^TF\lambda = P^Td + \underbrace{(I-G(G^TQG)^{-1}G^TQ)G\alpha}_{=G\alpha - G\alpha=0} \]
\[ \Leftrightarrow \boxed{P^TF\lambda = P^T d } \]
Alle Umformungen wurden gemacht unter der Vorr., dass $ (f-B^T\lambda)\orth \text{ker}(K)=\text{range}(R)$ \\
Es gilt:
\begin{align*}
 & (f-B^T\lambda)\orth \text{range}(R) \\
\Leftrightarrow & (f-B^T\lambda) \in \text{ker} (R^T) \\
\Leftrightarrow & R^T(f-B^T\lambda)=0 \\
\Leftrightarrow & G^T\lambda = R^Tf =:e
\end{align*}
Aus (3.2) erhalten wir
\[ \boxed{ \begin{aligned} P^TF\lambda &= P^Td \\ G^T\lambda &= e  \end{aligned}} \, (3.5) \]







\end{enumerate}• 
Damit haben das zu lösende System
\[ \boxed{P^TF\lambda =P^T d,\, G^T\lambda =e} \quad (3.5)  \]

nehmen wir an, dass (3.5) lösbar ist. Dann erfüllt $\lambda$ mit 
\[ \alpha := (G^TQG)^{-1} G^TQ (F\lambda -d) \]
die Gleichung (3.4):
\[ F\lambda = d+G\alpha \]
denn
\[ F\lambda = \d+G\alpha = d + \underbrace{G  (G^TQG)^{-1} G^TQ}_{=: (I-P^T)} (F\lambda -d) = d + (I-P^T)(F\lambda -d) \]
\[ \Leftrightarrow 0= -P^TF\lambda +P^Td \]
Hieraus berechnet man 
\[ u=K^+ (f-B^T\lambda) + R(G^T Q G)^{-1} G^TQ (F\lambda -d) \]
Weiterhin betrachten wir folgende Räume:
\[V:= \{ \lambda \in U :=\text{range} (B) \, :\,  \langle \lambda, Bz \rangle =0 \, \forall \, z \in \text{ker}(K) \} = \text{ker} (G^T) \]
\[V' := \{ \mu \in U \, :\,  \langle \mu, Bz \rangle_Q =0 \,  \forall \, z \in \text{ker}(K) \} = \text{ker} (P^T) \]
Das  FETI-Verfahren ist nun das vorkonditionierte CG-Verfahren in $V$ angewendet auf
\[ P^TF\lambda = P^T d \text{ für } \lambda \in \lambda_0 +V \]
mit einem Startvektor $\lambda_0: \, G^T\lambda_0=e $.\\

Es gilt tatsächlich:
\begin{align*}
  &	\lambda \in \lambda_0 + V \\
  \Leftrightarrow & \lambda=\lambda_0 + \lambda_1 \text{ mit } \lambda_1 \in V=\text{ker}(G^T) \\
  \Rightarrow & G^T\lambda = \underbrace{G^T \lambda_0}_{=e} + \underbrace{G^T \lambda_1}_{=0} =e
\end{align*}
Dafür muss über den Vorkonditionierer noch gewährleistet bleiben, dass die Iterierten in $V$ bleiben:\\
Sei dazu $PM^{-1}$ der noch zu wählende Vorkonditionierer, dann gilt es noch
\[ PM^{-1}P^TF\lambda = PM^{-1} P^T d \, \lambda \in \lambda_0 + V \]
mit PCG zu löden, wobei $G^T\lambda_0 = e$.\\
\underline{Bemerkung:}\\
Für $\lambda \in V=\text{range} (P)$:
\begin{align*}
PM^{-1}P^TF\lambda = PM^{-1}P^T FP \lambda \underbrace{=}_{(P^T)^2=P^T} (PM^{-1}P^T)(P^TFP)\lambda
\end{align*}
Daher kann man formal $PM^{-1}P^Tf$ auch als Produkt der zwei sym. Matrizen $( (PM^{-1}P^T)$ und $(P^TFP)$ ansehen.\\
$\rightarrow$\underline{Später:} Diese sind auch positiv definit!\\
\\

\underline{\textbf{Ein einfacher Vorkonditionierer:}}\\
Unterteilen wir die Freiheitsgrade in jedem Teilgebiet $\Omega_i$ in Innere, $u^{(i)}_I$, und solche auf dem Interface, $u^{(i)}_\Gamma$, dann ergibt sich folgende Partitionierung:
\[K^{(i)}=
  \begin{pmatrix} 
    K^{(i)}_{II} & K^{(i)}_{I\Gamma} \\
    K^{(i)}_{\Gamma I} & K^{(i)}_{\Gamma \Gamma} 
  \end{pmatrix} 
 \longrightarrow^{Block-GE} 
 \begin{pmatrix} 
   K^{(i)}_{II} & K^{(i)}_{I\Gamma} \\
   0 & \underbrace{(K^{(i)}_{\Gamma \Gamma}-K^{(i)}_{\Gamma I} (K^{(i)}_{II})^{-1} K^{(i)}_{I \Gamma })}_{=: S^{(i)}_{\Gamma \Gamma}=: S^{(i)}}
\end{pmatrix} 
\]
\begin{definition}
  $S:= \text{diag} (S^{(i)}) $. (Die $S^{(i)}$ von $i=1$ bis $i=N$ liegen auf der Diagonalen). Dann lautet der einfachste (Dirichlet)-Vorkonditionierer, den Farheit, Mandel und Roux (1994) vorschlagen, wie folgt:
  \[M^{-1} := B_\Gamma S B^T_\Gamma = \sum_{i=1}^N B^{(i)}_\Gamma S^{(i)} {B^{(i)}_\Gamma}^T \text{ wobei } B=(0 \, B_\Gamma ) \]
\end{definition}

\begin{algorithmus}
\underline{Initialisierung:}\\
\begin{align*}
  \lambda^{(0)} &:= QG(G^TQG)^{-1} e + \mu \\
  r^{(0)} &:= d- F \lambda^{(0)}
\end{align*}
\underline{Iteration:} $k=1,2,...$ bis Konv. : \\
\begin{align*}
  \text{\underline{Projektion:}} & q^{(k-1)} := P^T r^{(k-1)} \\
  \text{\underline{Vorkondition:}} & z^{(k-1)} := M^{-1} q^{(k-1)} \\
  \text{\underline{Projektion:}} & y^{(k-1)} := P z^{(k-1)}
\end{align*}
\begin{align*}
  \beta^{(k)} &= \frac{\langle y^{(k-1)} , q^{(k-1)} \rangle}{\langle y^{(k-2)} , q^{(k-2)} \rangle}, \, (\beta^{(1)}:=0 ) \\
  p^{(k)} &= y^{(k-1)} + \beta^{(k)}p^{(k-1)} , \, (p^{(1)}:=y^{(0)}) \\
  \alpha^{(k)} &:= \frac{\langle y^{(k-1)} , q ^{(k-1)} \rangle}{ \langle p^{(k)} , F p^{(k)} \rangle } \\
  \lambda^{(k)} &:= \lambda^{(k-1)} + \alpha^{(k)}p^{(k)} \\
  r^{(k)} &:= r^{(k-1)} - \alpha^{(k)}Fp^{(k)}
\end{align*}
\end{algorithmus}
Folgende Fragen:
\begin{itemize}
  \item
    Wie wählt man $M^{-1}$?
  \item
    Wie wählt man $B$?\\
    (redundant vs. nicht-redund. L.M.)
  \item
    Konditionsabhängigkeit bzw. Konvergenzanalyse?
\end{itemize}

\subsection{Nicht-redundante Lagrangesche Multiplikatoren}

Die Wahl des Sprungoperators $B$ ist nicht eindeutig! \\
Diese Fälle führen auf verschiedene Sprungoperatoren. Der nicht-redundante Fall (jeder Randpunkt wird nur mit einem anderen Randpunkt bzgl. Steitgkeit verglichen) führt auf ein $B$ mit vollem Rang und der redunate Fall (alle randkntoen werden entsprechend verglichen) nicht. \\
Vorteil des redundanten Falls: Man muss beim Programmieren keine spezielle Wahl der L.M. treffen. man minimiert einfach alle.\\
\textbf{Ab jetzt. nicht-redundante L.M. }\\

Wir betrachten folgenden Vorkonditionierer:
\[\hat M^{-1} := (B_\Gamma D^{-1} B^T_\Gamma )^{-1} B_\Gamma S B^T_\Gamma (B_\Gamma D^{-1} B^T_\Gamma)^{-1} \]
(Klawonn/Wildkunst, 2001). $D$ ist Diagonalmatrix mit pos. Diagonalelementen.\\
$\Rightarrow \, (B_\Gamma D^{-1}B^T_\Gamma)^{-1} $ lässt sich numerisch günstig berechnen.\\
Um ein Verfahren zu bekommen, welches möglichst unabh. von Sprüngen in den Koeff. der DGL konvergiert, def. wir $D$ wie folgt:
\[ D= \text{diag}\left( D^{(1)},\dots D^{(N)} \right) \]
wobei die $D^{(i)}$ auf den Knoten von $\partial \Omega_i$ operieren, d.h. jeder Eintrag gehört zu einem Knoten $x \in \partial \Omega_{i,h}$.\\
Dieser wird def. als:
\[ \delta^+_i := \delta^+_i (x) := \rho^\gamma_i(x)\mu^+_i(x), \quad \gamma \in [1/2,\infty) \]
(normalerweise $\gamma=1$), $\mu^+_i(x) :=\frac{1} { \sum_{j \in N_x} \rho^\gamma_j (x)} ,\, N_x := \{ j \in \{ 1,\dots N\} : x \in \Omega_{j,h} \}$.\\
\underline{Beispiele}\\
\[ (zeichnung 2 knoten)  N_x=\{i,j\} , \delta^+_i(x)=\frac{\rho^\gamma_i(x) }{\rho^\gamma_i(x)  + \rho^\gamma_j(x) } \]
\[ (Zeichnugn 4 Knoten) N_x=\{i,j,k,l \}, \,  \delta^+_i(x)=\frac{\rho^\gamma_i(x) }{\rho^\gamma_i(x)  + \rho^\gamma_j(x) + \rho^\gamma_k(x)  + \rho^\gamma_l(x) } \]








%\setcounter{chapter}{1}
%\setcounter{section}{2}
%\setcounter{amssatz}{10}
%\setcounter{equation}{16}



%\[
  K^{(i)}=:K=
  \begin{pmatrix}
    K_{II}& K_{I\Gamma}\\
    K_{\Gamma I}& K_{\Gamma\Gamma}
  \end{pmatrix}
  =
  \begin{pmatrix}
    I& 0\\
    K_{\Gamma I} K_{II}^{-1}& I
  \end{pmatrix}
  \begin{pmatrix}
    K_{II}& 0\\
    0& S_{\Gamma\Gamma}
  \end{pmatrix}
  \begin{pmatrix}
    I& K_{II}^{-1}K_{I\Gamma}\\
    0& I
  \end{pmatrix}\\
  \text{mit } S_{\Gamma\Gamma}=K_{\Gamma\Gamma}-K_{\Gamma I}K_{II}^{-1}K_{I\Gamma}
\]
$K^{(i)}$ ist positiv semidefinit $\underset{\overset{\text{Silvester}}{\text{Trägheitssatz}}}{\Longrightarrow}$ $S_{\Gamma\Gamma}=S_{\Gamma\Gamma}^{(i)}$ auch positiv semidefinit. Dazu mache man sich klar, dass $K_{II}$ positiv definit ist. Somit können wir $K^+$ wie folgt darstellen:

\[
  (*) K^+ = 
  \begin{pmatrix}
    I& K_{II}^{-1}K_{I\Gamma}\\
    0& I
  \end{pmatrix}
  \begin{pmatrix}
    K_{II}^{-1}& 0\\
    0& S_{\Gamma\Gamma}^+
  \end{pmatrix}
  \begin{pmatrix}
    I& 0\\
    -K_{\Gamma I}K_{II}^{-1}& I
  \end{pmatrix}\\
  \text{mit } S_{\Gamma\Gamma}S_{\Gamma\Gamma}^+S_{\Gamma\Gamma}=S_{\Gamma\Gamma}
\]

\begin{align*}
  \Rightarrow KK^+K &= 
  K
  \begin{pmatrix}
    I& -K_{II}^{-1}K_{I\Gamma}\\
    0& I
  \end{pmatrix}
  \begin{pmatrix}
    K_{II}^{-1}& 0\\
    0& S_{\Gamma\Gamma}^+
  \end{pmatrix}
  \begin{pmatrix}
    I& 0\\
    -K_{\Gamma I}K_{II}^{-1}& I
  \end{pmatrix}
  K\\
  &=
  K 
 \begin{pmatrix}
    I& -K_{II}^{-1}K_{I\Gamma}\\
    0& I
  \end{pmatrix}
  \begin{pmatrix}
    K_{II}^{-1}& 0\\
    0& S_{\Gamma\Gamma}^+
  \end{pmatrix}
  \begin{pmatrix}
    K_{II}& 0\\
    0& S_{\Gamma\Gamma}
  \end{pmatrix}
  \begin{pmatrix}
    I& K_{II}^{-1}K_{I\Gamma}\\
    0& I
  \end{pmatrix}\\
  &=
  \begin{pmatrix}
    I& 0\\
    K_{\Gamma I}K_{II}^{-1}& I
  \end{pmatrix}
  \begin{pmatrix}
    K_{II}& 0\\
    0&  \underbrace{S_{\Gamma\Gamma}S_{\Gamma\Gamma}^+S_{\Gamma\Gamma}}_{=S_{\Gamma\Gamma}}
  \end{pmatrix}
  \begin{pmatrix}
    I& K_{II}^{-1}K^{I\Gamma}\\
    0& I
  \end{pmatrix}\\
  &= K
\end{align*}


und somit ist $K^+$, wie in (*) definiert, eine Pseudoinverse von K.

CHIRSTOPHER

Sei $\hat{W}$ der Unterraum von $W$, dessen Elemente auf $\Gamma$ stetig sind. 

\begin{lemma}%3.2.2
  Für alle $w\in W$ gilt: 
  \[
    E_{D}w := w-P_Dw = (I-P_D)w \in \hat{W}
  \]
  d.h. $E_D w$ ist stetig über $\Gamma$ hinweg.
\end{lemma}

\begin{proof}
 Beweis über die Sprungerhaltende Eigenschaft. 
 \[
   BE_Dw=Bw-BP_Dw \underset{P_D\text{ sprungerh.}}{=} Bw-Bw =0
 \]
\end{proof}

\begin{lemma} %3.2.3
  Sei $w\in W$, dann ist $E_Dw(x)$ für $x\in\Gamma^h$ (also ein FE-Knoten auf $\Gamma$) der $D$-gewichtete Mittelwert aller Werte von $w$ in $x$
\end{lemma}

\begin{proof}
  Für ein $w\in W \overset{\text{Lemma 3.2.2}}{\Longrightarrow} E_Dw\in \hat{W}$ (also stetig). Sei $x\in\Gamma^h$ und $e_x\in\hat{W}$ wie folgt definiert:
  \[
    e_x(y) := 
    \begin{cases}
      1& x=y\\
      0& \text{sonst}
    \end{cases},
    \quad B_\Gamma e_x =0
  \]
  Der $D$-gewichtete Mittelwert von $w(x)$ ist wie folgt: (D ist eine Blockdiagonalmatrix; die Blöcke haben entweder x vertreten, oder gar nicht)
  \begin{align*}
    e_x^TDP_Dw = (P_D^TDe_x)^Tw =(B^T(BD^{-1}B^T)^{-1} \underbrace{B\underbrace{D^{-1}D}_{=I}e_x}_{=0})^Tw=0\\
    \Rightarrow e_x^TD_Dw = e_x^TDw=D\text{-gewichteter Mittelwert von W in x}
  \end{align*}
\end{proof}

Es gilt also 
\begin{equation}%3.6
  E_Dw = \sum_{x\in \Gamma^h} \langle e_x, Dw \rangle e_x
\end{equation}

Da $E_D$ nicht von der Wahl von $B$ abhängt, gilt dies auch für $P_D$!\\
Für $P_Dw=:v=(v_i)_{i=1,\ldots,N}\in W$ ergibt sich aus (3.6) folgende Beziehung für $x\in \partial\Omega_{i,h}, i=1,\cdots,N$
\[ %Zentraler Zugang zu dem analytischen Beweis von \ldots {BOXED?}
  (P_Dw(x))^{(i)} = v^{(i)}(x) = \sum_{j\in \mathcal{N}_x} \delta_j^+(x) (w^{(i)}(x)-w^{(j)}(x))
\]

Für unsere Analysis benötigen wir folgende Norm auf $V'=\range(P^T)$. Sei $\mu\in V'$, dann definieren wir
\[%seminorm, weil wir nicht triviale Kerne haben
  \norm{\mu}_{V'}^2 := |D^{-1}B^T(BD^{-1}B^T)^{-1}\mu|_S^2 = \langle\hat{M}^{-1}\mu,\mu \rangle
\]
mit $|u|_S^2:=\langle u,u \rangle_S = u^TSu$ und $S$ Seminorm auf $W$

CHIRSTOPHER

%Auf $V$ führen wir folgende Norm ein 
\begin{definition}
  \[
    \norm{\lambda}_V:=\sup_{\mu\in V'} \frac{\langle \lambda, \mu\rangle}{\norm{\mu}_{V'}}
  \]
  Aus $\norm{\mu}_{V'}^2 = \langle \hat{M}^{-1}\mu, \mu\rangle$ folgt mit direktem Rechnen
  \[
    \norm{\lambda}_V^2 = \langle \hat{M}\lambda,\lambda\rangle \\
    \Rightarrow P^T\hat{M}: V\to V's.p.d.
  \]
\end{definition}

\begin{lemma}
  Für jedes $w\in W$ exisitiert ein eindeutig bestimmtes $z_w\in \ker(S), s.d. B(w+z_w)\in V'$
  Weiterhin gilt: %(Der Sprung von w und zw ist in V')
  \[
    \norm{Bz_w}_Q \leq \norm{Bw}_Q
  \]
  \label{}
\end{lemma}

\begin{proof}
  \[
    B(w+z_w)\in V' \overset{\text{Def. } V'}{\Leftrightarrow} \langle B(w+z_w), Bz\rangle_Q = 0 \quad \forall z\in \ker(S)
  \]

  Diese Galerkinbedinung erlaubt es $z_w\in \ker(S)$ als Lösung des Variationsproblems 
  \begin{equation}
    \label{eqn:stern}
    \tag{$*$}
    \langle B^TQBz_w,z\rangle = -\langle B^TQBw,z\rangle \quad \forall z\in\ker(S)
  \end{equation}
  zu betrachten.\\
  
  Frage: Ist dies denn überhaupt lösbar? \\

  Da $\ker(S) \cap \ker(B) = \{0\}$ und $Q$ s.p.d., ist $B^TQB$ s.p.d. auf $\ker(S)$. Daher exisitiert eine eindeutige Lösung $z_w\in\ker(S)$, die $\norm{B(w+z_w)}_Q^2$ über $z\in\ker(S)$ minimiert. 
  
  Die Orthogonaleigenschaft \eqref{eqn:stern} garantiert auch
  \begin{align*}
    \norm{Bz_w}_Q^2 &= |\langle Bz_w,Bz_w\rangle_Q| = |-\langle Bw, Bz_w\rangle_Q|\\
    &\overset{C.S.}{\leq} \norm{Bw}_Q \norm{Bz_w}_Q\\
    \Rightarrow \norm{Bz_w}_Q \leq \norm{Bw}_Q
  \end{align*}
\end{proof}



Wir benötigen folgende Darstellung:
\begin{enumerate}
\item
$\langle F\lambda,\lambda\rangle = \sup_{w\in \text{range}(S)} \frac{\langle \lambda, Bw \rangle }{|w|^2_S}$
\item
$\langle \hat M\lambda,\lambda\rangle = \sup_{w\in W,\, Bw \in V'} \frac{\langle \lambda, Bw \rangle }{|P_D w|^2_S}$
\end{enumerate}
\begin{proof}
\begin{enumerate}
\item
\begin{align*}
\lambda \in V&=\{ \mu \in U : \, \langle \mu,Bz \rangle =0 \, \forall z \in \text{ker}(S) \} \\
 \Rightarrow &  \, B^T\lambda \in\text{ range } (S) , \, S^+ =: S^{-1/2}S^{-1/2} \\
\Rightarrow & \, S^{-1/2}B^T\lambda \text{ ist wohldefiniert}
\end{align*}
Dann gilt:
\begin{align*}
 \langle F\lambda, \lambda \rangle &= \langle S^+ B^T\lambda , B^T \lambda \rangle = \langle S^{-1/2} B^T\lambda , S^{-1/2}B^T \lambda \rangle \\
&= || S^{-1/2} B^T \lambda||^2_{l_2} \\
&= \sup_{v \in \text{range} (S)} \frac{\langle S^{-1/2}B^T\lambda,v\rangle^2}{||v||^2_{l_2}} \quad \text{(Dualnorm)} \\
&= \sup_{w \in \text{range} (S)} \frac{\langle B^T\lambda,w \rangle^2}{|w|^2_S} 
\end{align*}
\end{enumerate}
\begin{align*}
\langle \hat M \lambda , \lambda \rangle &= ||\lambda||^2_V = \sup_{\mu \in V'} \frac{\langle \lambda , \mu \rangle^2}{||\mu ||^2_{V'}} \\
&= \sup_{\mu \in V'} \frac{\langle \lambda , \mu \rangle^2}{\langle {\hat M}^{-1} \mu, \mu \rangle} \\
&= \sup_{w \in W,\, Bw \in V'}	 \frac{\langle \lambda , Bw \rangle^2}{|P_D w|^2_S}
\end{align*}
wobei wir benutzt haben:
\[ \langle {\hat M}^{-1} \mu, \mu \rangle \underbrace{=}_{\mu =Bw} = \langle (BD^{-1}B^T)^{-1} B D^{-1} S \underbrace{D^{-1} B^T (B D^{-1} B^T)^{-1} B}_{=P_D w}w, Bw \rangle \]
\end{proof}
Wir benötigen folgende Abschätzung für $P_D$:
\begin{lemma}(3.2.6)\\
Sei $w \in \text{ range}(S)$, dann gilt:
\[ |P_D w |^2_S \leq C(1+\text{log} \left( \frac{H}{h}\right))^2|w|^2_S \]
Hierbei ist $C$ eine positive Konstante unabhängig von $H,h$ und $S$.
\end{lemma}
\begin{proof}
später in der Vorlesung!
\[ H_i := \text{diam} (\Omega_i),\,  h^{(i)}_j := \text{diam} T^{(i)}_j,\, h_i := \max_j h^{(i)}_j , \, \frac{H}{h} := \max_j \left( \frac{H_i}{h_i} \right) \]
\end{proof}

%CHRISTOPHER

\begin{proof}
  Es genügt folgendes zu zeigen: 
  \begin{align*}
    \langle \hat{M}\lambda,\lambda\rangle &\underset{*}{\leq} \langle F\lambda,\lambda\rangle\\
    &\leq C(1+\log \frac{H}{h})^2 \langle \hat{M}\lambda,\lambda\rangle \quad \forall \lambda\in V
  \end{align*}
  \[
    P^TF\lambda=\mu I\hat{M}\lambda, \quad 
    \mu_{max} = \max_{\lambda\in V} \frac{\langle F\lambda,\lambda\rangle}{\langle\hat{M}\lambda,\lambda\rangle}
    \lambda_{min}(P\hat{M}^{-1}P^TF) \geq 1
    \lambda_{max}(P\hat{M}^{-1}P^F) \leq C(1+\log \frac{H}{h})^2
  \]

  Zur Erinnerung: Sei $\lambda\in V$:
  \begin{enumerate}
      \item
        \[
          \langle F\lambda,\lambda\rangle = \sup_{w\in\range(S)} \frac{\langle\lambda,Bw\rangle^2}{|w|_S^2}
        \]
      \item
        \[
          \langle\hat{M}\lambda,\lambda\rangle = \sup_{\lambda\in V'} \frac{\langle\lambda,\mu\rangle^2}{\norm{\mu}_{V'}^2}
        \]
      \item 
        \[
          \norm{\mu}_{V'}^2 = \langle\hat{M}^{-1}\mu,\mu\rangle
        \]
  \end{enumerate}

  Untere Schranke (*): 
  Sei $\mu\in V'$ beliebig. Aus Lemma 3.2.1 folgt, dass eine $\tilde{w}\in\range(P_D)$ existert, so dass $\mu=B\tilde{w}$. Sei nun $\tilde{w}^{\perp}$ die Komponennte von $\tilde{w}$, die orthogonal zu $\ker(S)$ ist, d.h. $\tilde{w}=\underbrace{\tilde{w}^{\perp}}_{\in\range(S)} + \underbrace{\tilde{w}^{(0)}}_{\in\ker(S)}$

  Dann gilt: 
  \begin{enumerate}
    \item $\langle S\tilde{w},\tilde{w}\rangle = \langle S\tilde{w}^{\perp},\tilde{w}^{\perp}$
    \item $\langle\lambda,B\tilde{w}\rangle = \langle\lambda,B\tilde{w}^{\perp} \quad \forall \lambda\in V = \left\{ \nu\in U: \langle\nu,Bz\rangle =0 \forall z\in\ker(S) \right\}$
  \end{enumerate}

  Für $\lambda\in V$ beliebig folgt 
  \begin{align*}
    \langle F\lambda,\lambda\rangle &= \subset_{\mu\in \range{S}} \frac{\langle\lambda,Bw\rangle^2}{|w|_S^2} \geq \frac{\rangle\lambda,B\tilde{w}^{\perp}\rangle^2}{|\tilde{w}^{\perp}|_S^2}\\
    &\underset{1),2)}{=} \frac{\langle\lambda,B\tilde{w}\rangle^2}{|\tilde{w}|_S^2} \underset{\tilde{w}=P_D\tilde{w}} \frac{\langle\lambda,B\tilde{w}\rangle^2}{|P_D\tilde{w}|_S^2}
    &= \frac{\langle\lambda,\mu\rangle^2}{|D^{-1}B^T(BD^{-1}B^T)^{-1}\mu|_S^2} = \frac{\langle\lambda,\mu\rangle^2}{\langle\hat{M}^{-1}\mu,\mu\rangle}
    &= \frac{\langle\lambda,\mu\rangle^2}{\norm{\mu}_{V'}^2}
  \end{align*}
  Da $\mu\in V'$ beliebig gewählt war, folgt:
  \[
    \langle F\lambda,\lambda\rangle \geq \langle \hat{M}\lambda,\lambda\rangle \quad \forall \lambda\in V \text{mit} \langle\hat{M}\lambda,\lambda\rangle = \sup_{\mu\in V} \frac{\langle\lambda,\mu \rangle^2}{\norm{\mu}_{V'}^2}
  \]

  Obere Schranke (**):
Sei $w\in\range(S)$ beliebig, aber fest. Nach Lemma 3.2.5. $\exists! z_w\in\ker(S)$, so dass $B(w+z_w)\in V'$
\[
  \underset{Lemma 3.2.6,3.2.7}{\Longrightarrow} |P_D(w+z_w)|_S^2 \leq C(1+\log \frac{H}{h})^2 |w|_S^2 (***)
\]
Für $\lambda\in V$ folgt mit den Darstellungsformeln für $\hat{M}$ und $F$: 
\begin{align*}
  \langle F\lambda,\lambda\rangle &= \sup_{w\in\rangle(S)}\frac{\langle\lambda,Bw\rangle^2}{|w|_S^2} \underset{(***)}{\leq} C(1+\log \frac{H}{h})^2 \sup_{w\in\range(S)\frac{\langle\lambda,Bw\rangle^2}{|P_D(w+z_w)}|_S^2}\\
  &\underset{\lambda\in V}{=} C(1+\log \frac{H}{h})^2 \sup_{w\in\range(S)} \frac{\langle\lambda,\overbrace{B(w+z_w)}^{\in V'}\rangle^2}{|_D(w+z_w)|_S^2}\\
  &\leq C(1+\log \frac{H}{h})^2 \sup_{\tilde{w}\in W\\B\tilde{w}\in V'} \frac{\langle\lambda,B\tilde{w}\rangle^2}{|P_D\tilde{w}|_S^2}\\
  &\underset{\text{Darst.} \langle\hat{M}\lambda,\lambda\rangle}{=} C(1+\log \frac{H}{h})^2 \langle\hat{M}\lambda,\lambda\rangle
\end{align*}
\end{proof}

Es bleibt noch der Beweis von Lemma 3.2.6 übrig. Dafür benötigen wir noch einige technische Hilfsmittel. Vorerst beschränken wir uns auf den $\R^2$. \\

Zunächst zerlegen wir das Interface 
\[
  \Gamma = \left( \bigcup_{i=1}^N \partial\Omega_i \right) \backslash \partial\Omega
\]
in Ecken $\nu$ und Kanten $\varepsilon$. Eine Kante ist dabei eine zusammenhängende Menge von Punkten, die alle zu genau zwei Teilgebieten gehören. Jede Kante ist eine offene Menge. Ecken sind Endpunkte von Kanten und gehören zu mehr als zwei Teilgebieten. Wir führen auf dem Interface eine Partition der Eins ein: 
Sei $\varepsilon$ eine Kante, dann definieren wir $\Teta_{\varepsilon} \in W^h(\Omega)$ durch
\[
  \Teta_{\varepsilon}(x):= 
  \begin{cases}
    1 &x\in \sum_h^n\\
    0 &x\in \Gamma_h\backslash\varepsilon_h
  \end{cases}
\]

und zusätzlich sei $\Teta_\varepsilon$ diskret harmonisch in $\Omega$. Zu einer Ecke $\nu$ sei $\Teta_\nu\in W^h(\Omega)$ die zugehörige nodale (lineare) Basisfunktion. Dann gilt: 
\[
  \sum_{\nu\in\partial\Omega_i} \Teta_\nu(x) + \sum_{\varepsilon\subset\partial\Omega_i} \Teta_\varepsilon(x) =1 \quad \forall x\in \partial\Omega_i, i=1,\cdots,N
\]
und für $w^{(i)}\inW^h(\partial\Omega_i)=: W_i$:
\[
  w^{(i)}=\sum_{\nu\in\partial\Omega_i} I^h(\Teta_\nuw^{(i)})+\sum_{\varepsilon\subset\partial\Omega_i} I^h(\Teta_\varepsilon)w^{(i)})
\],
wobei $I^h$ der FE-Interpolationsoperator nach $W^h(\Omega_i)$ ist. 


Wir benutzen hier und im Folgenden den Begriff der diskret harmonischen Fortsetzung: 
Eine Funktion $u^{(i)}\in W^h(\partial\Omega_i)$ heißt \underlinne{diskret harmonisch}, wenn für $u^{(i)}=(u_I^{(i)},u_\Gamma^{(i)})^T$ gilt:
\[
  K_{II}^{(i)}u_I^{(i)}+K_{I\Gamma}^{(i)}u_\Gamma^{(i)} = 0 
\]

%Es ist leicht zu sehen, dass $u_{\text{harm}}^{(i)}=\mathcal{H}^{(i)}(u_{\Gamma}^{i})$ vollkommen bestimmt ist durch $u_{\Gamma}^{(i)}$, denn 
\[
  u_{\text{harm},I}^{(i)} = - (K_{II}^{(i)})^{-1} K_{I\Gamma}^{(i)}u_{\Gamma}^{(i)}
\]

Diskret harmonische Funktionen haben folgende Eigenschaft

\begin{align*}
  u^{(i)}=\mathcal{H}^{(i)}(u_{\Gamma}^{(i)}): |u^{(i)}|_{K^{(i)}}^2 &= \langle K^{(i)}u^{(i)},u^{(i)}\rangle
  &= (u_I^{(i)^T}), u_{\Gamma}^{(i)^T} 
  \begin{pmatrix}
    K_{II}^{(i)} & K_{I\Gamma}^{(i)}\\
    K_{\Gamma I}^{(i)} & K_{\Gamma,\Gamma}^{(i)}
  \end{pmatrix}
  \begin{pmatrix}
    u_I^{(i)}\\
    u_\Gamma^{(i)}
  \end{pmatrix}\\
  &= u_{I}^{(i)^T} K_{II}^{(i)}u_I^{(i)}+u_I^{(i)^T}K_{I\Gamma}^{(i)}u_\Gamma^{(i)} + u_\Gamma^{(i)^T} + u_\Gamma^{(i)^T}K_{\Gamma\Gamma}^{(i)}u_\Gamma^{(i)}\\
  &\underset{(u_I^{(i)}=-K_{II}^{(i)}K_{I\Gamma}^{(i)}u_\Gamma^{(i)})}{=} %nur das gleichheitszeichen kein Umbruch
  u_\Gamma^{(i)^T}K_{\Gamma I}^{(i)}(K_{II}^{(i)})^{-1}K_{II}^{(i)} (K_{II}^{(i)})^{-1} K_{I\Gamma}^{(i)}u_\Gamma^{(i)}
  - 2u_\Gamma^{(i)^T} K_{\Gamma I}^{(i)} (K_{II}^{(i)})^{-1} K_{I\Gamma}^{(i)}u_\Gamma^{(i)} + u_\Gamma^{(i)^T}K_{\Gamma\Gamma}^{(i)}u_\Gamma^{(i)}\\
  &= u_\Gamma^{(i)^T}S_{\Gamma\Gamma}u_\Gamma^{(i)}
\end{align*}

Also $|u^{(i)}|_{K^{(i)}}^2 = |u_\Gamma^{(i)}|_{S_{\Gamma\Gamma}^{(i)}}^2$ für $u^{(i)}$ diskret harmonisch.\\

Da für unser Modellproblem gilt: $|u^{(i)}|_{K^{(i)}}^2=\rho_i |u^{(i)}|_{H^1(\Omega_i)}$

CHRISTOPHER

\begin{proof}
  Da $\Omega_i$ freies Teilgebiet ist, also $\partial\Omega_i \cap \partial\Omega_D = \emptyset$, gilt 
  \[
    l_i(w^{(i)}):=h_i^d(w^{(i)},1^{(i)})_{l_2} = 0 \quad \forall w^{(i)}\in\range(K^{(i)}), 1^{(i)}=(1,1,\cdots,1)^T\\
  \]
  \begin{equation}
    %\tag{(***)}
    l_i(w^{(i)})= 0 \Leftrightarrow w^{(i)}\in\range(K^{(i)}) 
    \label{}
  \end{equation}
  Sei nun $w^{(i)}\in W^h(\Omega_i)$, dann gilt
  \begin{align*}
    |l_i(w^{(i)})|^2 &\underset{C.S.}{\leq} h_i^d \langle w^{(i)},w^{(i)}\rangle_{l_2} h_i^d \langle 1^{(i)},1^{(i)}\rangle_{l_2}\\
    &\underset{\text{NPDGL I}} C\norm{w^{(i)}}_{L^2(\Omega_i)}^2 \norm{1^{(i)}}_{L^2(\Omega_i)}^2\\
    &= \int_{\Omega_i} (1^{(i)})^2 dx C \norm{w^{(i)}}_{L^2(\Omega_i)}^2 \leq CH_i^d \norm{w^{(i)}}_{L^2(\Omega_i)}^2\\
    &\leq C H_i^{d+2} \left( |w^{(i)}|_{H^1(\Omega_i)}^2 + \frac{1}{H_i^2} \norm{w^{(i)}}_{L^2(\Omega_i)} \right)\\
    &= C H_i^{d+2} \norm{w^{(i)}}_{H^1(\Omega_i)}^2 \Leftrightarrow |l_i(w^{(i)})| \leq CH_i^{\frac{d+2}{2}} \norm{w^{(i)}}_{H^1(\Omega_i)} \quad \forall w^{(i)}\in W^h(\Omega_i)
  \end{align*}

  Also: $l_i(\cdot): W^h(\Omega_i)\to\R$ ist ein stetiges, lineares Funktional auf dem Unterraum $W^h(\Omega_i)$ von $H^1(\Omega_i)$. Nach dem Satz von Hahn-Banach existert also eine stetige Fortsetzung auf $H^1(\Omega_i)$, die wir ohne Einschränkung auch mit $l_i(\cdot)$ bezeichnen. Mit $\nu=\text{const}$: 
  \[
    l_i(\nu)\underset{\text{Def.}}{=} h_i^d(\nu,1^{(i)})_{l_2} \underset{\nu=\text{const}}{=} h_i^d\nu(1^{(i)},1^{(i)})_{l_2} = 0 
  \]
  $\Leftrightarrow \nu=0$

  Die Anwendung von Satz 3.19 (aus dem SS13) und ein Skalierungsargument ergeben für $w^{(i)}\in W^h(\Omega_i)$
  \begin{align*}
    \norm{w^{(i)}}_{H^1(\Omega_i)}^2 &= |w^{(i)}|_{H^1(\Omega_i)}^2 + \frac{1}{H_i^2} \norm{w^{(i)}}_{L^2(\Omega_i)}^2\\
    &\undeset{Trafo}{=} \int_{\hat{\Omega}} (\nabla_{\tilde{x}}\hat{w}^{(i)})^2 H_i^{-2}H_i^d d\hat{x} + \frac{1}{H_i^2} \int_{\hat{\Omega}}(\hat{w}^{(i)})^2 H_i^d d\hat{x}\\
    &= H_i^{d-2} (|\hat{w}^{(i)}|_{H^1(\hat{\Omega})}^2 + \norm{\hat{w}^{(i)}_{L^2(\hat{\Omega})}^2})\\
    &\underset{Satz 3.19}{\leq} C_iH_i^{d-2} (|\hat{w}^{(i)}|_{H^1(\hat{\Omega})}^2 + (l_i(\hat{w}^{(i)}))^2)\\
    &\underset{\hat{w}^{(i)}(\hat{x})=w^{(i)}(x)}{\leq} C(\|w^{(i)}|_{H^1(\Omega_i)}^2 + H_i^{d-2}(l_i(w^{(i)})^2))
  \end{align*}.

  Für $w^{(i)}\in\range(k^{(i)})$ gilt mit \eqref{**} $l_i(w^{(i)})=0$ und somit gilt: $\norm{w^{(i)}}_{H^1(\Omega_i)}^2 \leq C|w^{(i)}|_{H^1(\Omega_i)}^2 \quad \forall w^{(i)}\in\range(K^{(i)})$
  \[
    \Rightarrow \frac{1}{H_i^2} \norm{w^{(i)}}_{L^2(\Omega_i)}^2 \leq C |w^{(i)}|_{H^1(\Omega_i)}^2 \quad \forall w^{(i)}\in\range(K^{(i)})
  \]
  Also $\norm{w^{(i)}}_{L^2(\Omega_i)} \leq CH_i |w^{(i)}|_{H^1(\Omega_i)} \quad \forll w^{(i)}\in\range(K^{(i)})$

\end{proof}

%VORLESUNG 13 FEHLT

%
\begin{enumerate}
  \item[1) Katenbeiträge:] Für jede Kante $\varepsilon^{ij}$ gilt mit $(*')$ und $\delta_j^+(x)$ konstant auf $\varepsilon^{ij}$
    \[
      I^h(\theta_{\varepsilon^{ij}} \nu^{(i)} = I^h(\theta_{\varepsilon^{ij}} \delta_j^+(w^{(i)}-w^{(j)}))
    \]
    Wir wollen nun folgende Abschätzung beweisen:
    \begin{align*}
      \rho_i|\mathcal{H}(I^h(\theta_{\varepsilon^{ij}}\delta_j^+(w^{(i)}-w^{(j)})))|^2_{H^1(\Omega_i)}
      &\leq C(1+\log(\frac{H_i}{h_i}))^2 \rho_i \left\{ |\mathcal{H}(w^{(i)})|^2_{H^1(\Omega_i)} + \frac{1}{H_i^2} \norm{\mathcal{H} (w^{(i)})}_{L62(\Omega_i)} \right\}  + 
      C(1+\log(\frac{H_j}{h_j}))^2 \rho_j \left\{ |\mathcal{H}(w^{(j)})|^2_{H^1(\Omega_i)} + \frac{1}{H_j^2} TAFEL \right\}
      \delta_i |\mathcal{H}(I^h(\theta_{\varepsilon^{ij}} \nu^{(i)}))|_{H^1(\Omega_i)}^2 
      = \delta_i |\mathcal{H}(I^h(\theta_{\varepsilon^{ij}}\delta_j^+(w^{(i)}-w^{(j)})))|_{H^{1(\Omega_i)}}^2
      &\underset{\delta_j^+ konst auf \varepsilon^{ij}, Lemma 3.2.13}{\leq} \min(\delta_i,\delta_j) |\mathcal{H} (I^h(\theta_{\varepsilon^{ij}}(w^{(i)}-w^{(j)})))|_{H^{1}(\Omega_i)}^2
      &\underset{Dreiecks-Ungl.}{\leq} 2 \min(\delta_i,\delta_j) \left( \underbrace{|{\mathcal{H}(I^h(\theta_{\varepsilon^{ij}}w^(i)))}|_{H^1(\Omega_i)}^2}_{=(I)}
      + \underbrace{|{\mathcal{H}(I^h(\theta_{\varepsilon^{ij}}w^{(j)}))}|_{H^1(\Omega_i)}^2}_{=(II)} \right)
    \end{align*}
    Zu (I): 
    \[
      |\mathcal{H} (I^h(\theta_{_\varepsilon^{ij}})w^{(i)})|_{H^1(\Omega_i)}^2 \underset{Lemma 3.2.1.2}{\leq} C(1+\log(\frac{H_i}{h_i}))^2 
      \left\{ |\mathcal{H}(w^{(i)})|_{H^1(\Omega_i)}^2 + \frac{1}{H_i^2} \norm{\mathcal{H}(w^{(i)})}_{L^2(\Omega_i)}^2 \right\}
    \]
    Zu (II): 
    \[
      |\mathcal{H}(I^h(\theta_{\varepsilon^{ij}}w^{(j)}))|_{H^1(\Omega_i)}^2 \leq |E_{ji}(\mathcal{H}(I^h(\theta_{\varepsilon^{ij}}w^{(j)})))|_{H^1(\Omega_i)}^2
      \underset{Fortsetzungssatz}{\leq} C|\mathcal{H}(I^h(\theta_{\varepsilon^{ij}}w^{(j)}))|_{H^1(\Omega_j)}^2 \underset{Kantenlemma}{leq} C(1+\log(\frac{H_j}{h_j})^2) \left\{ |\mathcal{H}(w^{(j)})|_{H^1(\Omega_i)}^2 + \frac{1}{H_j^2} \norm{\mathcal{H}(w^{(j)})}_{L^2(\Omega_j)}^2\right\}

    \]

    Benutzen wir nun Lemma 3.2.11 und $w^{(i)}\in range(S^{(i)}), w^{(j)}\in \range(S^{(j)})$, so erhalten wir:
    \[
      \delta_i |\mathcal{H}(I^h(\theta_{\varepsilon^{ij}})\nu^{(i)})|_{H^!(\Omega_i)}^2
      \leq C(1+\log(/\frac{H_i}{h_i}))^2 |\mathcal{H}(w^{(i)})|_{H^1(\Omega_i)}^2 + C(1+\log(\frac{H_j}{h_j}))^2 |\mathcal{H}(w^{(j)})|_{H^1(\Omega_i)}^2
    \]
\end{enumerate}

begin{proof}(zu Lemma 3.2.6)\\
\underline{zu zeigen:} \[ w \in \range (S) \, \Rightarrow \, |P_Dw|^2_S \leq C(1+ \log (H/h))^2|w|^2_S \]
Sei $w \in \range (S)$, dann $w=(w^{(i)})_{i=1,\dots,N} \, \Rightarrow w^{(i)}\in \range (S)$:\\
Sei $v := P_Dw = (v^{(i)})_{i=1,\dots,N}$. Also : \[v^{(i)}= (P_Dw)^{(i)\]
Dann gilt : 
\begin{align*}
|P_Dw|^2_S &= \sum_{i=1}^N |(P_Dw)^{(i)}|^2_{S^{(i)}} = \sum_{i=1}^N |v^{(i)}|^2_{S^{(i)}} \\
&= \sum_{i=1}^N |\mathcal{H} (v^{(i)})|^2_{K^{(i)}} \underset{=}{\text{Modellproblem}} \sum_i\rho_i |\mathcal{H} (v^{(i)})|^2_{H^1(\Omega_i)}
\end{align*}
\underline{Analog:}
\[|w|^2_S = \sum_{i=1}^N |w^{(i)}|^2_{S^{(i)}} = \sum_{i=1}^N \rho_i |\mathcal{H} (w^{(i)})|^2_{H^1(\Omega_i)} \]
Um die Abschätzung 
\[ |P_Dw|^2_S \leq C(1+\log (H/h))^2 |w|^2_S \]
zu beweisen, genügt es offenbar
\[ \boxed{\rho_i |\mathcal{H} (v^{(i)})|^2_{H^1(\Omega_i)} \leq C(1+\log(H/h))^2 \sum_{j \in I_i} \rho_j |\mathcal{H} (w^{(j)})|^2_{H^1(\Omega_j)}}\]
zu zeigen, wobei $I_i := \left\{ k\in \{1,\dots,N\}: \, \bar{\Omega}_i \cap \bar{\Omega}_k \neq \phi \right\}$.\\
\[(*) \, \begin{aligned}
v^{(i)} &= (P_Dw(x))^{(i)} = \sum_{j \in \mathcal{N}_x} \rho_j \mu^{\dagger}_j (w^{(i)}(x) - w^{(j)}(x)) \\
&= w^{(i)}(x) - \sum_{j \in \mathcal{N}_x}\rho_j \mu^{\dagger}_j w^{(j)}(x) = \left( (w-E_Dw)(x) \right)^{(i)}
\end{aligned} \]
(Dies folgt aus Lemmata 3.2.2 und 3.2.3)\\
Hierbei ist:
\[ mu^{\dagger}_j := \left\{ \begin{aligned} \mu^{-1}_j (x) &: \mu_j(x) \neq 0 \\ 0 &: \text{sonst} \end{aligned} \right, \quad \mu_j := \left\{ \begin{aligned} \sum_{j \in \mathcal{N}_x} \rho_j(x) &: x \in \partial\Omega_{i,h} \cap \partial \Omega_{j,h} \\ \rho_i (x) &: x \in \partial \Omega_{i,h}\cap (\partial\Omega_h\setminus \Gamma_h ) \\ 0 &: \text{ sonst} \end{aligned} \right, 
\]
Wir haben angewendet, dass $\sum_{i=1}^N \rho_i(x)\mu^{\dagger}_i(x) =1\, \forall \, x\in \Gamma_h \cup \partial\Omega_h$. \\
Es gilt: $\delta^{dagger}_i =\rho_i(x)\mu^{\dagger}_i(x) $ \\
(*) reduziert sich damit auf
\[ (*') :\quad v^{(i)} = \sum_{j \in \mathcal{N}_x} \delta^{\dagger}_j(x) (w^{(i)}(x)-w^{(j)}(x)) \quad x\in \partial \Omega_{i,h} \]
Part. der Eins $\{\Theta_{\epsilon^{1,j}},\dots , \Theta_{\epsilon^{?,j}}\}\, \Rightarrow  \, v^{(i)} = \sum_{\epsilon^{ij} \subset \partial\Omega_i} I^h (\Theta_{\epsilon^{ij}}v^{(i)})(x) + \sum_{\epsilon \in \partial\Omega_j} I^h (\Theta_{\epsilon^{ij}}v^{(i)})(x) $. 


%\input{Vorlesung15}
%\input{Vorlesung16}
%\input{Vorlesung17}
%\input{Vorlesung18}
%\input{Vorlesung19}
%\input{Vorlesung20}
%\input{Vorlesung21}
%\input{Vorlesung22}
%\input{Vorlesung23}
%\input{Vorlesung24}
%\input{Vorlesung25}
%\input{Vorlesung26}
%\input{Vorlesung27}
%\input{Vorlesung28}
%\input{Vorlesung29}

\printindex
\end{document}

