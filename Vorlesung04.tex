\begin{algorithmus}[Konjugiertes Gradientenverfahren (cg-Verfahren)]
  \begin{align*}
    \alpha^{(k)} := \frac{(p^{(k)},r^{(k)})}{(p^{(k)},Ap^{(k)})},\quad
    \beta^{(k)} := \frac{(p^{(k)},r^{(k+1)})_A}{(p^{(k)},p^{(k)})_A}
  \end{align*}
mit $ p^{(0)}:= r^{(0)}=b-Ax^{(0)}$.\\
Für $k=0,1,2,\ldots $:\\

  Berechne $\alpha^{(k)}$
  \begin{align*}
    x^{(k+1)}&:=x^{(k)}+\alpha^{(k)}p^{(k)} (*)\\
    r^{(k+1)}&:=r^{(k)} -\alpha^{(k)}Ap^{(k)} (**)
  \end{align*}
  Berechne $\beta^{(k)}$
  \[
    p^{(k+1)}:=r^{(k+1)}-\beta^{(k)}p^{(k)} (***)
  \]
  Es gilt auch (Übungen):
  \[
    \alpha^{(k)} = \frac{(r^{(k)},r^{(k)})}{(Ap^{(k)},p^{(k)})}, \beta^{(k)}=\frac{(r^{(k+1)},r^{(k+1)})}{(r^{(k)},r^{(k)})}
  \]
\end{algorithmus}

\begin{satz}[1.3.3]
  Sei A s.p.d. Dann konvergiert das cg-Verfahren und es gilt folgende Fehlerabschätzung:
  \[
    \|e^{(k)}\|_A \leq 2 \left( \frac{\sqrt{\kappa}-1}{\sqrt{\kappa}+1} \right)^k \|e^{(0)}\|, \quad k=0,1,\ldots
  \]
  mit $\kappa = \kappa(A) = \frac{\lambda_{max}(A)}{\lambda_{min}(A)}$
\end{satz}

\begin{proof}
  Folgt den Argumenten von Quaternioni/Valle Numerical Mathematics for PDE, Springenr, S.48-50 und Quarteroni/Suer/Sater, Numerical Mathematics, S.154
  \begin{enumerate}[1)]
    \item (*)$x^{(k+1)}= x^{(k)} + \alpha^{(k)}p^{(k)} \underbrace{=}_{Argument} x^{(0)} + \sum_{i=0}^{k} \alpha^{(i)}p^{(i)}$

    \item Sei $x=A^{-1}b$ die exakte Lösung , dann folgt aus 1) 
      \[
        (Ap^{(j)}, x^{(k+1)}-x^{(0)}) = \sum_{i=0}^{k} \alpha^{(i)}\underbrace{(Ap^{(j)},p^{(i)})}_{=0, i\neq j} 
        = \alpha^{(j)}(Ap^{(j)},p^{(j)}) \underbrace{=}_{Def. \alpha^{(j)}} (r^{(j)},p^{(j)}) = (b-Ax^{(j)},p^{(j)})
        = (A(x-x^{(j)}),p^{(j)}) 
        = (x-x^{(0)},Ap^{(j)}) + \underbrace{(\underbrace{x^{(0)}-x^{(j)}}_{\underbrace{=}_{1)}-\sum_{i=0}^{j-1}\alpha^{(i)}p^{(i)}},Ap^{j})}_{=0, \text{ da } (p^{(i)},Ap^{(j)})=0, i=0,\ldots,j-1}
        = (x-x^{(0)},Ap^{(j)})
        \Leftrightarrow (x^{(k+1)}-x^{(0)},p^{(j)})_A = (x-x^{(0)},p^{(j)})_A
      \]

    \item \begin{lemma}
      \[span\{p^{(0)},\dots p^{(k)} \} = span \{ r^{(0)},Ar^{(0)},\dots A^kr^{(0)}\} =: \underbrace{\Kc_{k+1}(r^{(0)})}_{\text{Krylovraum}} \]
      Offensichtlich gilt: $span \{ p^{(0)},\dots p^{(k)}\} =span \{ r^{(0)},r^{(1)},\dots , r^{(k)}\}$\\
      \end{lemma}

      \begin{proof}
        3) mit vollst. Induktion;\\
        IA: da $p^{(0)}=r^{(0)}$ \\
        $'\subset '$: IV: $ span \{ p^{(0)},\dots , p^{(k)} \} = \Kc_{k+1}(r^{(0)}) $\\
        IS: IV $\Rightarrow Ap^{(k)} \in \Kc_{k+2}(r^{(0)})$\\
        Aus $(\ast \ast)$ also $r^{(k+1)}=r^{(k)}-\alpha^{(k)}Ap^{(k)} \in \Kc_{k+2}(r^{(0)})$ 
        
        \begin{align*}
          ^{(\ast \ast \ast)}\Rightarrow& p^{(k+1)}=r^{(k+1)}-\beta^{(k)}p^{(k)} \in \Kc_{k+2}(r^{(0)}) \\
          \Rightarrow & span \{ p^{(0)},\dots p^{(k+1)} \} \subset \Kc_{k+2}(r^{(0)})
        \end{align*}
        $'\supset '$\\
        IV $span \{p^{(0)},\dots p^{(k)}\} = \Kc_{k+1}(r^{(0)})$ \\
        IS: IV $\Rightarrow A^kr^{(0)} \in span \{ p^{(0)},\dots p^{(k)} \}$ \\
        \begin{align*}
          ^{(\ast \ast)} \Rightarrow & A^{k+1}r^{(0)} \in span \{ r^{(0)},\dots , r^{(k+1)} \} = span \{ p^{(0)},\dots p^{(k+1)} \} \\
          \Rightarrow & '\supset '
        \end{align*}
      \end{proof}

    \item $^{2) + 3)} \Rightarrow \, x^{(k+1)}-x^{(k)}$ ist die Orthogonalprojektion des Anfangsfehlers $ x-x^{(0)} =: e^{(0)}$ auf den Krylovraum $\Kc_{k+1}(r^{(0)}$.\\

    \item es gilt mit 4) und $e^{(k+1)}=x-x^{(k+1)}$:
        \begin{align*}
          ||e^{(k+1)}||A &= || (x-x^{(0)})-(x^{(k+1)} -x^{(0)} ||_A \\
          &= \min_{y \in \Kc_{k+1}(r^{(0)})} ||x-x^{(0)} - y ||_A = \min_{y \in \Kc_{k+1}(r^{(0)})} ||e^{(0)} - y ||_A 
        \end{align*}
        Da $r^{(0)} =A(x-x^{(0)})=Ae^{(0)}$  lässt sich ein bel. $y \in \Kc_{k+1}(r^{(0)})$ schreiben als:
        \[ y= \sum_{j=0}^k \tilde \gamma_j A^jr^{(0)} = \sum_{j=1}^{k+1}\gamma_j A^j e^{(0)} = \tilde p (A)e^{(0)}, \,  p \in \Pc_{k+1} \]
        Also gilt $\|e^{(k+1)}\|_A = \min_{p\in \Pc^*_{k+1}} \|p(A)e^{(0)}\|_A$, wobei $\Pc^*_{k+1}=\left\{ q\in \Pc_{k+1} | q(0)=1 \right\}$

    \item A ist s.p.d. $\Rightarrow A=Q^TDQ, Q$ orthogonale Matrix, $D=diag(\lambda_i), \lambda_i$ EV von $A$\\
      $\Rightarrow p(A) = Q^Tp(D)Q$\\
      $\Rightarrow \|e^{(k+1)}\|^2_A = \min_{p\in \Pc^*_{k+1}} \|p(A)e^{(0)}\|_A^2 
      =  \min_{p\in \Pc^*_{k+1}} \|Q^Tp(D)Qe^{(0)}\|_A^2 $\\
      $ \min_{p\in \Pc^*_{k+1}} \left( D\underbrace{QQ^T}_{=I}p(D)Qe^{(0)}\underbrace{QQ^T}_{=I} \underbrace{p(D)Qe^{(0)}}_{=z} \right)$\\
      $  \min_{p\in \Pc^*_{k+1}} (Dp(D)z,p(D)z) = \sum_{i=1}^{n}\lambda_i z_i^2(p(\lambda_i))^2$\\
      $\leq  \min_{p\in \Pc^*_{k+1}} \max_{\lambda_i \in \sigma(A)} (p(\lambda_i))^2 \underbrace{\sum_{i=1}^{n}\lambda_i z_i^2}_{=(PQe^{(0)},Qe^{(0)})=\|e^{(0)}\|_A^2}$
      $= \left(  \min_{p\in \Pc^*_{k+1}}  \max_{\lambda_i \in \sigma(A)} (p(\lambda_i))^2 \right) \|e^{(0)}\|_A^2$\\
      $\Leftrightarrow \frac{\|e^{(k+1)}\|_A}{\|e^{(0)}\|_A} 
      \leq  \min_{p\in \Pc^*_{k+1}}  \max_{\lambda_i \in \sigma(A)} |p(\lambda)|$

    \item Als nächstes versuchen wir ein Polynom $q\in \Pc^*_{k+1}$ zu konstruieren, s.d. $|q(\lambda)|$ möglichst klein ist für $\lambda\in [\lambda_{min}, \lambda_{max}]$. Dazu führen wir die \underline{Tschebyscheff-Polynome} ein (vgl. Martin Haute-Bourgois, Grundlagen d. Num. Mathematik und WissRech, Vieweg-Teubner, 3. Auflage S.284). 
      Definition  $T_k(x) := \cos(k arccos(x)), \quad k\geq 0, x\in [-1,1]$ Tschebyscheff-Polynome k-ter Ordnung.
  $T_0(x)=1, T_1(x)=x, t:=arccos(x)$
  Behauptung: $T_{k-1}(x) + T_{k+1}(x) = \cos((k-1)t) + \cos ((k+1)t)$
  Somit gilt die rekursive Darstellung
  \[ T_0(x)=1,\, T_1(x)=x,\, T_{k+1}(x)=2xT_k (x) - T_{k-1}(x),\, k=1,2,\dots \]
  $ \Rightarrow T_k \in \Pc_k$ \\

\item Es gilt auch die Darstellung
  \[ T_K (x) =\cosh (k\cdot arccosh (x))\text{ für } x \geq 1,\, T_0(x)=1,\, T_1(x)=x \]
  Für den Rest muss man die Rekursionsformel anwenden:\\
  Setze für $x \geq 1$: $t:=arccosh(x) = ln (x+\sqrt{x^2-1})$\\
  Dann gilt
  \[ T_k(x) = cosh (kt) = \frac{1}{2} (e^{kt} + e^{-kt})\geq \frac{1}{2} (e^t)^k = \frac{1}{2} (x+\sqrt{x^2-1})^k \]

\item Wähle nun $q \in \Pc^*_{k+1}$ wie folgt
  \[ q(\lambda) := \frac{T_{k+1}\left( \frac{2\lambda - \lambda_{min} - \lambda_{max}}{\lambda_{min}-\lambda_{max}} \right) }{T_{k+1} \left( \frac{\lambda_{min}+\lambda_{max}}{\lambda_{max} - \lambda_{min}}\right)} \Rightarrow q(0)=1, q\in\Pc_{k+1} \rightarrow q\in\Pc^*_{k+1} 
  \]

