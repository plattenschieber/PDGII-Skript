CHRISTOPH

Betrachte dazu:
BILD1
Es soll gelten: 
\begin{align}
  \underbrace{u''(x^{(i)}_k)}_{u_k^{(i)}} &= \underbrace{u^{(j)}_k}_{u_k^{(j)}} \quad \forall k=1,\cdots,5\\
  \Leftrightarrow u_k^{(i)} - u^{(j)}_k &= 0 \quad \forall k=1,\cdots,5
  \Leftrightarrow 

  \begin{pmatrix}
    u_1^{(i)}-u_1^{(j)}\\
    u_2^{(i)}-u_2^{(j)}\\
    \vdots\\
    u_5^{(i)}-u_5^{(j)}  
  \end{pmatrix}
  =
  \begin{pmatrix}
    0\\ \vdots 0
  \end{pmatrix}

  \begin{pmatrix}
    1& 0& 0& 0& 0& | -1& 0& 0& 0& 0&\\
    0& 1& 0& 0& 0& | 0& -1& 0& 0& 0&\\
    0& 0& 1& 0& 0& | 0& 0& -1& 0& 0&\\
    0& 0& 0& 1& 0& | 0& 0& 0& -1& 0&\\
    0& 0& 0& 0& 1& | 0& 0& 0& 0& -1&
  \end{pmatrix}
 % \underbrace{=:B^{(i)}    =: B^{(j)}}_{:=B=(B^{(i)},B^{(j)})}
  \begin{pmatrix}
    u^{(i)}_1\\
    \vdots\\
    u^{(i)}_5\\
    u^{(j)}_1\\
    \vdots\\
    u^{(j)}_5
  \end{pmatrix}
  = 

  \Leftrightarrow
  
  (B^{(i)} B^{(j)}) %mal
  \begin{pmatrix}
    u^{(i)}\\
    u^{(j)}
  \end{pmatrix}
  =
  \begin{pmatrix}
    0\\
    0
  \end{pmatrix}

  \Leftrightarrow 

  Bu=0
\end{align}

Verallgemeinern wir dies auf alle Interfacesknoten, so ist folgende Notation sinnvoll. \\
Sei $u_\gamma$ der Knotenvektor zu den Interfacesknoten (bei entsprechender Nummerierung) dann lässt sich die Stetigkeitsnebenbedigung schreiben als 
\begin{align}
  0 = Bu = (\underbrace{0}_{convex Knoten?})
  \begin{pmatrix}
    u_I\\
    
  \end{pmatrix}<
  \label{}
\end{align}

FEHLT TAFEL VERSCHOBEN

CHRISTOPH


