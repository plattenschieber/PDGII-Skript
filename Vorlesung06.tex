\begin{satz}{2.2.3}\\
Es existiert eine Konstante $c>0$, unabhängig von $h$, s.d. 
\[\kappa (M) \leq c \frac{h^d}{(\min_{T \in \tau_h} h_T)^d} \]
wobei $M=(\varphi_i,\varphi_j)_{i,j}$ die Massenmatrix, $(\varphi_i)_i$ die nodale Basis und $\kappa (M)$ die spektrale Konditionszahl sei.
\end{satz}
\begin{proof}
\[\lambda_{max}=\max_{x  \in \mathbb{R}^n\setminus \{ 0 \}} \frac{x^TMx}{x^Tx} , \, \lambda_{min}=\min_{x  \in \mathbb{R}^n\setminus \{ 0 \}} \frac{x^TMx}{x^Tx} \]
$ \frac{x^TMx}{x^Tx}=\frac{x^TMx}{\norm{v}^2_{0,h}}\frac{\norm{v}^2_{0,h}}{x^Tx}$, wobei $v=\sum_{i=1}^n x_i\varphi_i \in V^h$ und es gilt:
\[x^TMx=(v,v)_{L^2(\Omega)}\, ^{Satz 2.2.1,i)} \Rightarrow\, (*) \, c_1 \leq \frac{x^TMx}{\norm{v}^2_{0,h}} \leq c_2 \, \forall x \in \mathbb{R}^n \]
Weiterhin ergibt sich für $x=(v(a_1),\dots,v(a_n))^T$:
\begin{align*}
(**) \, \min_{T \in \tau_h} h^d_T \norm{x}^2_{l_2} &= \min_{T \in \tau_h} h^2_T \sum_{i=1}^n (v(a_i))^2 \\
&= \sum_{T \in \tau_h} h^2_T \sum_{i=1}^n (v(a_i))^2 = \norm{v}^2_{0,h} \\
&=  \sum_{T \in \tau_h} h^2_T \sum_{i=1}^n (v(a_i))^2 \leq c_A h^d \norm{x}^2_{l_2}
\end{align*}
mit $h:= \max_{T \in \tau_h} |h_T| $.
\begin{align*}
^{(*),(**)}\Rightarrow& c_A \min_{T \in \tau_h} h^d_T \leq \frac{x^TMx}{x^Tx} \leq c_2c_1 h^d\\
\Rightarrow & \lambda_{min}(M) \geq c_A \min_{T \in \tau_h} h^d_T, \, \lambda_{max} (M) \leq c_2c_1 h^d \\
\Rightarrow & Beh. 
\end{align*}
\end{proof}
\underline{Bemerkung:} Für eine Familie quasi-uniformer Triangulierungen $(\tau_h)_h$ gilt nach Def.:\\
Es existiert eine Konstante $c>0$, s.d. 
\[ h:= \max_{T \in \tau_h} h_T \leq c h_T \, \forall  T \in \tau_h \, \Rightarrow \kappa (M) \leq \tilde C \]
$\rightarrow$ Für quasi-uniforme Triangulierungen ist die Massenmatrix gut konditioniert bzw. gleichmäßig beschränkt.\\

\begin{satz}%2.2.4
  Sei $K$ die Steifigkeitsmatrix und $M$ die Masseenmatrix. Dann existiert eine von $h$ unabhängige Konstante $c>0$, so dass für die spektrale Konditionszahl folgendes gilt:
  \begin{enumerate}
    \item $\kappa(M^{-1}K) \leq c(\min_{T\in\tau_h} h_T)^{-2}$
    \item $\kappa(k) \leq c(\min_{T\in\tau_h} h_T)^{-2} \kappa(M)$
  \end{enumerate}
\end{satz}

\begin{proof}[``Spielen mit dem Rayleigh Quotienten'']
  Für $x\in\R^n$ gilt
  \[
    \frac{x^TKx}{x^Tx} = \underbrace{\frac{x^TKx}{x^TMx}}_{\underset{y=M^{1/2}x}{=}\frac{y^TM^{-1/2}KM^{-1/2}y}{y^Ty}} \frac{x^TMx}{x^Tx}
  \]
  Es genügt somit zu Zeigen: $\exists c_1, c_2 >0:$
  \[
    c1 \leq \frac{x^TKx}{x^TMx} \leq c_2
  \]
  mit $K=(a(\varphi_i, \varphi_j)_{i,j})$ wie gehabt. 

  Es gilt für $x=(v(a_1),\ldots, v(a_n))^T$:
  \[
    \frac{x^TKx}{x^TMx} = \frac{a(v,v)}{(v,v)_{L^2(\Omega)}}
  \]

  \underline{V-Elliptisch:} $\forall v\in V^h: a(v,v) \geq \alpha \norm{v}^2_{H^1(\Omega)} \geq \alpha \norm{^2_{L^2(\Omega)}}$
  Aus der \underline{Stetigkeit} folgt: $\forall v\in V^h: a(v,v) \leq C \norm{v}^2_{H^1(\Omega)} \underset{\text{inv. Ungl.}{\leq}} c(\min_{T\in\tau_h} h_T)^{-2} \norm{v}^2_{L^2(\Omega)}$
  \[
    \Rightarrow \alpha \leq \frac{a(v,v)}{(v,v)_{L^2(\Omega)}} = \frac{x^TKx}{x^TMx} = \frac{a(v,v)}{(v,v)_{L^2(\Omega)}} \leq c(\min_{T\in\tau_h}h_T)^{-2}
  \]
  $\Rightarrow$ Behauptung
\end{proof}

\underline{Bemerkungen:}
\begin{enumerate}
\item
Für die Konditionszahl der Steifigkeitsmatrix einer regulären Familie von Triangulierungen gilt:
\[ \kappa (K)  \leq C \frac{h^d}{(\min_{t \in \tau_h} h_T)^{d+2}} \]
Für quasi-uniforme Triangulierungen gilt:
\[\kappa (K) \leq C \frac{1}{h^2} \]
\item
Das (nicht vorkonditionierte) CG-Verfahren konvergiert als mit einer Konvergenzrate von $\mathcal{O}(\sqrt{\kappa})=\mathcal{O}(h^{-1})$. \\
$\Rightarrow$ zunehmend schlechtere Konvergenz bei Gitterverfeinerung \\
$\Rightarrow$ Konstruktion effizienter Vorkonditionierer!
\end{enumerate}