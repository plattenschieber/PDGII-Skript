Damit haben das zu lösende System
\[ \boxed{P^TF\lambda =P^T d,\, G^T\lambda =e} \quad (3.5)  \]

nehmen wir an, dass (3.5) lösbar ist. Dann erfüllt $\lambda$ mit 
\[ \alpha := (G^TQG)^{-1} G^TQ (F\lambda -d) \]
die Gleichung (3.4):
\[ F\lambda = d+G\alpha \]
denn
\[ F\lambda = \d+G\alpha = d + \underbrace{G  (G^TQG)^{-1} G^TQ}_{=: (I-P^T)} (F\lambda -d) = d + (I-P^T)(F\lambda -d) \]
\[ \Leftrightarrow 0= -P^TF\lambda +P^Td \]
Hieraus berechnet man 
\[ u=K^+ (f-B^T\lambda) + R(G^T Q G)^{-1} G^TQ (F\lambda -d) \]
Weiterhin betrachten wir folgende Räume:
\[V:= \{ \lambda \in U :=\text{range} (B) \, :\,  \langle \lambda, Bz \rangle =0 \, \forall \, z \in \text{ker}(K) \} = \text{ker} (G^T) \]
\[V' := \{ \mu \in U \, :\,  \langle \mu, Bz \rangle_Q =0 \,  \forall \, z \in \text{ker}(K) \} = \text{ker} (P^T) \]
Das  FETI-Verfahren ist nun das vorkonditionierte CG-Verfahren in $V$ angewendet auf
\[ P^TF\lambda = P^T d \text{ für } \lambda \in \lambda_0 +V \]
mit einem Startvektor $\lambda_0: \, G^T\lambda_0=e $.\\

Es gilt tatsächlich:
\begin{align*}
  &	\lambda \in \lambda_0 + V \\
  \Leftrightarrow & \lambda=\lambda_0 + \lambda_1 \text{ mit } \lambda_1 \in V=\text{ker}(G^T) \\
  \Rightarrow & G^T\lambda = \underbrace{G^T \lambda_0}_{=e} + \underbrace{G^T \lambda_1}_{=0} =e
\end{align*}
Dafür muss über den Vorkonditionierer noch gewährleistet bleiben, dass die Iterierten in $V$ bleiben:\\
Sei dazu $PM^{-1}$ der noch zu wählende Vorkonditionierer, dann gilt es noch
\[ PM^{-1}P^TF\lambda = PM^{-1} P^T d \, \lambda \in \lambda_0 + V \]
mit PCG zu löden, wobei $G^T\lambda_0 = e$.\\
\underline{Bemerkung:}\\
Für $\lambda \in V=\text{range} (P)$:
\begin{align*}
PM^{-1}P^TF\lambda = PM^{-1}P^T FP \lambda \underbrace{=}_{(P^T)^2=P^T} (PM^{-1}P^T)(P^TFP)\lambda
\end{align*}
Daher kann man formal $PM^{-1}P^Tf$ auch als Produkt der zwei sym. Matrizen $( (PM^{-1}P^T)$ und $(P^TFP)$ ansehen.\\
$\rightarrow$\underline{Später:} Diese sind auch positiv definit!\\
\\

\underline{\textbf{Ein einfacher Vorkonditionierer:}}\\
Unterteilen wir die Freiheitsgrade in jedem Teilgebiet $\Omega_i$ in Innere, $u^{(i)}_I$, und solche auf dem Interface, $u^{(i)}_\Gamma$, dann ergibt sich folgende Partitionierung:
\[K^{(i)}=
  \begin{pmatrix} 
    K^{(i)}_{II} & K^{(i)}_{I\Gamma} \\
    K^{(i)}_{\Gamma I} & K^{(i)}_{\Gamma \Gamma} 
  \end{pmatrix} 
 \longrightarrow^{Block-GE} 
 \begin{pmatrix} 
   K^{(i)}_{II} & K^{(i)}_{I\Gamma} \\
   0 & \underbrace{(K^{(i)}_{\Gamma \Gamma}-K^{(i)}_{\Gamma I} (K^{(i)}_{II})^{-1} K^{(i)}_{I \Gamma })}_{=: S^{(i)}_{\Gamma \Gamma}=: S^{(i)}}
\end{pmatrix} 
\]
\begin{definition}
  $S:= \text{diag} (S^{(i)}) $. (Die $S^{(i)}$ von $i=1$ bis $i=N$ liegen auf der Diagonalen). Dann lautet der einfachste (Dirichlet)-Vorkonditionierer, den Farheit, Mandel und Roux (1994) vorschlagen, wie folgt:
  \[M^{-1} := B_\Gamma S B^T_\Gamma = \sum_{i=1}^N B^{(i)}_\Gamma S^{(i)} {B^{(i)}_\Gamma}^T \text{ wobei } B=(0 \, B_\Gamma ) \]
\end{definition}

\begin{algorithmus}
\underline{Initialisierung:}\\
\begin{align*}
  \lambda^{(0)} &:= QG(G^TQG)^{-1} e + \mu \\
  r^{(0)} &:= d- F \lambda^{(0)}
\end{align*}
\underline{Iteration:} $k=1,2,...$ bis Konv. : \\
\begin{align*}
  \text{\underline{Projektion:}} & q^{(k-1)} := P^T r^{(k-1)} \\
  \text{\underline{Vorkondition:}} & z^{(k-1)} := M^{-1} q^{(k-1)} \\
  \text{\underline{Projektion:}} & y^{(k-1)} := P z^{(k-1)}
\end{align*}
\begin{align*}
  \beta^{(k)} &= \frac{\langle y^{(k-1)} , q^{(k-1)} \rangle}{\langle y^{(k-2)} , q^{(k-2)} \rangle}, \, (\beta^{(1)}:=0 ) \\
  p^{(k)} &= y^{(k-1)} + \beta^{(k)}p^{(k-1)} , \, (p^{(1)}:=y^{(0)}) \\
  \alpha^{(k)} &:= \frac{\langle y^{(k-1)} , q ^{(k-1)} \rangle}{ \langle p^{(k)} , F p^{(k)} \rangle } \\
  \lambda^{(k)} &:= \lambda^{(k-1)} + \alpha^{(k)}p^{(k)} \\
  r^{(k)} &:= r^{(k-1)} - \alpha^{(k)}Fp^{(k)}
\end{align*}
\end{algorithmus}
Folgende Fragen:
\begin{itemize}
  \item
    Wie wählt man $M^{-1}$?
  \item
    Wie wählt man $B$?\\
    (redundant vs. nicht-redund. L.M.)
  \item
    Konditionsabhängigkeit bzw. Konvergenzanalyse?
\end{itemize}

\subsection{Nicht-redundante Lagrangesche Multiplikatoren}

Die Wahl des Sprungoperators $B$ ist nicht eindeutig! \\
Diese Fälle führen auf verschiedene Sprungoperatoren. Der nicht-redundante Fall (jeder Randpunkt wird nur mit einem anderen Randpunkt bzgl. Steitgkeit verglichen) führt auf ein $B$ mit vollem Rang und der redunate Fall (alle randkntoen werden entsprechend verglichen) nicht. \\
Vorteil des redundanten Falls: Man muss beim Programmieren keine spezielle Wahl der L.M. treffen. man minimiert einfach alle.\\
\textbf{Ab jetzt. nicht-redundante L.M. }\\

Wir betrachten folgenden Vorkonditionierer:
\[\hat M^{-1} := (B_\Gamma D^{-1} B^T_\Gamma )^{-1} B_\Gamma D^{-1} S D^{-1} B^T_\Gamma (B_\Gamma D^{-1} B^T_\Gamma)^{-1} \]
(Klawonn/Wildkunst, 2001). $D$ ist Diagonalmatrix mit pos. Diagonalelementen.\\
$\Rightarrow \, (B_\Gamma D^{-1}B^T_\Gamma)^{-1} $ lässt sich numerisch günstig berechnen.\\
Um ein Verfahren zu bekommen, welches möglichst unabh. von Sprüngen in den Koeff. der DGL konvergiert, def. wir $D$ wie folgt:
\[ D= \text{diag}\left( D^{(1)},\dots D^{(N)} \right) \]
wobei die $D^{(i)}$ auf den Knoten von $\partial \Omega_i$ operieren, d.h. jeder Eintrag gehört zu einem Knoten $x \in \partial \Omega_{i,h}$.\\
Dieser wird def. als:
\[ \delta^+_i := \delta^+_i (x) := \rho^\gamma_i(x)\mu^+_i(x), \quad \gamma \in [1/2,\infty) \]
(normalerweise $\gamma=1$), $\mu^+_i(x) :=\frac{1} { \sum_{j \in N_x} \rho^\gamma_j (x)} ,\, N_x := \{ j \in \{ 1,\dots N\} : x \in \Omega_{j,h} \}$.\\
\underline{Beispiele}\\
\[ (zeichnung 2 knoten)  N_x=\{i,j\} , \delta^+_i(x)=\frac{\rho^\gamma_i(x) }{\rho^\gamma_i(x)  + \rho^\gamma_j(x) } \]
\[ (Zeichnugn 4 Knoten) N_x=\{i,j,k,l \}, \,  \delta^+_i(x)=\frac{\rho^\gamma_i(x) }{\rho^\gamma_i(x)  + \rho^\gamma_j(x) + \rho^\gamma_k(x)  + \rho^\gamma_l(x) } \]

Dann benötigen wir noch folgenden Operator:
\[ P_D := D^{-1}B^T_\Gamma (B_\Gamma D^{-1} B^T_\Gamma)^{-1} B_\Gamma \]
$P_D$ ist die orth. Projektion bzgl. $(x,y)=y^TDx$ und es gilt:
\[ B_\Gamma P_D = B_\Gamma \text{ (sprungerhaltend) } \]




