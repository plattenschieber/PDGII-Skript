\documentclass[12pt]{scrbook}

\usepackage{geometry}
\geometry{a4paper,left=2.5cm,right=2.5cm, top=2cm, bottom=3cm, marginpar=2cm} 

%% Sonderzeichen
\usepackage[ngerman]{babel}
\usepackage[utf8]{inputenc}
\usepackage[T1]{fontenc}
\usepackage{amsmath,amssymb,amsfonts,stmaryrd,mathtools,amsthm,esint}
\usepackage{algpseudocode}
\usepackage{dsfont} % bb für Zahlen
%\usepackage{MnSymbol} %adobe minion font

%% minted
%\usepackage{minted}
%\newminted{cpp}{linenos}

%\usepackage{marginnote}
%\renewcommand*{\marginfont}{\footnotesize} 

\usepackage{graphicx}
%\graphicspath{{Bilder/}}
%\usepackage{tikz}
%\usepackage[all]{xy}
%\usetikzlibrary{arrows,calc,shadows,patterns,through,backgrounds}
%\pgfdeclarepatternformonly{mynwlines}{%
%\pgfqpoint{-1pt}{-1pt}}{\pgfqpoint{8pt}{8pt}}{\pgfqpoint{6pt}{6pt}}%
%{
%  \pgfsetlinewidth{0.4pt}
%  \pgfpathmoveto{\pgfqpoint{0pt}{3pt}}
%  \pgfpathlineto{\pgfqpoint{6.1pt}{-3.1pt}}
%  \pgfusepath{stroke}
%}

\usepackage{hyperref}
\usepackage{makeidx}
\usepackage{listings}
\usepackage{enumerate}

%% Counters
%\renewcommand\theequation{\thesection.\arabic{equation}}
%\AtBeginSection[]{ \setcounter{section}{0} }
\numberwithin{equation}{section}


\newcommand{\N}{\mathbb{N}}
\newcommand{\Z}{\mathbb{Z}}
\newcommand{\R}{\mathbb{R}}
\newcommand{\C}{\mathbb{C}}
\newcommand{\F}{\mathbb{F}}
\newcommand\K{\mathbb{K}}
\renewcommand{\c}[1]{\mathcal{#1}} % beliebiger Buchstabe kann mit \op in cal geschrieben werden 
\renewcommand\P{\mathcal{P}} % tolles P, anstatt eines Zeilenumbruchzeichens 
\newcommand\Ac{\mathcal{A}}
\newcommand\Pc{\mathcal{P}}
\newcommand\Lc{\mathcal{L}}
\newcommand\Kc{\mathcal{K}}
\newcommand\Mc{\mathcal{M}}
\newcommand\Tc{\mathcal{T}}
\newcommand\LC{\mathcal{L}}
\newcommand\ii{\mathrm{i}} % schlichtes i, ungeschwungen
\newcommand{\ud}{\,\textnormal{d}} % aufrechtes d fürs Integral
\newcommand{\floor}[1]{\left\lfloor #1 \right\rfloor}
\newcommand{\ceil}[1]{\left\lceil #1 \right\rceil}
%\def\default{}
\renewcommand{\emph}[2][\default]{\ifx#1\default\textbf{#2}\index{#2}\else\textbf{#2}\index{#1}\fi}
%\newcommand{\ocirc}{\, \raisebox{1pt}{\footnotesize\textcircled{$\circ $}}\,}
%\newcommand{\ostar}{\, \raisebox{1pt}{\footnotesize\textcircled{$*$}}\,}

\newcommand{\bs}[1]{\boldsymbol{#1}} % fett gedruckt
\newcommand{\entspr}{\mathop{\widehat{=}}} % entspricht Zeichen
\newcommand{\eps}{\varepsilon} % schönes epsilon 
\def\pre{\textnormal{pre}}
\def\suc{\textnormal{suc}}

% coole Abkürzungen für die Norm und Doppelnorm
\newcommand{\norm}[1]{\|#1\|}
\newcommand{\znorm}[1]{|\!|\!|#1|\!|\!|}

% ein paar Befehle in Normalschrift, statt kursiv
\newcommand\dist{\textnormal{dist}}
\newcommand\im{\textnormal{Im}\,}
\newcommand\re{\textnormal{Re}\,}
\newcommand\ran{\textnormal{Ran}\,}
\renewcommand\ker{\textnormal{Ker}\,}
\newcommand\range{\textnormal{range}\,}
\newcommand\var{\textnormal{Var}}
\newcommand\cov{\textnormal{Cov}}
\newcommand\ggt{\textnormal{ggT}}
\newcommand\periode{\textnormal{Periode}}
\newcommand{\trace}{\textnormal{trace}\,}
\newcommand{\sign}{\textnormal{sign}\,}
\newcommand{\cond}{\textnormal{cond}}
\newcommand{\spanf}{\textnormal{span}\,}
\newcommand{\spann}{\textnormal{span}}
\newcommand{\grad}{\textnormal{grad}\,}
\newcommand{\sgn}{\textnormal{sgn}}
\newcommand{\rd}{\textnormal{rd}}
\newcommand{\diag}{\textnormal{diag}}
\newcommand{\blockdiag}{\textnormal{blockdiag}}
\newcommand{\supp}{\textnormal{supp}}
\newcommand{\fie}{\varphi}
\newcommand{\eins}{\mathds{1}}
\newcommand{\diam}{\mathrm{diam}}
\newcommand{\const}{\mathrm{const}}
\newcommand{\divt}{\mathrm{div}}
\newcommand{\vol}{\mathrm{vol}}
\newcommand{\inn}{\textnormal{ in }}
\newcommand{\auf}{\textnormal{ auf }}
\renewcommand{\i}{\textnormal{i}}
\def\dtilde{\stackrel{\approx}}
\def\dabs{\phantom{a}\quad}


\newcommand{\tn}{\ensuremath{| \! | \! |}}
\newcommand{\operateson}{\rcirclearrowright}
\usepackage{framed,color}                          % Farbe
\setlength{\fboxsep}{0.2cm}
\setlength{\fboxrule}{1pt}
\definecolor{shadecolor}{rgb}{0.91,0.91,1}      % fuer shaded-Umgebungen

\newtheoremstyle{note}% name
  {1ex}  % Space above
  {1ex}  % Space below
  {\sl}  % Body font
  {}     % Indent amount (empty = no indent, \parindent = para indent)
  {\bfseries}  % Thm head font
  {.}    % Punctuation after thm head
  {.5em} % Space after thm head: " " = normal interword space;
         % \newline = linebreak
  {\thmname{#1}\thmnumber{ #2}\bf\thmnote{(#3)}}     % Thm head spec (can be left empty, meaning `normal')

\newtheoremstyle{remark}% name
  {1ex}  % Space above
  {1ex}  % Space below
  {}     % Body font
  {}     % Indent amount (empty = no indent, \parindent = para indent)
  {\bfseries}  % Thm head font
  {.}    % Punctuation after thm head
  {.5em} % Space after thm head: " " = normal interword space;
         % \newline = linebreak
  {\thmname{#1}\thmnumber{ #2}\bf\thmnote{(#3)}}     % Thm head spec (can be left empty, meaning `normal')

\theoremstyle{note}
\newtheorem{amssatz}{Satz}[section]
\newtheorem{amslemma}[amssatz]{Lemma}
\newtheorem{amsproblem}[amssatz]{Problem}
\newtheorem{amsdefi}[amssatz]{Definition}
\theoremstyle{remark}
\newtheorem{amsbsp}[amssatz]{Beispiel}
\newtheorem*{amsbsp*}{Beispiel}
\newtheorem{amsalgo}[amssatz]{Algorithmus}
\newtheorem*{amsalgo*}{Algorithmus}
\newtheorem{amskorollar}[amssatz]{Korollar}
\newtheorem*{bemerkung*}{Bemerkung}
\newtheorem{bemerkung}[amssatz]{Bemerkung}

\newenvironment{satz}[1][]{\begin{amssatz}[#1]\begin{shaded}}
                          {\end{shaded}\end{amssatz}}
\newenvironment{lemma}[1][]{\begin{amslemma}[#1]\begin{shaded}}
                           {\end{shaded}\end{amslemma}}
\newenvironment{problem}[1][]{\begin{amsproblem}[#1]\begin{shaded}}
                           {\end{shaded}\end{amsproblem}}
\newenvironment{korollar}[1][]{\begin{amskorollar}[#1]\begin{shaded}}
                           {\end{shaded}\end{amskorollar}}
\newenvironment{algorithmus}[1][]{\begin{amsalgo}[#1]\begin{shaded}}
                                 {\end{shaded}\end{amsalgo}}
\newenvironment{algorithmus*}[1][]{\begin{amsalgo*}[#1]\begin{shaded}}
                                 {\end{shaded}\end{amsalgo*}}
\newenvironment{definition}[1][]{\begin{amsdefi}[#1]\begin{shaded}}
                                {\end{shaded}\end{amsdefi}}
\newenvironment{beispiel}[1][]{\begin{amsbsp}[#1]\begin{framed}}
                               {\end{framed}\end{amsbsp}}
\newenvironment{beispiel*}[1][]{\begin{amsbsp*}[#1]}
                               {\end{amsbsp*}}



